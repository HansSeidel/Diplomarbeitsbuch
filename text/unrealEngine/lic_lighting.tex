\section{Lighting}
Die Unterkapitel des Lighting, bis auf das Vorwissen, beziehen sich auf die Beleuchtung in der Unreal Engine.

\subsection{Vorwissen}
Eine uns bekannte und Beleuchtungsmethode ist die 3\verb+-+Punkt\verb+-+Beleuchtung. Bei ihr wird klassischer weiße ein Objekt
von 3 Seiten mittels Key-, Fill- und Backlight ausgeleuchtet. Dadurch entsteht eine Tiefe und eine Stimmung die je nach Lichteinfall und
Lichtstärke variieren kann.

Es gibt verschiedene Lampen, um Personen oder Objekte auszuleichten. Je konzentrierter das Licht von einer Lampe wegstrahlt und auf ein Objekt
fällt, desto härter ist der Schatten den es wirft. Wenn eine Lampe weiter weg ist, kommt weniger Licht am Objekt an.

\subsection{Directional Light}
Das Directional Light simuliert ein Licht was unendlicht weit weg ist. Somit kommt das Licht nur von einer Seite, weshalb
man es gut als Sonne verwenden. \citep{unreal:directional_light}

\subsection{Sky Light}
Das Sky Light strahlt Licht von allen Seiten aus. Man kann somit die Farbe und Lichtstärke auf der Rückseite von Objekten kontrollieren.

\subsection{Sky Sphere}
Die Sky Sphere gibt definiert den Himmel in Unreal Engine. Mit ihr, kann man Wolken und Sterne bearbeiten. Manche Einstellungen funktionieren nur in Relation mit anederen
Lichtquellen. Man kann zum Beispiel nur Sterne sehen, wenn man die Sonne in einem bestimmten Winkel platziert.
Außerdem kann man mehrere Farbeinstellungen für den Himmel vornehmen.

\subsection{Post Process Volume}
Mit der Post Process Volume kann man das Aussehen im Spiel nachbearbeiten. \citep{unreal:postProcessVolume} Wir haben sie dazu benutzt um die Helligkeit im Spiel auf einen gewissen Bereich zu beschränken.

\subsection{Exponential Height Fog}
Mit dem Exponential Height Fog kann man Nebel über die ganze Welt erzeugen. Dieser Nebel hat unten eine höhere Dichte als weiter oben. Weiters kann man zwei Farben einstellen.
Die von der Seite wo das Sonnenlicht kommt und die von der Gegenüberliegenden Seite, das heißt der Nebel hat je nach Seitenansicht eine andere Farbe. \citep{unreal:exponentialHeightFog}

\subsection{Lightmass Importance Volume}
Mit der Lightmass Importance Volume, kann man Bereiche im Spiel einstellen, wo das Licht genau berechnet werden soll. Das ist besonders wichtig, da der Spieler nur in einem
Bestimmten Bereich Licht mit guter Qualität sehen kann und somit Rechenaufwand für Bereiche in denen es nicht so ist, gespart wird. \citep{unreal:lightmassImportanceVolume}

\subsection{Light Mobility}
\subsubsection{Static}
Beim Static Light wird das Licht direkt am Anfang berechnet. Das heißt man kann es wärend dem Spiel nicht ändern, dafür wird die Performance verbessert, da
die Schatten nicht immer neu berechnet werden müssen. \citep{unreal:types_of_light}

\subsubsection{Stationary}
Das Stationary Light kann man während dem Spielen nicht bewegen, allerdings kann man die Farbe und die Intensität verändern. Somit stellt es vom Rechenaufwand und
der Funktion einen Kompromiss zwischen Static und Movable Light da. \citep{unreal:types_of_light}

\subsubsection{Movable}
Das Movable Light kann man während dem Spielen vollständig verändern. Dadurch dass man seine Position verändern kann und die Schatten dadurch neu berechnet werden müssen,
ist es das Performancelastigste Light.

\subsection{Das Lighting im Spiel}
Das Ziel beim Lighting im Spiel war es, eine düstere Stimmung zu bekommen. Dazu mussten mehrere Objekte die das Licht und den Himmel bestimmen aufeinander abgestimmt werden.

Diese Objekte sind:
\begin{enumerate}
    \item Directional Light
    \item Sky Light
    \item Sky Sphere
    \item Post Process Volume
    \item Exponential Height Fog
    \item Lightmass Importance Volume
\end{enumerate}

Um die Farben abzustimmen und die gesamt Helligkeit zu regeln, muss man an drei Objekten einstellungen vornehmen.
\begin{enumerate}
    \item Directional Light
    \begin{enumerate}
              \item Das Licht fällt von einer Seite auf ein Objekt.
    \end{enumerate}
    \item Sky Light
    \begin{enumerate}
        \item Das Licht fällt von allen Seiten auf ein Objekt.
    \end{enumerate}
    \item Sky Sphere
    \begin{enumerate}
        \item Bestimmt die Farbe des Himmels in relation zu dem Directional Light und dem Sky Light.
    \end{enumerate}
\end{enumerate}

Um das ganze zu veranschaulichen, wurden die Werte etwas verändert und in die Lighting Only Ansicht der Unreal Engine gewechselt.
In Abbildung \ref{lighting:Lighting-Sky_Directional} kann man nun sehen,
dass von der Sonnenseite ein helles lilanes Licht (das Licht des Directional Lights), und von der Schattenseite ein eher dunkles, blaues Licht
(das Licht des Sky Lights) kommt.
Die Farbe des Himmels ist violett. Dies wurde durch die Sky Sphere bestimmt.
\begin{figure}[h]
    \centering
    \includegraphics[width=.8\textwidth]{Lighting-Sky_Directional.png}
    \caption{Objekte in der Lighting Only Ansicht von Unreal Engine}
    \label{lighting:Lighting-Sky_Directional}
\end{figure}

Um eine Nachtatmossphere zu bekommen, wurde die Sonne auf der y\verb+-+Achse auf 90$^\circ$ gesetzt. Dadurch kann man nun Sterne sehen,
welche in der Sky Sphere heller oder dünkler stellen kann.

Um ein die Umgebungshelligkeit anzupassen, wurde in der Post Process Volume, die Helligkeit auf einen bestimmten Wert eingegrenzt.

Damit das Spiel eine düstere Stimmung bekommt, wurde noch Nebel hinzugefügt. Die effizienteste Methode war, einen Exponential Height Fog einzubauen, welcher
Nebel im ganzen Spiel erzeugt. Dieser wurde relativ Dicht eingestellt und hat eine zum Himmel passende Farbe bekommen. Zusätzlich wurden in der Sky Sphere noch die Wolken
und Sterne angepasst, damit man sie noch gut durch den Nebel sehen kann.

Damit die Beleuchtung nicht zu viel Rechenaufwand in anspruch nimmt, wurde noch eine Lightmass Importance Volume hinzugefügt, welche den Bereich eingrenzt,
indem das Licht genau berechnet wird. Dieser Bereich liegt über der Spielwelt.