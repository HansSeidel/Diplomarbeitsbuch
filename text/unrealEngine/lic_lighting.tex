\section{Lighting}
Die Unterkapitel des Lighting, bis auf das Vorwissen, beziehen sich auf die Beleuchtung in der Unreal Engine.

\subsection{Vorwissen}
Eine uns bekannte und Beleuchtungsmethode ist die 3\verb+-+Punkt\verb+-+Beleuchtung. Bei ihr wird klassischer weiße ein Objekt
von 3 Seiten mittels Key-, Fill- und Backlight ausgeleuchtet. Dadurch entsteht eine Tiefe und eine Stimmung die je nach Lichteinfall und
Lichtstärke variieren kann.

Es gibt verschiedene Lampen, um Personen oder Objekte auszuleichten. Je konzentrierter das Licht von einer Lampe wegstrahlt und auf ein Objekt
fällt, desto härter ist der Schatten den es wirft. Wenn eine Lampe weiter weg ist, kommt weniger Licht am Objekt an.

\subsection{Directional Light}
\citep{unreal:directional_light} Das Directional Light simuliert ein Licht was unendlicht weit weg ist. Deshalb kann man es gut als Sonne verwenden.\todo{Zitat - übersetzt}
\todo{Die Settings im Spiel für Licht dokumentieren}

\subsection{Sky Light}
\subsection{Exponential Height Fog}
\subsection{Light Mobility}
\subsubsection{Static}
Beim Static Light wird das Licht direkt am Anfang berechnet. Das heißt man kann es wärend dem Spiel nicht ändern, dafür wird die Performance verbessert, da
die Schatten nicht immer neu berechnet werden müssen. \todo{Mehr darüber - Quelle..?}
\todo{Wie genau muss die Quelle sein? reicht types of lighst?}
\citep{unreal:types_of_lights}

\subsubsection{Stationary}
Das Stationary Light kann man während dem Spielen nicht bewegen, allerdings kann man die Farbe und die Intensität verändern. Somit stellt es vom Rechenaufwand und
der Funktion einen Kompromiss zwischen Static und Movable Light da. \todo{Mehr darüber - Quelle..?}

\subsubsection{Movable}
\todo{Mehr darüber - Quelle..?}