\section{Lighting}
Die Unterkapitel des Lighting, bis auf das Vorwissen, beziehen sich auf die Beleuchtung in der Unreal Engine.

\subsection{Vorwissen}
Eine uns bekannte Beleuchtungsmethode ist die 3\verb+-+Punkt\verb+-+Beleuchtung. Bei ihr wird klassischer Weiße ein Objekt
von 3 Seiten mittels Key-, Fill- und Backlight ausgeleuchtet. Dadurch entsteht eine Tiefe und somit eine Stimmung die je nach Lichteinfall und
Lichtstärke variieren kann.

Weiters sind uns verschiedene Lichtarten bereits aus Blender bekannt, die es in der Unreal Engine auch gibt.

Dazu zählen:

\begin{enumerate}
    \item Point Light
    \item Sun (entspricht dem Directional Light in Unreal Engine)
    \item Spot Light
    \item Hemi (entspricht dem Skylight in Unreal Engine)
\end{enumerate}

Außerdem gibt es verschiedene Lampen, um Personen oder Objekte auszuleuchten. Je konzentrierter das Licht von einer Lampe ausstrahlt und auf ein Objekt
fällt, desto härter ist der Schatten den es wirft. Wenn eine Lampe weiter weg ist kommt weniger Licht am Objekt an.

\subsection{Directional Light}
Das Directional Light simuliert ein Licht welches unendlich weit weg ist. Somit strahlt das Licht nur aus einer Richtung, weshalb
man es gut als Sonne verwenden kann. \citep{unreal:directional_light}

\subsection{Sky Light}
Das Sky Light strahlt Licht von allen Seiten aus. Man kann somit die Farbe und Lichtstärke auf der Rückseite von Objekten kontrollieren.

\subsection{Sky Sphere}
Die Sky Sphere definiert die Struktur des Himmels in der Unreal Engine.
Mit ihr kann man Wolken und Sterne bearbeiten. Manche Einstellungen funktionieren nur in Relation mit anderen
Lichtquellen. Man kann zum Beispiel nur Sterne sehen, wenn man die Sonne in einem bestimmten Winkel platziert.
Außerdem kann man mehrere Farbeinstellungen für den Himmel vornehmen.

\subsection{Post Process Volume}
Mit der Post Process Volume kann man das Aussehen im Spiel nachbearbeiten. \citep{unreal:postProcessVolume}
Wir haben sie dazu benutzt um die Helligkeit im Spiel auf einen gewissen Bereich zu beschränken.

\subsection{Exponential Height Fog}
Mit dem Exponential Height Fog kann man Nebel über die ganze Welt erzeugen. Dieser Nebel hat unten eine höhere Dichte als weiter oben. Weiters kann man zwei Farben einstellen.
Die von der Seite wo das Sonnenlicht kommt und die von der gegenüberliegenden Seite, das heißt der Nebel hat je nach Seitenansicht eine andere Farbe. \citep{unreal:exponentialHeightFog}

\subsection{Lightmass Importance Volume}
Mit der Lightmass Importance Volume kann man Bereiche im Spiel einstellen, wo das Licht genau berechnet werden soll. In Bereichen in denen das
Licht genauer berechnet wird, prallt das Licht öfters von Oberflächen ab und wird pro abprallen (Bounce) schwächer und erzeugt somit genauere Schatten.
Das ist besonders wichtig, da der Spieler nur in einem bestimmten Bereich Licht mit guter Qualität sehen kann und somit Rechenaufwand für Bereiche,
in denen es nicht so ist, gespart wird. \citep{unreal:lightmassImportanceVolume}

\subsection{Point Light}
Das Point Light strahlt in alle Richtungen für eine begrenzte Reichweite, daher eignet es sich gut als Lampe.

\subsection{Spot Light}
Das Spotlight leuchtet in eine Richtung für eine begrenzte Reichweite. Es eignet sich ebenfalls gut als Lampe.

\subsection{Light Mobility}
\subsubsection{Static}
Beim Static Light wird das Licht direkt am Anfang berechnet und in einer Lightmap gespeichert. Das heißt man kann es während dem Spiel nicht ändern.
Dafür wird die Performance verbessert, da die Schatten nicht immer neu berechnet werden müssen. \citep{unreal:static_light}

\subsubsection{Stationary}
Das Stationary Light kann man während dem Spielen nicht bewegen, allerdings kann man die Farbe und die Intensität verändern. Es kann aber nur das direkte Licht, also kein
abprallendes Licht (Licht prallt normalerweise öfters ab und wird immer schwächer) verändert werden.
Somit stellt es vom Rechenaufwand und der Funktion einen Kompromiss zwischen Static und Movable Light da. \citep{unreal:stationary_light}

\subsubsection{Movable}
Das Movable Light wird während dem Spiel berechnet.
Dadurch kann man es während dem Spiel bewegen und vollständig ändern.
Es ist das Performance lastigste Licht. \citep{unreal:movable_light}

\subsection{Das Lighting im Spiel}
Das Ziel beim Lighting im Spiel war es, eine düstere Stimmung zu bekommen. Dazu mussten mehrere Objekte die das Licht und den Himmel bestimmen aufeinander abgestimmt werden.

Diese Objekte sind:
\begin{enumerate}
    \item Directional Light
    \item Sky Light
    \item Sky Sphere
    \item Post Process Volume
    \item Exponential Height Fog
    \item Lightmass Importance Volume
    \item Point Light
    \item Spot Light
\end{enumerate}

Um die Farben abzustimmen und die Gesamthelligkeit zu regeln, muss man an drei Objekten Einstellungen vornehmen.
\begin{enumerate}
    \item Directional Light
    \begin{enumerate}
        \item Das Licht fällt von einer Seite auf ein Objekt.
    \end{enumerate}
    \item Sky Light
    \begin{enumerate}
        \item Das Licht fällt von allen Seiten auf ein Objekt.
    \end{enumerate}
    \item Sky Sphere
    \begin{enumerate}
        \item Bestimmt die Farbe des Himmels in Relation zu dem Directional Light und dem Sky Light.
    \end{enumerate}
\end{enumerate}

Um das Ganze zu veranschaulichen, wurden die Werte etwas verändert und in die Lighting Only Ansicht der Unreal Engine gewechselt.
In \textit{Abbildung \ref{lighting:Lighting-Sky_Directional}} kann man nun sehen,
dass von der Sonnenseite ein helles lilianes Licht (das Licht des Directional Lights), und von der Schattenseite ein eher dunkles, blaues Licht
(das Licht des Sky Lights) kommt.
Die Farbe des Himmels ist violett. Dies wurde durch die Sky Sphere bestimmt.
\begin{figure}[h]
    \centering
    \includegraphics[width=.8\textwidth]{Lighting-Sky_Directional.png}
    \caption{Objekte in der Lighting Only Ansicht von Unreal Engine}
    \label{lighting:Lighting-Sky_Directional}
\end{figure}

Um eine Nachtatmossphäre zu bekommen wurde die Sonne auf der y\verb+-+Achse auf 90$^\circ$ gesetzt. Dadurch kann man nun Sterne sehen,
welche man in der Sky Sphere heller oder dunkler stellen kann.

Um die Umgebungshelligkeit anzupassen wurde in der Post Process Volume die Helligkeit auf einen bestimmten Wert eingegrenzt.

Damit das Spiel eine düstere Stimmung bekommt, wurde noch Nebel hinzugefügt. Die effizienteste Methode war, einen Exponential Height Fog zu verwenden, welcher
Nebel im ganzen Spiel erzeugt. Dieser wurde relativ dicht eingestellt und hat eine zum Himmel passende Farbe bekommen. Zusätzlich wurden in der Sky Sphere noch die Wolken
und Sterne angepasst, damit man sie noch gut durch den Nebel sehen kann.

Um eine passende Atmosphäre in den Gebäuden zu schaffen wurden je nach Lampenart, Spot Lights oder Point Lights verwendet.

Damit die Beleuchtung nicht zu viel Rechenaufwand in Anspruch nimmt, wurde noch eine Lightmass Importance Volume hinzugefügt, welche den Bereich eingrenzt
indem das Licht genau berechnet wird. Dieser Bereich liegt über der Spielwelt.

