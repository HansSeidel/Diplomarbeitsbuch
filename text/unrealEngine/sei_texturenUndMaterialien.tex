\section{Texturierung}
\label{sec:tex}
Texturen geben dem erstellten Objekt komplexe Farben. Um einen Bezug zu realen Objekten zu schaffen, müssen Texturen
auf 3Dimensionalen Objekten angewandt werden. Zum Beispiel kann man einen Baum bis
zum kleinsten Detail modellieren und mittels verschiedener Materialien einfärben. Die Natürlichkeit
eines echten Baums wird dieses Modell nicht erreichen.

Es gibt Spiele, welche die Texturanzahl sehr gering halten. In diesem Fall wird eher ein Comic
oder Low-Polygon Stil verfolgt.

Allerdings werden Texturen nicht nur für die Farbe verwendet, sondern auch für Materialeinstellungen.
Einige dieser Texturarten wurden für dieses Projekt verwendet und
sind in den nächsten Kapiteln zu finden.

%-----------------Mapping Methoden-----------------Mapping Methoden------------------Mapping Methoden-------------------
\subsection{Mapping Methoden}
\label{sec:tex_mapping}

Bei der Texturierung eines Objektes unterscheidet man zwischen verschiedenen Mapping Methoden. Mapping bezeichnet man
den Prozess, eine Textur aus dem 2-Dimensionalen Bereich auf ein Objekt im 3-Dimensionalen Bereich zu übertragen. Die
meist verwendeten Mapping Methoden sind das generierte Mapping und das UV-Mapping\citep{blender:tex_mapping_modes}:


Generiertes Mapping:

Generiertes Mapping bedeutet, dass man die Grundform eines komplexen Objektes angibt (Z.b. Cube / Zylinder ). Die
eingefügte Textur wird dann entsprechend der Grundform über das Objekt "gemappt". Die Angabe der Grundform beschreibt
wie die 2-Dimensionale Textur in der 3Dimensionalen Welt auf dem Objekt angezeigt werden soll.
Bei der generierten Texturierung kann es allerdings zu Fehlern kommen. Es kommt vor, dass die Textur von einer
bestimmten Seiten korrekt und auf anderen Seite langezogen angezeigt wird. Dass liegt daran, dass die Pixel einer
Textur (2Dimensionaler Bereich) auf eine 3Dimensionale Fläche projiziert werden, in welcher sie keine Informationen
über die dritte Achse besitzen (Siehe Abbildung \ref{picture:tex_streching}).

\begin{figure}[h]
    \centering
    \includegraphics[width=.9\textwidth]{SEI_Texturierung_Methoden-Pixelstretching.png}
    \caption{Die Pixel der Textur werden der Y-Achse entlang auf der oberen Fläche des Würfels gestreckt}
    \label{picture:tex_streching}
\end{figure}

Der einfachste Weg, diesen Fehler zu umgehen ist es, ein UV-Layout zu generieren. Allerdings ist es auch mit dem
Node-System der Unreal-Engine\citep{ue:node_introduction} möglich (Siehe Abbildung \ref{picture:tex_streching_solve}).


\begin{figure}[h]
    \centering
    \includegraphics[width=.9\textwidth]{SEI_Texturierung_Methoden-PixelstretchingMaterialKorrektur.png}
    \caption{Eine komlexere Lösung der Fehler-Behebung. Die einfache Methode ist UV-Mapping}
    \label{picture:tex_streching_solve}
\end{figure}


UV-Mapping:

UV-Mapping ist eine sehr sichere Art der Texturierung. Beim UV-Mapping wird eine Textur dem UV-Layout des Objektes
angepasst. Dies kann automatisch passieren oder bewusst beeinflusst. Bei komplexeren Objekten wird das UV-Layout
exportiert und eine Textur mittels eines Fotoprogramms oder dem Blender Texture-Painting Tool erstellt
(Mehr Informationen zu UV-Mapping und dem Blender Texture-Painting Tool sind in den Kapiteln \ref{sec:unwrapping} und
dem Kapitel \ref{sec:texture_painting})


Weiter aufbauende Mapping-Methoden die mittels dem Node-System der Unreal Engine möglich sind, sind die WorldAlign
Mapping Methode, View-Ported Mapping Methode und die Objektrelative Mapping Methode. Diese Arten wurden im Laufe
des Projektes nicht verwendet und sind daher nicht Teil dieses Buches.

Die nächsten Kapitel (Diffuse Texturen, Bump- Normal-Texturen, Ambient Occlusion Texturen, Roughness Texturen)
beschreiben verschiedene Textur-Arten, die am häufigsten bei der Spielentwicklung eingesetzt werden. Für die
Entwicklung und Bearbeitung dieser Texturen wurde das Programm "Materialize" - von Bounding Box
Software\citep{bbs:materialize} - verwendet. Dieses Programm ist eine OpenSource Entwicklung, mit freier Lizenz.


%-----------------Diffuse Texturen-----------------Diffuse Texturen------------------Diffuse Texturen-------------------
\subsection{Diffuse Texturen}
\label{sec:tex_diffuse}

Diffuse Texturen\citep{blender:tex_introduction} bestimmen das grundlegende Aussehen des Objektes. Sie beinhalten die Farbinformationen des
Objektes.
In vielen Fällen werden diffuse Texturen seamless gestaltet. Eine Textur ist seamless, wenn eine Kopie der Textur an
jede Seite gelegt werden kann und trotzdem einen perfekten Übergang bildet.

%----------Bump und Normal Texturen------Bump und Normal Texturen------------Bump und Normal Texturen-------------------
\subsection{Bump und Normal Texturen}
\label{sec:tex_normal}

Eine Bump- oder Normal-Textur (meisten Bump- / Normal-Map genannt), simuliert die Höhen und Tiefen eines Objektes.
Das funktioniert, indem das Licht auf dem Objekt verschieden berechnet wird.\citep{unity:normal_vs_bump}


Bump-Maps:

Bump-Maps sind die ältere Methode der Höhenberechnung. Bump-Maps sind schwarz-weiß Texturen. Die schwarzen Pixel
zeigen Vertiefungen an, die weißen Pixel Erhöhungen. Je nachdem, wie die Normals der betreffenden Fläche ausgerichtet
sind, wird diese Fläche an gewissen Stellen schwäche beleuchtet / reflektieren als an anderen Stellen. Die Normals
einer Fläche sind die belichteten Seiten. Die Normal-Direction gibt an, aus welcher Richtung die Textur angezeigt
wird und welche Neigung die Fläche hat (Siehe Abbildung \ref{picture:normals}).

\begin{figure}[h]
    \centering
    \includegraphics[width=.9\textwidth]{SEI_Texturierung_Bump-NormalTexturen-Normals.png}
    \caption{In diesem Bild sieht man die Normal-Direction der verschiedenen Flächen}
    \label{picture:normals}
\end{figure}


Normal-Maps:

Normal-Texturen speichern etwas mehr Informationen. Während eine Bump-Textur nur die \textit{gefälschte} Tiefe
speichert, beinhalten Normal-Texturen auch die Lichtberechnung aus verschiedenen Perspektiven (rechte Seite,
linke obere Seite, frontale Seite). Diese Informationen werden in einer Normal-Map in den R-G-B Kanälen gespeichert.

In unserer Diplomarbeit wurden ausschließlich Normal-Maps verwendet, da sie mittels dem Programm
Materialize\citep{bbs:materialize} relative einfach zu erstellen sind. Bump und Normal-Texturen sind in den meisten
Fällen gleichwertig außer bei Objekten die Rundungen haben. In diesem Fall sind Normal-Texturen die bessere Wahl
um Details von seitlicher Perspektive zu generieren.

%----------------Roughness Texturen---------------Roughness Texturen---------------Roughness Texturen-------------------
\subsection{Roughness Texturen}
\label{sec:tex_roughness}

Roughness Texturen werden für die Spiegelung verwendet. Die Roughness bestimmt, mit welcher Streuung das Objekt Licht
reflektiert wird. Ein Roughness Wert von null (schwarz) entspricht kompletter Spiegelung. Ein Roughness Wert von eins
(weiß) entspricht gar keiner Spiegelung (Siehe Abbildung \ref{picture:roughness_illustration}).

\begin{figure}[h]
    \centering
    \includegraphics[width=.9\textwidth]{SEI_Texturierung_Roughness-Roughness.png}
    \caption{Darstellung der Lichtberechnung eines Roughness-Wertes}
    \label{picture:roughness_illustration}
\end{figure}


Das Programm Materialize verwendet Smoothness-Maps. Wenn eine Smoothness-Map für den Roughness-Wert verwendet wird,
muss er invertiert werden. In der Unreal-Engine funktioniert das mittels der 1-x Funktion (Siehe
Kapitel \ref{sec:algorithms} oder Abbildung \ref{picture:tex_mat_roughness}).

Im Normalfall befindet sich der Roughness-Wert zwischen 0,3 und 0,9. Allerdings wird für Materialien, die Wasser oder
Pfützen darstellen sollen gerne ein Roughnesswert von 0 verwendet.

%----------------Texturen in der Engine---------------Texturen in der Engine---------------Texturen in der Engine-------
\subsection{Texturen in der Engine}
\label{sec:tex_inside_engine}

Für dieses Kapitel werden einige Informationen über Materials vorab beschrieben.


Wenn in der Unreal-Engine ein neues Material erstellt wird, wird der Node-Editor\citep{ue:node_introduction} mit einer
Material Output Node geöffnet. Die Node hat verschiedene Inputs, die die physikalischen Eigenschaften des Materials
bestimmen. Die meisten Inputs akzeptieren eine konstante Value oder eine Textur. Bei einer Material Node sind nicht
alle Inputs verfügbar. Welche Inputs verfügbar sind, ist von der Material-Art abhängig (Siehe
Abbildung \ref{picture:mat_ue4_node}).

\begin{figure}[h]
    \centering
    \includegraphics[width=.9\textwidth]{SEI_Texturierung_Texturen in der Engine-MaterialOutput.png}
    \caption{Eine Material Output Node und die aktiven Inputs}
    \label{picture:mat_ue4_node}
\end{figure}


Um das System der Inputs genauer zu beschreiben verwendet wir den roughness Input. Nehmen wir als Beispiel ein Material
für eine grüne Türe und ein Material für eine Terrasse (Siehe Abbildung \ref{picture:tex_mat_roughness}).
Das erste Bild zeigt die grüne Türe. Die gesamte Türe ist ohne große Unterschiede. Die Türklinke und das Schloss haben
eigene Materialien. Daher hat die Türe einen konstanten Roughness-Wert von 0,8 (Die linke Hälfte von dem Bild zeigt
den Wert als Farbe an).

Bei der Terrasse sollen die Ziegel eine etwas geringere Roughness besitzen als die Zwischenräume. Daher wurde eine
Smoothness-Textur erstellt. Wie man sehen kann, sind die Zwischenräume der Textur dünkler, was bedeutet, dass die
Zwischenräume eine geringer Roughness hätten als die Ziegel. Damit der erwünschte Effekt erzielt wird, wurde die
1-X Node verwendet um die Textur zu invertieren.

\begin{figure}[h]
    \centering
    \includegraphics[width=.9\textwidth]{SEI_Texturierung_Texturen in der Engine-InputMoeglichkeit.png}
    \caption{Der Roughness input wurde einmal mit einem konstanten Wert verwendet und einmal mit einer Textur}
    \label{picture:tex_mat_roughness}
\end{figure}


Property-Maps / Texture Masks\citep{ue:tex_property_map}:

Die letzte Art von Texturen, die in diesem Buch näher gebracht wird, ist eine Property-Map. Property-Maps werden
verwendet, um die Performance des Spiels zu optimieren. Bei Texturen, wie zum Beispiel eine Roughness-Textur oder eine
Bump-Textur, werden nur die schwarz-weiß Informationen benötigt. Jede Textur wird allerdings einzeln von der Engine
geladen, wenn das Spiel / Level gestartet wird. Um diesen Vorgang zu optimieren, kombiniert man 3 schwarz-weiß Texturen
in einer Property-Map. Das bedeutet, dass man den Rot-Wert, den Grün-Wert und den Blau-Wert verwendet, um die
Informationen zu speichern. Da man in der Unreal-Engine die Kanäle einzeln verwenden kann und diese nur einen Wert
zwischen 0 und 1 ausgeben, funktioniert das einwandfrei. Damit hat man drei Texturen in einer
Property-Map vereint und es muss nur eine Textur geladen werden. (Siehe Abildung \ref{picture:property_map})

\begin{figure}[h]
    \centering
    \includegraphics[width=.9\textwidth]{SEI_Texturierung_Texturen in der Engine-Property-Map.png}
    \caption{Eine Property-Map und die einzelnen Informationen innerhalb der RGB-Kanäle}
    \label{picture:property_map}
\end{figure}


%-----------------------Materialien----------------------Materialien----------------------Materialien-------------------
\section{Materialien}
\label{sec:materials}

Texturen und Materialien sind eng miteinander verbunden. Texturen werden verwendet, um Materialien zu erstellen.

Eine Diffuse-Textur gibt dem Material nur verschiedene Farben. Bump-Texturen, Normal-Texturen, Roughness-Texturen, usw.
sind werden für die Verwendung innerhalb Materialien erstellt. Um ein Objekt weich, oder hart wirken zu lassen, dem
Objekt Tiefe zu geben oder um zu bestimmen ob das Objekt metallisch oder nicht ist, muss ein Material erstellt werden.
Diese Aspekte wurden auch bei der Umsetzung der Materialien im Rahmen dieser Diplomarbeit berücksichtig. Weiterführend
können Materialien auch das Objekt \textit{verformen} oder physikalische Prozesse simulieren (Schmelzen, Glühen,
gefrieren, etc.).

%-----------------------Material Inputs----------------------Material Inputs----------------------Material Inputs-------
\subsection{Material Inputs}
\label{sec:mat_inputs}

Wie in einem vorherigen Kapitel bereits erwähnt (Kapitel \ref{sec:tex_inside_engine}), wird eine Material-Output Node
erstellt, wenn man ein neues Material hinzufügt. In diesem Kapitel werden die wichtigsten Inputs, die im Laufe des
Projektes verwendet wurden unter die Lupe genommen:

\begin{longtable}{|p{4cm}|p{9.6cm}|}
    \hline
    \endfirsthead
    & \textbf{Input} & \textbf{Beschreibung} &\\
    \hline
    \endhead

    & Base Color & Die Base-Farbe bestimmt das diffuse Aussehen des Materials & \\

    & Metallic & Der Metallic Input bestimmt, ob das Material ein Metall ist oder nicht. Im Normalfall wird für
    diesen Wert 0 oder 1 verwendet. 0 bedeutet, das Material ist metallisch, 1 bedeutet, es ist nicht
    Metallisch. Ein Wert zwischen 0 und 1 lässt das Material unrealistisch erscheinen, was gerne
    für Science-Fiction Materialien verwendet wird (z.B. organisches Metall) & \\

    & Roughness & Der Roughness Input bestimmt die Lichtberechnung der Reflektion. Umso niedriger der Wert
    ist, desto stärker spiegelt das Material & \\

    & Emissive Color & Die Emissive Color lässt das Material leuchten. Dieser Input kann für Ampeln, Items die man im Spiel
    finden muss oder Neon-Lichter gut eingesetzt werden & \\

    & Opacity & Der Opacity Input ist standardmäßig nicht verfügbar. Um die Opacity zu verwenden, muss man die
    Material-Node anklicken und im Details-Bereich den Blend-Mode auf "Translucent" umstellen (Siehe
    Abbildung \ref{picture:translucent}). Der Opacity Input bestimmt, ob und wie stark man durch das Material hindurchsehen
    kann. Dieser Input wird hauptsächlich durch Texture-Masken (Siehe Kapitel \ref{sec:texture_painting}) verwendet & \\

    & Normal & Der Normal Input ist für Normal-Texturen gedacht. Er bestimmt die \textit{gefälschte} Tiefe des Objektes & \\

    \caption{Material Inputs der Output Node}
    \label{table:mat_inputs}
\end{longtable}

\begin{figure}[h]
    \centering
    \includegraphics[width=.9\textwidth]{SEI_Materialien_Material_Inputs-TranscluentType.png}
    \caption{In diesem Bereich kann man die Material-Art verändert. Dadurch werden einige Inputs aktiviert oder deaktiviert}
    \label{picture:translucent}
\end{figure}


%Material-Ebenen,-Funktionen, -InstanzenMaterial-Ebenen, -Funktionen, -InstanzenMaterial-Ebenen, -Funktionen, -Instanzen
\subsection{Material-Ebenen, -Funktionen, -Instanzen}
\label{sec:mat_lay_func_ins}

Für die Umsetzung einiger Materialien in diesem Projekt wurden Material-Ebenen, -Funktionen und -Instanzen verwendet.

Material-Ebenen\citep{ue:mat_landscape}:

Material-Ebenen werden verwendet um verschiedene Materialien übereinander darzustellen. Sie werden ausschließlich für
Landscape-Materials verwendet. Ein Material wird Landscape-Material genannt, wenn es für den Boden der Spiel-Welt
verwendet wird. Um ein Landscape-Material zu erstellen muss zuerst im Details Bereich der Material Output Node, ein
Häkchen in dem Kontrollkästchen "Use Material Attributes" gesetzt werden. Dadurch wird die Material Output Node auf
einen Input reduziert: "Material Attributes". Mit einem Rechtklick im Material-Graphen kann man nach einer bestimmten
Funktion suchen. Im Suchbereich mittels dem Suchbegriff "Landscape Layer Blend", kann eine Layer Blend Node hnizugefügt
werden. Diese wird mit dem Material Attributes Input verbunden (Siehe Abbildung \ref{picture:layer_blend}).

\begin{figure}[h]
    \centering
    \includegraphics[width=.9\textwidth]{SEI_Materialien_Material-Layer-MatLayerBegin.png}
    \caption{Layer Blend Node und Material Attributes Einstellung}
    \label{picture:layer_blend}
\end{figure}

Wenn man nun die Layer Blend Node anklickt, kann man im Details Bereich Layers hinzufügen. Jeder hinzugefügte Layer
ist ein Input in der Layer Blend Node (Siehe Abbildung \ref{picture:layers}). Die einzelnen Layers stellen am Ende die verschiedenen
Materialien dar. Der Blend Type bestimmt, wie ein Material hinzugefügt wird und wie es sich zu anderen vorhandenen
Materialien verhält. Das Material, dass das Basis Material sein wird, sollte die Einstellung "LB Weight Blend"
verwenden.

\begin{figure}[h]
    \centering
    \includegraphics[width=.9\textwidth]{SEI_Materialien_Material-Layer-MatLayerLayers.png}
    \caption{Die Layer Blend Node hat mehrere Inputs bekommen}
    \label{picture:layers}
\end{figure}

Um die Layer-Inputs verbinden zu können muss die "MakeMaterialAttributes"-Funktion verwendet werden. Dieser
Funktions-Output wird mit dem "Layer <name>" Input der Layer Blend Node verbunden. Die
"MakeMaterialAttributes"-Funktion wird wie eine normale Material Output Node verwendet. Der Height-Input wird mit einer
Height-Textur (ähnlich einer Bump-Textur) verbunden.

Material-Funktionen\ref{sec:mat_lay_func_ins}:

Unreal-Engine bietet die Möglichkeit, eigene Funktionen für Materialien zu schreiben. Mit einem Rechtsklick im
Content-Browser kann man unter Materials & Textures eine neue Material Function erstellen (Siehe
Abbildung \ref{picture:make_mat_func}). Eine Material-Function kann in jedem Material verwendet werden. Eingefügt wird sie, wie jede andere
Funktion auch (Rechtsklick im Graph-Editor und nach dem Namen suchen). Material-Funktionen verwendet man, wenn man
den gleichen Ablauf öfters mit verschiedenen Parameter wiederholen muss.

\begin{figure}[h]
    \centering
    \includegraphics[width=.9\textwidth]{SEI_Materialien_Material-Layer-MatFunctionErstellen.png}
    \caption{Erstellen einer Material Funktion}
    \label{picture:make_mat_func}
\end{figure}

Material-Funktionen haben Input und Output-Parameter. Diese bestimmen die Inputs und Outputs der Node, wenn die
Funktion implementiert wird (Siehe Abbildung \ref{picture:use_mat_func}). Die Funktionen können entweder per Drag and
Drop in einem Material hinzugefügt werden oder mittels der "MaterialFunctionCall" Funktion.

\begin{figure}[h]
    \centering
    \includegraphics[width=.9\textwidth]{SEI_Materialien_Material-Layer-MatFunctionVerwenden.png}
    \caption{Darstellung der Inputs / Outputs einer Material Funktion und die Verwendung in einem Material}
    \label{picture:use_mat_func}
\end{figure}

Material-Instanzen\citep{ue:mat_instances}:

Material-Instanzen sind auf einem ähnlichen Prinzip aufgebaut wie Material-Funktionen, man verwendet sie um gleiche
Progammabläufe zu vereinein, damit man sie nicht doppelt schreiben muss. Der Entwickler kann dann innerhalb
verschiedenen Instanzen kleinere Unterschiede des Basis- / Eltern-Materials bewirken. In unserem Projekt haben wir
Material-Instanzen nur für die optionalen Erweiterungen entwickelt. Es wurde ein einfaches Metall-Material entwickelt,
dass für verschiedene Objekte verwendet wurde (Siehe Abbildung \ref{picture:base_metal_material}). Die Unterschiede auf
den einzelnen Material-Instanzen sind: Die Metall-Farbe, ob das Metall Dellen hat, wie stark die Dellen sein sollen,
welche Grundform die Dellen haben sollen, ob Rost vorhanden ist, wie viel des Metalls soll rostig sein und welche
Grundform hat der Rost.

\begin{figure}[h]
    \centering
    \includegraphics[width=.9\textwidth]{SEI_Materialien_Material-Layer-MetalBaseMat.png}
    \caption{Basis Material um Instanzen des Basis-Materials zu erstellen}
    \label{picture:base_metal_material}
\end{figure}

Um eine Material-Instanz zu erstellen, klickt man mit der rechten Maustaste auf ein bestehendes Material. Danach klickt
man auf Create Material Instance. Um die erstellte Material-Instanz zu individualisieren muss das "Eltern-Material",
Material-Parameter besitzen.

Material-Parameters:

    Material-Parameter können auf verschiedene Arten  erstellt werden:
    \begin{itemize}
        \item  Mittels Rechtsklick auf einen Input und der Option "Promote to Parameter"
        \item  Mittels Rechtsklick auf eine Konstante oder Textur und der Option "Convert to Parameter"
        \item  Mittels Rechtsklick im Graph-Editor und der Suche nach "Parameters"
    \end{itemize}

Die zwei größten Vorteile von Material-Instanzen sind, dass Änderungen sofort in der 3D-View ersichtlich sind und
verschiedene Varianten eines Basis-Materials leicht zu erstellen sind. Die erstelleten Parameter kann man dann im
Details Bereich der Material-Instanz einstellen (Siehe Abbildung \ref{picture:edit_mat_instance}).

\begin{figure}[h]
    \centering
    \includegraphics[width=.9\textwidth]{SEI_Materialien_Material-Layer-MetalMatInst.png}
    \caption{Editieren einer Material Instance in dem Details Bereich}
    \label{picture:edit_mat_instance}
\end{figure}

%---Wichtige Algorithmen/Funktionen---------Wichtige Algorithmen/Funktionen-------Wichtige Algorithmen/Funktionen-------
\subsection{Wichtige Algorithmen/Funktionen}
\label{sec:algorithms}

Im letzten Kapitel über die Materialien in der Unreal-Engine werden die wichtigsten Algorithmen und eingebauten
Funktionen aufgezählt und beschrieben, die im Laufe der Diplomarbeit verwendet wurden.

\begin{longtable}{|p{4cm}|p{9.6cm}|}
    \hline
    \endfirsthead
    & \textbf{Algorithmus / Funktion} & \textbf{Beschreibung} &\\
    \hline
    \endhead

    & 1-X & Invertiert den Wert & \\

    & Lerp (LinearInterpolation) & Dieser Algorithmus orientiert sich an einem Aplha Wert. Der Alpha Wert muss ein
    Input zwischen 0 und 1 sein. Wenn der Alpha Wert 0 ist wird als Output der Wert
    verwendet, welcher sich in dem A-Input befindet. Wenn der Alpha Wert 1 ist wird
    als Output der Wert verwendet, welcher sich in dem B-Input befindet. Wenn der
    Alpha Wert sich zwischen 0 und 1 befindet, wird der - dem entsprechende prozentuale -
    Wert zwischen dem A- und B-Input als Output-Wert verwendet & \\

    & Add & Addiert zwei Werte & \\

    & Multiply & Multipliziert zwei Werte. Bei dieser Funktion ist zu beachten, dass die Farbe Schwarz
    den Wert 0 hat. Sobald zwei Farben / oder Texturen miteinander multipliziert
    werden, wird es bei schwarzen Stellen keine Veränderung geben & \\

    \caption{Material Funktionen}
    \label{table:mat_algorithms}
\end{longtable}

Unreal-Engine bietet noch viele weiter Funktionen an, die allerdings nicht hier gelistet werden. Die hier gelisteten
Funktionen stellen das Basis-Wissen dar, welches benötigt wird um gute Materialien zu erstellen.
Um Materialien zu perfektionieren werden auch andere Funktionen verwendet, die unter der Quellenangabe gelistet
sind\citep{ue:mat_algorithms}.
