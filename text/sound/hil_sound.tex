\section{Sound}
\subsection{Soundtrack in Spielen und Horro}
Soundtrack und Effekte sind in Videospielen und anderen Medien besonders wichtig.
Sie erzeugen ein Gefühl von Realismus. Spezifisch im Horror Genre sind Soundtrack und Effekte nochmal wichtiger, da sie die ganze Spannung aufbauen können.

\subsection{Audacity}
Audacity ist eine gratis open-source Software unter der Generel Public Liscense.
Somit ist Audacity ideal für unser Projekt geeignet. Audacity ermöglicht das Aufnehmen und bearbeiten von Audio Dateien, auch mit mehreren Spuren.
Die eingebauten Effekte wie Verstärker, Kompressor, Rückwärts, Echo etc. sind ausreichend für den Umfang der aufgenommenen Sound Dateien.

\subsection{Umsetzung der Effekte und Musik}
\subsubsection{Audioplan}
Da Sound im vollen Ausmaß den Rahmen für die Diplomarbeit sprengen würde haben wir, das ganze Team, uns zusammengesetzt und eine
Audioplan erstellt. Im Audioplan steht welche Sounds wir unbedingt brauchen,
um eine ausreichende Immersion zu erreichen, und wie sie grob umgesetzt werden können. Aufgeteilt wurde der Audioplan in: Interface, Charaktere und Mechanismus.

\paragraph{Interface:}
In den Interfaces wie Start Menü und Pause Menü gibt es zwei verschiedene Arten von Sounds. Einmal die Hintergrund Musik und Klick-Sounds.
Die Hintergrund Musik sollte passend für ein Horrorspiel sein – düster und von der Geschwindigkeit eher langsam.
Der Klick-Sound ist dafür gedacht, bei Aktionen, die in einem Menü stattfinden, z.B. einen Unterpunkt auswählen oder mit der Maus über den Menüpunkt landen, zu spielen.

\paragraph{Charaktere:}
Alle relevanten Sounds, welche von den beiden Charakteren stammen. Dazu gehören die klassischen Geh- und Renn-Sounds auf
verschiedenen Untergründen, ein Fang-Sound, wenn der Hauptcharakter gefangen wird, ein Sound, der dann abspielt, wenn man Objekte aufsammelt
und Sounds für Interaktionen mit der Spielewelt wie z.B. Tür oder Schubladengeräusche. Die Geh- und Renngeräusche sind zu
schwierig, um sie gut umzusetzen, deswegen suchten wir uns Sound Dateien, die wir gratis benutzen konnten. Insgesamt gibt es drei
verschiedene Untergründe: Beton, Gras und Erde oder Steine. Ein Rucksackgeräusch für das aufsammeln der Objekte wirkte am passendsten für uns
und für den Fang-Sound planten wir einen Schlag auf den Rücken.
Außerdem wird es einen Jumpscare-Sound beim Fangen geben. Ein Jumpscare-Sound ist ein plötzlich lauter Ton oder Geräusch, das abgespielt wird.

\paragraph{Mechanismus:}
Die Cutscene am Ende des Spiels beinhaltet die Hauptobjekte. Diese, und der gesamte Mechanismus
machen Geräusche, während dieser ausgeführt wird. Zum Mechanismus gehört der Zement,
der in das Grab fließt, eine Zementtrommel, die aktiv ist, ein Seil, das knarrt und Grabstein, der neben der Graböffnung platziert wird.

\subsubsection{Instrumente und Hilfsmittel}
Um die Sounds umsetzen zu können, braucht man gewisse Hilfsmittel wie Instrumente, Mikrofone und
passende Objekte. Viele Sounds konnten wir selbst im Studio der HTL3Rennweg umsetzen. Andere wiederum wurden aus dem Internet frei besorgt.

\subsubsection{Aufnahme}
Bei der Aufnahme muss man einige Faktoren beachten. Die Platzierung des Mikrofons, die Aufnahmequalität und die Wahl
des Aufnahmeprogramms. Als Software benutzten wir Audacity und darin passten wir die Aufnahmequalität auf mindestens 192kHz und 24 Bittiefe an.

\paragraph{Merkmale der Audioqualität:}
Besonders zu beachten sind zwei Werte: die Abtastrate und die Bittiefe / Auflösung. Beide dieser Werte sollten bei den Aufnahmen
eher hoch gewählt sein, da beim Bearbeiten und Manipulieren der Aufnahmen oft Qualität verloren geht. Die Abtastrate wird in Kilohertz angegeben und die Auflösung in Bit. Typische
CD-Qualität hat eine Abtastrate von 44,1kHz und eine Auflösung von 16 Bit. Das menschliche Gehör ist vergleichbar mit etwa 192kHz und 24 Bit.

\paragraph{Hintergrund Musik:}
Die Hintergrund Musik nahmen wir mithilfe von einem Cello im Studio auf. Wir ließen dem Spieler des Cellos verschiedene Melodien und Tricks mit dem Cello spielen.
Da wir keine klaren Noten hatten, sondern nur eine grobe Vorstellung wie die finale Musik werden könnte, durfte der Spieler improvisieren lassen.

\paragraph{Klick-Sound:}
Die Klick-Sounds wurden im Studio aufgenommen. Um einen passenden Sound zu erzeugen nahmen wir
einen Zahnstocher und brachen ihn in zwei Hälften. Das Resultat war ein passender Sound für die Klick-Sounds.

\paragraph{Idle-Sounds:}
Für die Idle Sounds haben wir eine Jacke angezogen und uns leicht bewegt. Des Weiteren haben wir unser eigenes Atmen aufgenommen.

\paragraph{Geh- und Renn-Sounds:}
Da diese zu aufwendig umzusetzen sind im Studio mussten wir auf freie Online-Quellen zurückgreifen.
Geräusche für die drei verschiedenen Untergründe, Gras, Beton und Erde oder Stein, waren schnell gefunden.

\paragraph{Fang-Sounds:}
Da diese zu aufwendig umzusetzen sind im Studio mussten wir auf freie Online-Quellen zurückgreifen.
Geräusche für die drei verschiedenen Untergründe, Gras, Beton und Erde oder Stein, waren schnell gefunden.

\paragraph{Objekte aufsammeln:}
Geplant haben wir einen Rucksack zu nehmen und neben dem Mikrofon in zu platzieren und darin herumzuwühlen.

\paragraph{Tür öffnen:}
Um einen passenden Tür-Sound aufzunehmen haben wir die Sensitivität des Mikros erhöht und die Tür des Studios auf und
zu gemacht. Die Resultate sind ein passender Druck des Tür Schließens und ein Sound einer quietschenden Tür für das Öffnen.

\paragraph{Fließender Zement:}
Als Idee hatten wir einen dickflüssige fallende Flüssigkeit. Dies ist schwer im Studio nachzumachen,
deswegen suchten wir einen freien Sound im Internet.

\paragraph{Zementtrommel:}
Ürsprünglich war geplannt eine echte Zementtrommel aufzunehmen. Dies stellte sich schwieriger
aus als gedacht. Als Alternative wurde ein Mixer aufgenommen und in der Post-Production verlangsamt und Tonlage verändert.

\paragraph{Grabstein:}
Für das Niederstellen des Grabsteines nahmen wir zwei Steine und schlugen sie gegeneinander. Dies erzeugt einen passenden Sound.


\subsubsection{Post-Production}

\paragraph{Exportierung der Audio Dateien:}
