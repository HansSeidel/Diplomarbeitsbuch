\section{Einleitung}
Bei der 3D Modellierung, werden Objekte in einem dreidimensionalen Raum erstellt. Mit anderen Worten, es gibt ein Koordinatensystem mit einer X-,Y- und Z-Achse.
In diesem werden Punkte gesetzt, die auch zu Linien und Flächen verbunden werden können. Die Summe solcher Punkte, Linien und Flächen ergibt dann ein 3D-Objekt.
Diese 3D-Objekte können für alle möglichen Zwecke eingesetzt werden. Damit man sich etwas darunter vorstellen kann, folgen nun ein paar Beispiele für
eine mögliche Anwendungen.

Anwendungsmöglichkeiten:
\begin{itemize}
    \item Echte Objekte nachmodellieren und digital z.B. in einem Bild verwenden.
    \item Objekte erschaffen und für z.B. Spiele verwenden.
    \item Objekte für Veranschaulichungen von Zukunftsprojekten erstellen z.B. ein Wohnhaus.
\end{itemize}

\section{Modellierungsprogramm}
Es gibt viele Programme mit denen man 3D-Objekte erstellen kann.
\citep{wiki:modellierungsprogramm_beispiele} \todo{Nachschauen ob Wikipediazitat passt} Populäre Programme dieser Art sind u. a.:

\begin{itemize}
    \item Maya
    \item Cinema 4D
    \item LightWave 3D
    \item Blender
    \item 3ds Max
    \item Houdini
\end{itemize}

Unser Team hat sich entschieden die benötigten Modelle in Blender zu erstellen, weil wir mit dem Programm im Unterricht arbeiten und
es kostenlos verwendbar ist.

\todo{Satz "In folgenden Punkten,..." hinzufügen ???}

\subsection{Object Mode}
Der Object Mode in Blender ist dazu da um ein ganzes 3D-Objekt zu verändern. Man kann es Beispielsweise im Koordinatensystem ausrichten, aber
auch rotieren, skalieren oder sogar den Mittelpunkt des Objekts verändern. Das verändern des Mittelpunkts ist besonders nützlich, um Objekte auf
einem bestimmten Punkt zu verändern.\todo{d1} Wie wir im \autoref{chap:Exportieren_von_Blender_zu_Unreal_Engine_4} noch erfahren, hat uns dieses Feature besonders geholfen.

\subsection{Edit Mode}
\subsection{Modifikatoren}
\section{Exportieren von Blender zu Unreal Engine 4}
\todo{d1}
\label{chap:Exportieren_von_Blender_zu_Unreal_Engine_4}