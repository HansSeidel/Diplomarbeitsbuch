\section{Object Mode}
Der Object Mode in Blender ist dazu da, um ein ganzes 3D-Objekt zu verändern. Man kann es beispielsweise im Koordinatensystem ausrichten, aber
auch rotieren, skalieren oder sogar den Origin des Objekts verändern. Das verändern des Origins ist besonders nützlich, um Objekte auf
einem bestimmten Punkt zu verändern. Wie wir im Kapitel \ref{Exportieren_von_Blender_zu_Unreal_Engine_4:ref1} \dq \todo{Referenz kursiv} Exportieren von Blender zu Unreal Engine 4\dq noch erfahren, hat uns dieses Feature besonders geholfen.

\subsection{Smooth Shading}
\label{objectMode:smoothshading}
Mit Smooth Shading werden die Schatten eines Objektes so berechnet, dass das ganze Objekt Glatt aussieht (Abbildung \ref{modifikatoren:image7}).
Man kann diesen Effekt, sehr gut für Runde Objekte benutzen, da man dank dieser Funktion, nicht so viele Flächen für ein Objekt braucht, um
es Rund aussehen zu lassen. Man kann also sehr gut die Performance des Spiels verbessern, wenn man statt vielen Flächen Smooth Shading verwendet.

\subsection{Flat Shading}
Flat Shading lässt Objekte Kantig aussehen. Es ist standardmäßig für Objekte ausgewählt und ist gut für nicht Runde Objekte.

\subsection{Ebenen}
Sehr nützlich im Object Mode sind Ebenen. Man kann dadurch mehrere Objekte Ein- bzw. Ausblenden.
Das war Beispielsweise hilfreich, um das Haus zu modellieren und zusammenzufügen (Kapitel \ref{haus:ref1}).
Somit konnte man nämlich die Objekte je nach Stockwerk anzeigen.

Neben Ebenen gibt es noch die Möglichkeit, ein Objekt im Local View und im Global View anzuzeigen.
Die Standardansicht eines Objektes ist im Global View. Man sieht das ausgewählte Objekt und alle anderen Objekte in derselben Ebene, die nicht ausgeblendet sind.
Im Local View kann man alle ausgewählten Objekte einzeln beobachten. Dazu wählt man die Objekte die man beobachten möchte aus und drückt dann \keys{NUM + /}.
