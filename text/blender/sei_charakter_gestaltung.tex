\section{Charakter Gestaltung}
Bevor man Blender öffnet und beginnt einen Charakter zu gestalten, ist es wichtig sich ein Bild von
dem Charakter zu machen. Wenn man das erste Mal einen Charakter gestaltet, unterläuft vielen der
Fehler, dass sie den Charakter aus dem Kopf heraus modellieren. Allerdings passiert es dann sehr
schneller, dass man ein Modell entwickelt, welches einem nicht gefällt oder welches nicht
gebrauchen kann. Daher ist es wichtig, den Charakter zuerst auf Papier zu zeichnen. Auch wenn das
Endresultat eines Charakters neutral sein sollte, ist es hilfreich sich bereits im Vorhinein einige
Eigenschaften des Charakters zu überlegen.

Ist der Charakter eher ein grimmiger Mensch oder ein sehr freundlicher? Ist er überhaupt ein
Mensch? Wie kleidet er sich? Wie bewegt er sich? Hat er irgendwelche Verletzung?

Diese Details helfen, den Charakter bereits am Papier detailliert darzustellen und seine Rolle im Spiel
authentischer zu gestalten.

%----------Skizzen zu dem Modell-------Skizzen zu dem Modell-------Skizzen zu dem Modell-------Skizzen zu dem Modell----
\subsection{Skizzen zu dem Modell}

In unserem Fall haben wir die beiden Hauptcharaktere gleich zu Beginn des Objektes charakterisiert.
Mit Beispielbildern und einer Personenbeschreibung war für alle Teammitglieder ein klares Bild der
Charaktere vor Augen. Vor allem wenn man beginnt Charaktere zu modellieren, sollte man sich
Vorbilder nehmen. Niemand wird bei seinem ersten Versuch einen perfekten Charakter modellieren
können. Für Übungszwecke ist es daher hilfreich sich bereits vorhandenen Charakteren als Vorlage
zu verwenden.
Sobald die erste Vorstellung des Charakters entstanden ist, kann man beginnen die Skizzen zu
entwerfen. Die Skizzen sollen ein klareres und einheitlicheres Bild erstellen. Sie transportieren die
Vorstellung des Charakters auf Papier. Für 3D-Entwickler ist es am angenehmsten, wenn die Skizzen ein Modell
(in diesem Fall den Charakter) von drei Seiten anzeigen. Eine Front-Perspektive, eine Right-Perspektive und eine
Top-Perspektive. Für die Entwicklung des Charakters kann man diese Skizzen leicht in Blender einfügen und genau
in diesen Perspektiven anzeigen.



Einfügen von Skizzen in Blender\citep{blender:background_images}:
In der 3D-View von Blender gibt es die Möglichkeit Bilder dem Hintergrund hinzuzufügen. Diese
Bilder sind jedoch nur in der orthographischen Ansicht verfügbar. Die orthographische Ansicht ist
eine zwei Dimensionale Darstellung aus der Sicht einer bestimmten Achse.
Wenn sich der Mauszeiger in der 3D-View befindet und die Taste \keys{N} gedrückt wird, öffnet sich
auf der linken Seite des Fensters die Properties Region. Unter dem Bereich: Background-Images kann
man die Skizzen nun hinzufügen. Sobald man ein Bild hinzugefügt hat, kann dieses mit
den vorhandenen Einstellungen so angepasst werden, dass es am angenehmsten zum Arbeiten ist.
Hier sind einige der wichtigsten Einstellungen kurz beschrieben.


\begin{longtable}{|p{4cm}|p{9.6cm}|}
    \hline
    \endfirsthead
    \textbf{Einstellung} & \textbf{Beschreibung} \\
    \hline
    \endhead

    & Axis & Die Achsen Einstellung gibt die Möglichkeit auszuwählen von welcher Seite
    das Bild angezeigt werden soll. Es wird dann nur angezeigt, wenn man sich in der
    3D-View in dieser Seite befindet & \\

    & Opacity & Die Opacity bestimmt zu wie viel Prozent das Bild sichtbar ist. (0 = Komplett transparent, 1 = 1
    überhaupt nicht transparent & \\

    & Back/Front & Diese Einstellung bestimmt, ob das Bild das Objekt verdecken soll (Front) oder nicht (Back) & \\

    & X/Y/Rotation/Size & Mit diesen Einstellungen kann man die Skizzen dann perfekt
    an das Modell anpassen & \\

    \caption{Einstellungen um ein Bild in der Blender-View als Hintergrund zu importieren}
\end{longtable}


%----------Menschliche Relationen------Menschliche Relationen------Menschliche Relationen------Menschliche Relationen---
\subsection{Menschliche Relationen}
    Wenn man erst seit kurzen Charaktere modelliert, passiert es sehr schnell, dass die Proportionen
    nicht korrekt sind. Unabhängig davon, ob man Charaktere im 3-Dimensionalen Bereich oder auf Papier
    designt, sollte man einige Proportionsregeln beachten und diese einzeichnen bevor man mit dem
    Modellieren beginnt. (Auch Blender hat eine Möglichkeit zu skizieren. Allerdings wird das nicht in
    diesem Buch behandelt\citep{blender:grease_pencil}.)

Einige Linien in einem erwachsenen Gesicht, stehen immer in einem bestimmten Verhältnis zueinander
(Siehe Abbildung\ref{blender:proportions}). Die Linie \textbf{A} grenzt die Schädeldecke ab. Von
der Schädeldecke bis zum Stirnansatz ist zirka ein Abstand von 1/7 der Gesichtsgröße. Die Linie
\textbf{C} zeigt, dass die Augenbrauen und der Abschluss der Ohren auf gleicher Höhe sein sollten.
\textbf{C.1} verbindet das obere Ende der Nase mit der Mitte des Auges und dem Ansatz der Ohren.
Außerdem sollte die Linie  \textbf{C.1} genau in der Mitte des Gesichts liegen. Auf der Linie
\textbf{D} befindet sich das Ohrläppchen und der Ansatz der Nase. Linie  \textbf{E} stellt das Kinn
und den Abschluss des Gesichtes dar. Da dieser Charakter übergewichtig modelliert wurde, ragt sein
Doppelkinn über die Linie hinaus. Die Linie  \textbf{D.1} zeigt die untere Grenze des Mundes.
Aus der frontalen Ansicht kann man auch vertikale Linien ziehen. Diese würden die Tränendrüse mit den Nasenflügeln
verbinden und die Iris mit den Mundwinkeln.\citep{book:kunst_des_zeichnens}
\begin{figure}[h]
    \centering
    \includegraphics[width=.8\textwidth]{SEI_CharakterGestaltung_MenschlicheRelationen_KopfInRealtionen.png}
    \caption{Proportionen im Gesicht eines Erwachsenen}
    \label{blender:proportions}
\end{figure}

%----------Arten der Charakter-Modellierung------Arten der Charakter-Modellierung------Arten der Charakter-Modellierung-
\subsection{Arten der Charakter-Modellierung}
Im Laufe der Diplomarbeit wurden bei der Entwicklung der zwei Hauptcharaktere zwei verschiedene Arten der
Charakter-Modellierung verwendet. Der Protagonist wurde mittels der Subdivision-Methode modelliert. Der Antagonist
wurde mittels dem Blender Sculpt Tool\citep{blender:sculpting_general} erstellt .
Bei diesen Methoden gibt es zwei bedeutende Unterschiede.


Ein Charakter, der mittels des Sculpt-Tools modelliert wird kann meistens nicht für ein Spiel verwendet werden. Die
Detailarbeit die bei dem Sculpting durchgeführt wird generiert zu viele Polygone (Flächen) auf dem Charakter. Diese
Anzahl an Flächen kann der Computer nicht berechnen. (Im Falle des Antagonisten sind zirka 1.5 Millionen Flächen und 2
Millionen Punkte entstanden.) Das fertige Modell wird verwendet, um Graphiken zu erstellen, welche die Details des
Modells \textit{faken}. Diese Graphiken werden Normal-Maps genannt (\textit{Kapitel\ref{texturierung:bnTex}}).

Bei der Subdivision Modellierung hat der Entwickler eine größere Kontrolle über die Flächenanzahl. Allerdings ist es
dafür schwieriger Details oder Gesichtszüge zu modellieren. Diese müssen im Nachhinein gezeichnet oder entwickelt
werden.

Der zweite Unterschied der Methoden ist die tatsächliche Umsetzung des Charakters. Mit dem Sculpt-Tool von Blender
kann man Objekte modellieren als wären sie aus Ton gemacht. Bei der Subdivision-Methode werden die einzelnen
Körperteile erstellt in dem man die Flächen des Objektes einzeln vervielfältigt und manipuliert. Um das genauer zu
erklären werden die Prozesse der verschiedenen Methoden beschrieben, um ein Auge zu modellieren.

Sculpting-Methode:
Mittels verschiedenen Pinseln (\textit{siehe Kapitel\ref{sculpting:process}}) drückt man Teile des Objekte ein,
hebt andere hervor, verschiebt sie, glättet sie ab und flacht sie ab. Während dieser Prozesse werden (mit den richtigen
Einstellungen) so viele Fläche erstellt, wie es die Detailarbeit verlangt.

Subdivision Modellierung:
Man wählt direkt die Flächen aus, die sich bei dem Auge befinden (in den meisten Fällen zwischen 4 und 16 Flächen).
Diese Flächen löscht man. Dann fügt man eine Kugel hinzu und passt sie dem Kopf an. Wenn die Position passt,
verbindet man die zwei Objekte und entfernt die überflüssigen (nicht sichtbaren) Flächen. Bei diesem Prozess kann und
muss selbst bestimmt werden, wie viele Flächen entfernt werden und wie viele mittels der Kugel hinzugefügt werden.

Bei der Subdivision Modellierung wird auch gerne das Sculpt-Tool für den letzten Feinschliff verwendet. Es ist auch
möglich das Sculpt-Tool zu verwenden, ohne das neue Fläche generiert werden (KORR: VERLINKEN MIT DEM KAPITEL
DYNTOPO).

%--------------------------Sculpting--------------------------Sculpting--------------------------Sculpting--------------
\subsection{Sculpting}
\label{sculpting:process}

Das Sculpt-Tool von Blender teilt sich in die Bereiche \textit{Brush, Texture, Stroke, Curve, Dyntopo, Symmetry}
und \textit{History} auf. Die wichtigsten Bereiche, die auch bei der Umsetzung des Antagonisten verwendet wurden sind
unten angeführt.

Wenn man ein Objekt mit dem Sculpt-Tool bearbeiten möchte, ist die Verwendung von Short-Cuts
sehr hilfreich. Erfahrungsgemäß sind Short-Cuts in Blender  der beste Weg um sich mindestens 20\% der Zeit zu
sparen. Beim Sculpten bringen Short-Cuts sogar 40\% der Zeit zurück.

\begin{figure}[h]
    \centering
    \includegraphics[width=.9\textwidth]{SEI_CharakterGestaltung_Arten_Sculpting-SculptMode.png}
    \caption{Der Sculpt-Mode von Blender ist über die Listenauswahl im unteren Bereich der 3D-View erreichbar}
    \label{picture:sculpt_mode}
\end{figure}

\textit{Brush} Bereich\citep{blender:sculpting_brushes}:

Der \textit{Brush} Bereich bestimmt die grundlegenden Einstellungen des Pinsels.
Wenn man auf das Bild in dem Bereich klickt
kann man den aktuellen Pinsel wechseln (Siehe Abbildung\ref{picture:sculpt_brushes}).
\begin{figure}[h]
    \centering
    \includegraphics[width=.9\textwidth]{SEI_CharakterGestaltung_Arten_Sculpting-Brushes.png}
    \caption{Der Sculpt-Mode von Blender ist über die Listenauswahl im unteren Bereich der 3D-View erreichbar}
    \label{picture:sculpt_brushes}
\end{figure}
Die darunterliegenden Einstellungen bestimmen den Radius
und die Intensität (Strength) des ausgewählten Pinsels. Die
letzte wichtige Einstellung in diesem Bereich ist die
Auswahlmöglichkeit zwischen Add oder Subtract.

"Add" hebt die Fläche an, "Substract" druckt sie ein. (Diese Einstellung ist
nicht bei allen Pinseln gleich). Wenn man zeitsparend und
professionell das Sculpting-Tool verwendet, braucht man den
\textit{Brush} Bereich nie zu öffnen, da alles mit Short-Cuts
kontrollierbar ist. In der Tabelle (Siehe Tabelle\ref{table:sculpt_brushes}) sind die
wichtigsten Pinsel und Einstellung mit Shortcuts ersichtlich.

\begin{longtable}{|p{3.6cm}|p{3.6cm}|p{6.4cm}|}
    \hline
    \endfirsthead
    & \textbf{Pinsel/Einstellung} & \textbf{Shortcut} & \textbf{Beschreibung} & \\
    \hline
    \endhead

    & Verwenden & Linke Maustaste & Wendet den aktuellen Pinsel auf dem Objekt an & \\

    & Subtract & STRG + Linke Maustaste & Wendet die Gegenfunktion des aktuellen Pinsels auf dem Objekt an.
    (Falls der ausgewählte Pinsel eine Gegenfunktion hat) & \\

    & Radius & F & Ermöglicht den Pinselradius einzustellen & \\

    & Strength & Shift + F & Ermöglicht die Intensität des Pinsels einzustellen & \\

    & Clay Strips & 3 & Dieser Pinsel erhöht die Mitte und die Seiten des Pinselbereiches mit einer leichten Abstufung.
    Die Erhöhung ist klar abgegrenzt und der Höhepunkt wird bis zum Rand abgeflacht. Der Rand des Pinsels bildet eine
    Abstufung zum restlichen Objekt.

    Dieser Pinsel kann gut verwendet werden um Haare, Muskeln und Venen darzustellen & \\

    & Flatten & Shift + T & Dieser Pinsel probiert den ausgewählten Bereich aus der eigenen Perspektive auszugleichen,
    sodass er glatt ist. Allerdings sollte man bei diesem Pinsel aufpassen, dass man nicht zu viel Fläche eindrückt,
    da der Pinsel bei jeder Bewegung die vorher abgeflachte Fläche, noch einmal zu glätten und sie daher eindrückt.
    Dadurch wird die ausgewählte Stelle immer tiefer eingedrückt.

    Dieser Pinsel kann gut verwendet werden um Irritationen zu entfernen (Zum Beispiel im Auge oder bei Zähnen) & \\

    & Grab & G & Mit diesem Pinsel kann man einen Bereich des Objektes verschieben. Die naheliegenden Flächen werden
    in abgeschwächt Form mitgezogen, damit das Objekt nicht die Struktur verliert. Die Verschiebung ist immer von der
    eigenen Perspektive abhängig.

    Dieser Pinsel kann gut verwendet werden um Proportionen anzupassen und zu verbessern. (Zum Beispiel, wie sehr die
    Nase absteht.) Allerding muss man aufpassen, dass man keine Shading-Fehler auslöst\ref{picture:shading_fehler} & \\

    & Inflate/Deflate & I & Dieser Pinsel erhebt die ausgewählte Stelle in einer gleichmäßigen Glockenform.

    Dieser Pinsel wird verwendet um Schwellungen, Beulen, Pickel und ähnliche Hautirritationen zu modellieren & \\

    & Mask & M & Der Maskenpinsel verändert das Objekt nicht. Er wird verwendet um bestimmte Stellen von Manipulation
    zu beschützen. Wenn man den Masken-Pinsel aufträgt wird das Objekt schwarz angezeichnet. Das sind die geschützten
    Stellen. Die Maske bleibt erhalten wenn man auf einen anderen Pinsel wechselt.

    Dieser Pinsel wird verwendet, um sehr detailreiche Stellen eines Objektes zu schützen während man einen anderen
    Pinsel sehr großflächig verwenden möchte. Allerdings muss man darauf achten, keine intensiven Veränderungen
    in der Nähe der Maske zu modellieren, da dies zu Shading-Fehlern\ref{picture:shading_fehler} führen kann oder die
    Struktur\ref{picture:struktur_fehler} des Objektes an dieser Stelle stark beschädigt & \\

    & Maske aufheben & ALT + M & Die gezeichnete Maske wird aufgelöst & \\

    & Nudge & - & Dieser Pinsel verschiebt die Flächen in die Richtung, in die man streicht. Der Effekt ist kremig
    und mit absteigendender Stärke zum Rand.

    Dieser Pinsel wird verwendet um Falten oder Kleidung zu modellieren & \\

    & Sculpt & X & Der Sculpt-Pinsel ist dem Inflate Pinsel sehr ähnlich. Jedoch hat der Sculpt Pinsel eine sehr
    viel schwächere Glockenform.

    Dieser Pinsel wird verwendet, um grobe Strukturen einzuzeichnen, die später mit dem Smooth-Pinsel und dem
    Clay Strips Pinsel verfeinert werden & \\

    & Smooth & S & Dieser Pinsel ist dem Flatten Pinsel sehr ähnlich. Während der Flatten-Pinsel aus der eigenen
    Perspektive das Objekt abflacht, ist die Perspektive bei dem Smooth Pinsel nicht von Relevanz. Außerdem probiert
    der Smooth-Pinsel die Flächen abzurunden und näher zueinander zu bringen.

    Dieser Pinsel wird verwendet, um gröbere Hautirritationen zu entfernen, Objektstellen abzurunden und
    glätter / sauberer wirken zu lassen. Der Smooth Pinsel kann auch direkt verwendet werden, ohne das man
    ihn auswählt. Wenn man \keys{Shift + Rechte Maustaste} drückt, wird der Smooth-Pinsel mit den Einstellungen die man auf
    dem Smooth-Pinsel eingestellt hat eingesetzt & \\

    \caption{Einstellungen um ein Bild in der Blender-View als Hintergrund zu importieren}
    \label{table:sculpt_brushes}
\end{longtable}

Shading-Fehler:
Shading Fehler entstehen, wenn Flächen übereinander liegen oder wenn
die innere Seite einer Fläche von außen ersichtlich ist. Shading-Fehler
verursachen, dass das Licht falsch berechnet wird (Siehe Abbildung\ref{picture:shading_fehler}).

\begin{figure}[h]
    \centering
    \includegraphics[width=.9\textwidth]{SEI_CharakterGestaltung_Arten_Sculpting-ShadingFehler.png}
    \caption{Der blau markierte Bereich zeigt einen Shading-Fehler}
    \label{picture:shading_fehler}
\end{figure}


Struktur-Fehler:
Die Struktur wird beschädigt wenn eine größere Position-Veränderung von Flächen passiert, während die unmittelbar
anliegenden Flächen nicht mitbewegt werden. Dadurch erscheinen die Flächen, welche die Verbindung bilden
verhältnismäßig riesig (Siehe Abbildung\ref{picture:struktur_fehler}).

\begin{figure}[h]
    \centering
    \includegraphics[width=.9\textwidth]{SEI_CharakterGestaltung_Arten_Sculpting-StrukturFehler.png}
    \caption{Der blau markierte Bereich zeigt einen Struktur-Fehler}
    \label{picture:struktur_fehler}
\end{figure}



Diese Fehler sind mit dem Smooth-Pinsel leicht zu korrigieren.
Jedoch sind diese Fehler auch leicht zu übersehen.


\textit{Texture} Bereich\citep{blender:tex_mask}:
Im \textit{Texture} Bereich kann man dem Pinsel eine Maske hinzufügen. Eine Maske ist eine schwarz-weiß Textur.
Die Maske bestimmt die Intensität des Pinsels. An den schwarzen Stellen der Textur wird das Objekt nicht
beeinflusst. An den weißen Stellen wird der Pinsels mit eingestellten Intensität beeinflusst.
An den grauen Stellen wird der Prozentsatz des Grauwert in Relation zu der eingestellten
Stärke des Pinsels verwendet. (Siehe Abbildung\ref{picture:sculpt_tex_mask})
(KORR:Mehr Informationen zu dem "Textur" Bereich sind in dem Kapitel Texture_Painting unter Texture Mask)

\begin{figure}[h]
    \centering
    \includegraphics[width=.9\textwidth]{SEI_CharakterGestaltung_Arten_Sculpting-SculptMask.png}
    \caption{Die Auswirkung einer Textur-Maske auf den Sculpt-Pinsel}
    \label{picture:sculpt_tex_mask}
\end{figure}

\textit{Dyntopo} Bereich\citep{blender:dyn_top}:
Der \textit{Dyntopo} Bereich beinhaltet die Schlüsseleinstellungen um detaillierte Objekte zu erstellen.
\textit{Dyntopo} ist die Abkürzung für \textit{Dynamic Topology}. Wenn die dynamische Topologie aktiviert ist,
werden die meisten Pinsel dem Objekt Flächen hinzufügen. Deshalb wird bei der Subdivision Charaktermodellierung
das Sculpt-Tool ohne \textit{Dyntopo} verwendet, da man die Anzahl der Flächen unter Kontrolle halten möchte.
Sobald \textit{Dyntopo} aktiviert wird, wird das Objekt automatische in dreieckige Flächen eingeteilt.
Die Detaileinstellungen können sehr schnell zu viele Flächen generieren, was zum Absturz von Blender führen kann,
daher sollte man bei diesen Einstellungen vorsichtig vorgehen. Bei der Abbildung\ref{picture:dynTopo} wurden nur die
Einstellungen im \textit{Dyntopo} Bereich geändert (Außer bei der Fläche im linken unteren Teil des Bildes, dort
wurde auch der Radius des Pinsels verändert).

\begin{figure}[h]
    \centering
    \includegraphics[width=.9\textwidth]{SEI_CharakterGestaltung_Arten_Sculpting-Dyntopo.png}
    \caption{Darstellung der verschiedenen \textit{Dyntopo} Einstellungen. (BU = Blender Units)}
    \label{picture:dynTopo}
\end{figure}

Die erste Einstellung in dem Bereich \textit{Dyntopo} bestimmt, die Stärke der Unterteilungen. Diese Einstellung
unterteilt sich - abhängig von der Einstellung Detail-Arten - in eine Auflösung in Blender-Einheit, in die
Detailgröße in Prozenten und in die Detailgröße in Pixel.

Die Detail-Art Einstellung unterteilt sich in Constant Detail, Brush Detail und Realtive Detail.

Constant Detail fügt Details auf der gesamten Fläche hinzu (Diese Einstellung ist am Performance-
lastigsten und kann sehr schnell zum Absturz führen).

Brush Detail fügt Flächen in prozentualem Anteil zum eingestellten Radius des Pinsels hinzu.

Relative Detail fügt Details abhängig von dem Radius und dem Abstand zum Objekt
hinzu. Umso näher man ist, desto mehr Details werden hinzugefügt. Umso weiter entfernt man ist, desto
weniger Details. Wenn man mit "Relative Detail" arbeitet, muss man aufpasse, nicht bereits modellierte Details
zu zerstören, weil man mit einem Pinsel zu weit entfernt, an bereits vorhandenen detailreichen Stellen des
Objektes arbeitet.

\textit{Symetry / Lock} Bereich\citep{blender:sym_lock}:
Im \textit{Symetry / Lock} Bereich kann man das symetrische Verhalten einstellen. Standardmäßig werden alle
Manipulationen entlang der X-Achse gespiegelt. Das kann man im Bereich \textit{Symetry / Lock} unter Mirror
deaktivieren. Die anderen Einstellungen wurden für die Modellierung des Antagonisten nicht verwendet.



Umsetzung des Antagonisten:

Der Antagonist wurde zum größten Teil mittels dem Sculpt-Tool erstellt. Die einzige Ausnahmen sind die Hände. Die
Hände wurden zu beginn mittels der Subdivision-Methode modelliert und im Nachhinein mit dem Sculpt Tool verfeinert. Die Teile des
Antagonisten (Kopf - Rüstung - Jacke - Hände - Hose - Schuhe) wurden einzeln erstellt und am Ende zusammengefügt.

Die dynamische Topologie war bei der gesamten Entwicklung aktiviert. Als Detail-Art wurde Relative Detail
verwendet. Die oben genannten Pinsel wurden verwendet um die einzelnen Teile zu modellieren.

Zu Beginn wurde ein Objekt verwendet, welches die annähernde Grundform hat. Beim Kopf war das eine Kugel. Mit
einem Pinsel, der bei aktiver Dyntopo details hinzufügt, wurde das Objekt in mehrere Flächen unterteilt (Wenn die
Stärke des Pinsels auf 0 gesetzt ist, wird das Modell nicht verändert, allerdings werden Flächen hinzugefügt). In Folge
wurde mit dem Grab-Pinsel die Grundform des Kopfes hergestellt. Der Schädel wurde heraus und
das Kinn nach vorne gezogen. Dann wurde mit dem Sculpt-Pinsel das Gesicht vorgezeichnet (Nase, Ohren,
Augen, Augenbrauen, Mund, Kinn, Wangen und Nacken). Ab dann begann das genauere Sculpten, wo regelmäßig zwischen den
verschiedenen Pinseln gewechselt wurde. Nachdem das Gesicht und die Proportionen passten, wurden als letzten Schritt
die Details hinzuzufügen: Narben, Gesichtsfalten, Hautirritationen, Beulen, etc. (Siehe Abbildung\ref{picture:antagonist_process})

\begin{figure}[h]
    \centering
    \includegraphics[width=.9\textwidth]{SEI_CharakterGestaltung_Arten_Sculpting-GWKopfModellierungProzess.png}
    \caption{Darstellung der verschiedenen \textit{Dyntopo} Einstellungen. (BU = Blender Units)}
    \label{picture:antagonist_process}
\end{figure}

Wenn man das Sculpt-Tool von Blender produktiv verwenden möchte, muss man sich in der 3D-Welt bewegen
können. Vorallem wenn man mit relativen Details arbeitet ist es wichtig immer die richtige Perspektive zu erreichen.
Die Tabelle\ref{table:blender_movement}) zeigt die wichtigsten Shortcuts um sich in Blender richtig bewegen zu können.

\begin{longtable}{|p{4cm}|p{9.6cm}|}
    \hline
    \endfirsthead
    & \textbf{Shortcut} & \textbf{Beschreibung} &\\
    \hline
    \endhead

    & MMB & Bewegung um den Sichtpunkt & \\
    & NUMPAD(ENTF) & Setzt den Sichtpunkt auf das ausgewählte Objekt / die ausgewählten Objekte & \\
    & SHIFT + Scrollen & Bewegung um den Sichtpunkt & \\
    & ALT + Scrollen & Bewegt den Sichtpunkt nach rechts / links & \\
    & SHIFT+ MMB & Bewegt den Sichtpunkt nach oben / unten / rechts / links & \\
    & Scrollen & Zoomt grob heran / heraus & \\
    & STRG + MMB & Zoomt fein heran / heraus & \\

    \caption{Shortcuts um sich in Blender einfach zu bewegen. (MMB = Mittlere Maustaste)}
    \label{table:blender_movement}
\end{longtable}

Wenn man sich mit der mittleren Maustaste nur um sich selbst drehen kann und nicht mehr bewegen kann, hat man
so nahe herangezoomt, dass man sich genau auf dem Sichtpunkt befindet. In diesem Fall muss man herauszoomen, bis
man sich wieder bewegen kann.

\subsection{Subdivision Modelling}
\subsection{UV-Unwrapping}
\subsection{Normal Baking}
\subsection{Texture Painting}
