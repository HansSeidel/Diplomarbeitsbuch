\section{Charakter Gestaltung}
Bevor man Blender öffnet und beginnt einen Charakter zu gestalten, ist es wichtig sich ein Bild von
dem Charakter zu machen. Wenn man das erste Mal einen Charakter gestaltet, unterläuft vielen der
Fehler, dass sie den Charakter aus dem Kopf heraus modellieren. Allerdings passiert es dann sehr
schneller, dass man ein Modell entwickelt, welches einem nicht gefällt oder welches nicht
gebrauchen kann. Daher ist es wichtig, den Charakter zuerst auf Papier zu zeichnen. Auch wenn das
Endresultat eines Charakters neutral sein sollte, ist es hilfreich sich bereits im Vorhinein einige
Eigenschaften des Charakters zu überlegen.

Ist der Charakter eher ein grimmiger Mensch oder ein sehr freundlicher? Ist er überhaupt ein
Mensch? Wie kleidet er sich? Wie bewegt er sich? Hat er irgendwelche Verletzung?

Diese Details helfen, den Charakter bereits am Papier detailliert darzustellen und seine Rolle im Spiel
authentischer zu gestalten.

%----------Skizzen zu dem Modell-------Skizzen zu dem Modell-------Skizzen zu dem Modell-------Skizzen zu dem Modell----
\subsection{Skizzen zu dem Modell}

In unserem Fall haben wir die beiden Hauptcharaktere gleich zu Beginn des Objektes charakterisiert.
Mit Beispielbildern und einer Personenbeschreibung war für alle Teammitglieder ein klares Bild der
Charaktere vor Augen. Vor allem wenn man beginnt Charaktere zu modellieren, sollte man sich
Vorbilder nehmen. Niemand wird bei seinem ersten Versuch einen perfekten Charakter modellieren
können. Für Übungszwecke ist es daher hilfreich sich bereits vorhandenen Charakteren als Vorlage
zu verwenden.

Sobald die erste Vorstellung des Charakters entstanden ist, kann man beginnen die Skizzen zu
entwerfen. Die Skizzen sollen ein klareres und einheitlicheres Bild erstellen. Sie transportieren die
Vorstellung des Charakters auf Papier. Für 3D-Entwickler ist es am angenehmsten, wenn die Skizzen ein Modell
(in diesem Fall den Charakter) von drei Seiten anzeigen. Eine Front-Perspektive, eine Right-Perspektive und eine
Top-Perspektive. Für die Entwicklung des Charakters kann man diese Skizzen leicht in Blender einfügen und genau
in diesen Perspektiven anzeigen.



Einfügen von Skizzen in Blender \citep{blender:background_images}:

In der 3D-View von Blender gibt es die Möglichkeit Bilder dem Hintergrund hinzuzufügen. Diese
Bilder sind jedoch nur in der orthographischen Ansicht verfügbar. Die orthographische Ansicht ist
eine zwei Dimensionale Darstellung aus der Sicht einer bestimmten Achse.
Wenn sich der Mauszeiger in der 3D-View befindet und die Taste \keys{N} gedrückt wird, öffnet sich
auf der linken Seite des Fensters die Properties Region. Unter dem Bereich: Background-Images kann
man die Skizzen nun hinzufügen. Sobald man ein Bild hinzugefügt hat, kann dieses mit
den vorhandenen Einstellungen so angepasst werden, dass es am angenehmsten zum Arbeiten ist.
Hier sind einige der wichtigsten Einstellungen kurz beschrieben.


\begin{tabular}{p{3.4cm}|p{3.4cm}|p{3.4cm}|p{3.4cm}}
    \textbf{Einstellung} & \textbf{Beschreibung} \hline
    Axis & Die Achsen Einstellung gibt die Möglichkeit auszuwählen von welcher Seite
    das Bild angezeigt werden soll.    Es wird dann nur angezeigt, wenn man in der
    3D-View sich in dieser Seite befindet.\hline
    Opacity & Die Opacity bestimmt zu wie viel Prozent das Bild sichtbar ist.
    (0 = Komplett transparent, 1 = 1 überhaupt nicht transparent.\hline
    Back/Front & Diese Einstellung bestimmt, ob das Bild das Objekt verdecken soll
    (Front) oder nicht (Back) \hline
    X/Y/Rotation/Size & Mit diesen Einstellungen kann man die Skizzen dann perfekt
    an das Modell anpassen. \hline
\end{tabular}


%----------Menschliche Relationen------Menschliche Relationen------Menschliche Relationen------Menschliche Relationen---
\subsection{Menschliche Relationen}
    Wenn man erst seit kurzen Charaktere modelliert, passiert es sehr schnell, dass die Proportionen
    nicht korrekt sind. Unabhängig davon, ob man Charaktere im 3-Dimensionalen Bereich oder auf Papier
    designt, sollte man einige Proportionsregeln beachten und diese einzeichnen bevor man mit dem
    Modellieren beginnt. (Auch Blender hat eine Möglichkeit zu skizieren. Allerdings wird das nicht in
    diesem Buch behandelt \citep{blender:grease_pencil}.)

Einige Linien in einem erwachsenen Gesicht, stehen immer in einem bestimmten Verhältnis zueinander
(Siehe Abbildung \ref{blender:proportions}). Die Linie \textbf{A} grenzt die Schädeldecke ab. Von
der Schädeldecke bis zum Stirnansatz ist zirka ein Abstand von 1/7 der Gesichtsgröße. Die Linie
\textbf{C} zeigt, dass die Augenbrauen und der Abschluss der Ohren auf gleicher Höhe sein sollten.
\textbf{C.1} verbindet das obere Ende der Nase mit der Mitte des Auges und dem Ansatz der Ohren.
Außerdem sollte die Linie  \textbf{C.1} genau in der Mitte des Gesichts liegen. Auf der Linie
\textbf{D} befindet sich das Ohrläppchen und der Ansatz der Nase. Linie  \textbf{E} stellt das Kinn
und den Abschluss des Gesichtes dar. Da dieser Charakter übergewichtig modelliert wurde, ragt sein
Doppelkinn über die Linie hinaus. Die Linie  \textbf{D.1} zeigt die untere Grenze des Mundes.
Aus der frontalen Ansicht kann man auch vertikale Linien ziehen. Diese würden die Tränendrüse mit den Nasenflügeln
verbinden und die Iris mit den Mundwinkeln. \citep{book:kunst_des_zeichnens}
\begin{figure}[h]
    \centering
    \includegraphics[width=.8\textwidth]{SEI_CharakterGestaltung_MenschlicheRelationen_KopfInRealtionen.png}
    \caption{Proportionen im Gesicht eines Erwachsenen}
    \label{blender:proportions}
\end{figure}

%----------Menschliche Relationen------Menschliche Relationen------Menschliche Relationen------Menschliche Relationen---
\subsection{Arten der Charakter-Modellierung}
\subsection{Sculpting}
\subsection{Grundmodellierung}
\subsection{UV-Unwrapping}
\subsection{Normal Baking}
\subsection{Texture Painting}
