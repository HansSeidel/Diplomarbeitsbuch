\section{Charakter Gestaltung}
Bevor man Blender öffnet und beginnt einen Charakter zu gestalten, ist es wichtig sich ein Bild von
dem Charakter zu machen. Wenn man das erste Mal einen Charakter gestaltet, unterläuft vielen der
Fehler, dass sie den Charakter aus dem Kopf heraus modellieren. Allerdings passiert es dann sehr
schnell, dass man ein Modell entwickelt, welches einem nicht gefällt oder welches man nicht
gebrauchen kann. Daher ist es wichtig, den Charakter zuerst auf Papier zu zeichnen. Auch wenn das
Endresultat eines Charakters neutral sein sollte, ist es hilfreich sich bereits im Vorhinein einige
Eigenschaften des Charakters zu überlegen.

Ist der Charakter eher ein grimmiger Mensch oder ein sehr freundlicher? Ist er überhaupt ein
Mensch? Wie kleidet er sich? Wie bewegt er sich? Hat er irgendwelche Verletzung?

Diese Details helfen, den Charakter bereits am Papier detailliert darzustellen und seine Rolle im Spiel
authentischer zu gestalten.

%----------Skizzen zu dem Modell-------Skizzen zu dem Modell-------Skizzen zu dem Modell-------Skizzen zu dem Modell----
\subsection{Skizzen zu dem Modell}

In unserem Fall haben wir die beiden Hauptcharaktere gleich zu Beginn des Objektes charakterisiert.
Mit Beispielbildern und einer Personenbeschreibung war für alle Teammitglieder ein klares Bild der
Charaktere vor Augen. Vor allem wenn man beginnt Charaktere zu modellieren, sollte man sich
Vorbilder nehmen. Niemand wird bei seinem ersten Versuch einen perfekten Charakter modellieren
können. Für Übungszwecke ist es daher hilfreich bereits vorhandene Charaktere als Vorlage
zu verwenden.
Sobald die erste Vorstellung des Charakters entstanden ist, kann man beginnen die Skizzen zu
entwerfen. Die Skizzen sollen ein klareres und einheitlicheres Bild erstellen. Sie transportieren die
Vorstellung des Charakters auf Papier. Für 3D-Entwickler ist es am angenehmsten, wenn die Skizzen ein Modell
(in diesem Fall den Charakter) von drei Seiten beleuchten. Eine Front-Perspektive, eine Right-Perspektive und eine
Top-Perspektive. Für die Entwicklung des Charakters kann man diese Skizzen leicht in Blender einfügen und genau
in diesen Perspektiven anzeigen.



Einfügen von Skizzen in Blender\citep{blender:background_images}:
In der 3D-View von Blender gibt es die Möglichkeit Bilder dem Hintergrund hinzuzufügen. Diese
Bilder sind jedoch nur in der orthographischen Ansicht verfügbar. Die orthographische Ansicht ist
eine zwei Dimensionale Darstellung aus der Sicht einer bestimmten Achse.
Wenn sich der Mauszeiger in der 3D-View befindet und die Taste \keys{N} gedrückt wird, öffnet sich
auf der linken Seite des Fensters die Properties Region. Unter dem Bereich: Background-Images kann
man die Skizzen nun hinzufügen. Sobald man ein Bild hinzugefügt hat, kann dieses mit
den vorhandenen Einstellungen so angepasst werden, dass es am angenehmsten zum Arbeiten ist.
Hier sind einige der wichtigsten Einstellungen kurz beschrieben.

\raggedbottom
\begin{longtable}{|p{4cm}|p{9.6cm}|}
    \endhead
    \hline
    \textbf{Einstellung} & \textbf{Beschreibung} \\
    \hline
    \endhead

    Axis & Die Achsen Einstellung gibt die Möglichkeit auszuwählen von welcher Seite
    das Bild angezeigt werden soll. Es wird dann nur angezeigt, wenn man sich in der
    3D-View in dieser Seite befindet \\
    \hline
    Opacity & Die Opacity bestimmt zu wie viel Prozent das Bild sichtbar ist. (0 = Komplett transparent, 1 = 1
    überhaupt nicht transparent \\
    \hline
    Back/Front & Diese Einstellung bestimmt, ob das Bild das Objekt verdecken soll (Front) oder nicht (Back) \\
    \hline
    X/Y/Rotation/Size & Mit diesen Einstellungen kann man die Skizzen dann perfekt
    an das Modell anpassen \\
    \hline
    \caption{Einstellungen um ein Bild in der Blender-View als Hintergrund zu importieren}
\end{longtable}


%----------Menschliche Relationen------Menschliche Relationen------Menschliche Relationen------Menschliche Relationen---
\subsection{Menschliche Proportionen}
    Wenn man erst seit kurzen Charaktere modelliert, passiert es sehr schnell, dass die Proportionen
    nicht korrekt sind. Unabhängig davon, ob man Charaktere im 3-Dimensionalen Bereich oder auf Papier
    designt, sollte man einige Proportionsregeln beachten und diese einzeichnen bevor man mit dem
    Modellieren beginnt. (Auch Blender hat eine Möglichkeit zu skizieren. Allerdings wird das nicht in
    diesem Buch behandelt\citep{blender:grease_pencil}.)

Einige Linien in einem erwachsenen Gesicht, stehen immer in einem bestimmten Verhältnis zueinander
(\textit{Abbildung \ref{blender:proportions}}). Die Linie \textbf{A} grenzt die Schädeldecke ab. Von
der Schädeldecke bis zum Stirnansatz ist zirka ein Abstand von 1/7 der Gesichtsgröße. Die Linie
\textbf{C} zeigt, dass die Augenbrauen und der Abschluss der Ohren auf gleicher Höhe sein sollten.
\textbf{C.1} verbindet das obere Ende der Nase mit der Mitte des Auges und dem Ansatz der Ohren.
Außerdem sollte die Linie  \textbf{C.1} genau in der Mitte des Gesichts liegen. Auf der Linie
\textbf{D} befindet sich das Ohrläppchen und der Ansatz der Nase. Linie  \textbf{E} stellt das Kinn
und den Abschluss des Gesichtes dar. Da dieser Charakter übergewichtig modelliert wurde, ragt sein
Doppelkinn über die Linie hinaus. Die Linie  \textbf{D.1} zeigt die untere Grenze des Mundes.
Aus der frontalen Ansicht kann man auch vertikale Linien ziehen. Diese würden die Tränendrüse mit den Nasenflügeln
verbinden und die Iris mit den Mundwinkeln.\citep{book:kunst_des_zeichnens}
\begin{figure}[H]
    \centering
    \includegraphics[width=.8\textwidth]{SEI_CharakterGestaltung_MenschlicheRelationen_KopfInRealtionen.png}
    \caption{Proportionen im Gesicht eines Erwachsenen}
    \label{blender:proportions}
\end{figure}

%----------Arten der Charakter-Modellierung------Arten der Charakter-Modellierung------Arten der Charakter-Modellierung-
\subsection{Arten der Charakter-Modellierung}
Im Laufe der Diplomarbeit wurden bei der Entwicklung der zwei Hauptcharaktere zwei verschiedene Arten der
Charakter-Modellierung verwendet. Der Protagonist wurde mittels der Subdivision-Methode modelliert. Der Antagonist
wurde mittels dem Blender Sculpt Tool\citep{blender:sculpting_general} erstellt .
Bei diesen Methoden gibt es zwei bedeutende Unterschiede.


Ein Charakter, der mittels des Sculpt-Tools modelliert wird kann meistens nicht für ein Spiel verwendet werden. Die
Detailarbeit die bei dem Sculpting durchgeführt wird generiert zu viele Polygone (Flächen) auf dem Charakter. Diese
Anzahl an Flächen kann der Computer nicht berechnen. (Im Falle des Antagonisten sind zirka 1.5 Millionen Flächen und 2
Millionen Punkte entstanden.) Das fertige Modell wird verwendet, um Graphiken zu erstellen, welche die Details des
Modells \textit{faken}. Diese Graphiken werden Normal-Maps genannt
(\textit{Kapitel \ref{sec:tex_normal} Bump und Normal Texturen}).

Bei der Subdivision Modellierung hat der Entwickler eine größere Kontrolle über die Flächenanzahl. Allerdings ist es
dafür schwieriger Details oder Gesichtszüge zu modellieren. Diese müssen im Nachhinein gezeichnet oder entwickelt
werden.

Der zweite Unterschied der Methoden ist die tatsächliche Umsetzung des Charakters. Mit dem Sculpt-Tool von Blender
kann man Objekte modellieren als wären sie aus Ton gemacht. Bei der Subdivision-Methode werden die einzelnen
Körperteile erstellt indem man die Flächen des Objektes einzeln vervielfältigt und manipuliert. Um das genauer zu
erklären werden die Prozesse der verschiedenen Methoden beschrieben, um ein Auge zu modellieren.

Sculpting-Methode:
Mittels verschiedenen Pinseln (\textit{Kapitel \ref{sec:sculpting} Sculpting}) drückt man Teile des Objekte ein,
hebt andere hervor, verschiebt sie, glättet sie ab und flacht sie ab. Während dieser Prozesse werden (mit den richtigen
Einstellungen) so viele Fläche erstellt, wie es die Detailarbeit verlangt.

Subdivision Modellierung:
Man wählt direkt die Flächen aus, die sich bei dem Auge befinden (in den meisten Fällen zwischen 4 und 16 Flächen).
Diese Flächen löscht man. Dann fügt man eine Kugel hinzu und passt sie dem Kopf an. Wenn die Position stimmt,
verbindet man die zwei Objekte und entfernt die überflüssigen (nicht sichtbaren) Flächen. Bei diesem Prozess kann und
muss selbst bestimmt werden, wie viele Flächen entfernt werden und wie viele mittels der Kugel hinzugefügt werden.

Bei der Subdivision Modellierung wird auch gerne das Sculpt-Tool für den letzten Feinschliff verwendet. Es ist auch
möglich das Sculpt-Tool zu verwenden, ohne dass neue Flächen generiert werden
(\textit{Kapitel \ref{sec:sculpting} Sculpting} - \textit{DynTopo}).

%--------------------------Sculpting--------------------------Sculpting--------------------------Sculpting--------------
\subsection{Sculpting}
\label{sec:sculpting}

Das Sculpt-Tool von Blender teilt sich in die Bereiche \textit{Brush, Texture, Stroke, Curve, Dyntopo, Symmetry}
und \textit{History} auf. Die wichtigsten Bereiche, die auch bei der Umsetzung des Antagonisten verwendet wurden sind
unten angeführt.

Wenn man ein Objekt mit dem Sculpt-Tool bearbeiten möchte, ist die Verwendung von Short-Cuts
sehr hilfreich. Erfahrungsgemäß sind Short-Cuts in Blender  der beste Weg um sich mindestens 20\% der Zeit zu
sparen. Beim Sculpten bringen Short-Cuts sogar 40\% der Zeit zurück.

\begin{figure}[H]
    \centering
    \includegraphics[width=.9\textwidth]{SEI_CharakterGestaltung_Arten_Sculpting-SculptMode.png}
    \caption{Der Sculpt-Mode von Blender ist über die Listenauswahl im unteren Bereich der 3D-View erreichbar}
    \label{picture:sculpt_mode}
\end{figure}

\textit{Brush} Bereich\citep{blender:sculpting_brushes}:

Der \textit{Brush} Bereich bestimmt die grundlegenden Einstellungen des Pinsels.
Wenn man auf das Bild in dem Bereich klickt
kann man den aktuellen Pinsel wechseln (\textit{Abbildung \ref{picture:sculpt_brushes}}).
\begin{figure}[H]
    \centering
    \includegraphics[width=.9\textwidth]{SEI_CharakterGestaltung_Arten_Sculpting-Brushes.png}
    \caption{Die verschiedenen Pinsel}
    \label{picture:sculpt_brushes}
\end{figure}
Die darunterliegenden Einstellungen bestimmen den Radius
und die Intensität (Strength) des ausgewählten Pinsels. Die
letzte wichtige Einstellung in diesem Bereich ist die
Auswahlmöglichkeit zwischen Add oder Subtract.

"Add" hebt die Fläche an, "Substract" drückt sie ein. (Diese Einstellung ist
nicht bei allen Pinseln gleich). Wenn man zeitsparend und
professionell das Sculpting-Tool verwendet, braucht man den
\textit{Brush} Bereich nie zu öffnen, da alles mit Short-Cuts
kontrollierbar ist. In der Tabelle (\textit{Tabelle \ref{table:sculpt_brushes}}) sind die
wichtigsten Pinsel und Einstellung mit Shortcuts ersichtlich.

\begin{longtable}{|p{3.6cm}|p{3.6cm}|p{6.4cm}|}
    \hline
    \endfirsthead
    \textbf{Pinsel/Einstellung} & \textbf{Shortcut} & \textbf{Beschreibung} \\
    \hline
    \endhead
    \textbf{Pinsel/Einstellung} & \textbf{Shortcut} & \textbf{Beschreibung} \\
    \hline

    Verwenden & Linke Maustaste & Wendet den aktuellen Pinsel auf dem Objekt an \\
    \hline

    Subtract & STRG + Linke Maustaste & Wendet die Gegenfunktion des aktuellen Pinsels auf dem Objekt an.
    (Falls der ausgewählte Pinsel eine Gegenfunktion hat) \\
    \hline

    Radius & F & Ermöglicht den Pinselradius einzustellen \\
    \hline

    Strength & Shift + F & Ermöglicht die Intensität des Pinsels einzustellen \\
    \hline

    Clay Strips & 3 & Dieser Pinsel erhöht die Mitte und die Seiten des Pinselbereiches mit einer leichten Abstufung.
    Die Erhöhung ist klar abgegrenzt und der Höhepunkt wird bis zum Rand abgeflacht. Der Rand des Pinsels bildet eine
    Abstufung zum restlichen Objekt.

    Dieser Pinsel kann gut verwendet werden um Haare, Muskeln und Venen darzustellen \\
    \hline

    Flatten & Shift + T & Dieser Pinsel probiert den ausgewählten Bereich aus der eigenen Perspektive auszugleichen,
    sodass er glatt ist. Allerdings sollte man bei diesem Pinsel aufpassen, dass man nicht zu viel Fläche eindrückt,
    da der Pinsel bei jeder Bewegung die vorher abgeflachte Fläche noch einmal glättet und sie daher eindrückt.
    Dadurch wird die ausgewählte Stelle immer tiefer eingedrückt.

    Dieser Pinsel kann gut verwendet werden um Irritationen zu entfernen (Zum Beispiel im Auge oder bei Zähnen) \\
    \hline

    Grab & G & Mit diesem Pinsel kann man einen Bereich des Objektes verschieben. Die naheliegenden Flächen werden
    in abgeschwächter Form mitgezogen, damit das Objekt nicht die Struktur verliert. Die Verschiebung ist immer von der
    eigenen Perspektive abhängig.

    Dieser Pinsel kann gut verwendet werden um Proportionen anzupassen und zu verbessern. (Zum Beispiel, wie sehr die
    Nase absteht.) Allerding muss man aufpassen, dass man keine Shading-Fehler auslöst \textit{Abbildung \ref{picture:shading_fehler}} \\
    \hline

    Inflate/Deflate & I & Dieser Pinsel erhebt die ausgewählte Stelle in einer gleichmäßigen Glockenform.

    Dieser Pinsel wird verwendet um Schwellungen, Beulen, Pickel und ähnliche Hautirritationen zu modellieren \\
    \hline

    Mask & M & Der Maskenpinsel verändert das Objekt nicht. Er wird verwendet um bestimmte Stellen vor Manipulation
    zu schützen. Wenn man den Masken-Pinsel aufträgt wird das Objekt schwarz angezeichnet. Das sind die geschützten
    Stellen. Die Maske bleibt erhalten wenn man auf einen anderen Pinsel wechselt.

    Dieser Pinsel wird verwendet, um sehr detailreiche Stellen eines Objektes zu bewahren während man einen anderen
    Pinsel sehr großflächig verwenden möchte. Allerdings muss man darauf achten, keine intensiven Veränderungen
    in der Nähe der Maske zu modellieren, da dies zu Shading-Fehlern \textit{Abbildung \ref{picture:shading_fehler}} führen kann oder die
    Struktur \textit{Abbildung \ref{picture:struktur_fehler}} des Objektes an dieser Stelle stark beschädigt \\
    \hline

    Maske aufheben & ALT + M & Die gezeichnete Maske wird aufgelöst \\
    \hline

    Nudge & - & Dieser Pinsel verschiebt die Flächen in die Richtung, in die man streicht. Der Effekt ist kremig
    und mit absteigendender Stärke zum Rand.

    Dieser Pinsel wird verwendet um Falten oder Kleidung zu modellieren \\
    \hline

    Sculpt & X & Der Sculpt-Pinsel ist dem Inflate Pinsel sehr ähnlich. Jedoch hat der Sculpt Pinsel eine sehr
    viel schwächere Glockenform.

    Dieser Pinsel wird verwendet, um grobe Strukturen einzuzeichnen, die später mit dem Smooth-Pinsel und dem
    Clay Strips Pinsel verfeinert werden \\
    \hline

    Smooth & S & Dieser Pinsel ist dem Flatten Pinsel sehr ähnlich. Während der Flatten-Pinsel aus der eigenen
    Perspektive das Objekt abflacht, ist die Perspektive bei dem Smooth Pinsel nicht von Relevanz. Außerdem probiert
    der Smooth-Pinsel die Flächen abzurunden und näher zueinander zu bringen.

    Dieser Pinsel wird verwendet, um gröbere Hautirritationen zu entfernen, Objektstellen abzurunden und
    glätter / sauberer wirken zu lassen. Der Smooth Pinsel kann auch direkt verwendet werden, ohne dass man
    ihn auswählt. Wenn man \keys{Shift + Rechte Maustaste} drückt, wird der Smooth-Pinsel mit den Einstellungen die man auf
    dem Smooth-Pinsel eingestellt hat eingesetzt \\
    \hline

    \caption{Die einzelnen Pinzel des Sculpting-Tools}
    \label{table:sculpt_brushes}
\end{longtable}

Shading-Fehler:

Shading Fehler entstehen, wenn Flächen übereinander liegen oder wenn
die innere Seite einer Fläche von außen ersichtlich ist. Shading-Fehler
verursachen, dass das Licht falsch berechnet wird (\textit{Abbildung \ref{picture:shading_fehler}}).

\begin{figure}[H]
    \centering
    \includegraphics[width=.9\textwidth]{SEI_CharakterGestaltung_Arten_Sculpting-ShadingFehler.png}
    \caption{Der blau markierte Bereich zeigt einen Shading-Fehler}
    \label{picture:shading_fehler}
\end{figure}


Struktur-Fehler:

Die Struktur wird beschädigt wenn eine größere Position-Veränderung von Flächen passiert, während die unmittelbar
anliegenden Flächen nicht mitbewegt werden. Dadurch erscheinen die Flächen, welche die Verbindung bilden
verhältnismäßig riesig (\textit{Abbildung \ref{picture:struktur_fehler}}).

\begin{figure}[H]
    \centering
    \includegraphics[width=.9\textwidth]{SEI_CharakterGestaltung_Arten_Sculpting-StrukturFehler.png}
    \caption{Der blau markierte Bereich zeigt einen Struktur-Fehler}
    \label{picture:struktur_fehler}
\end{figure}



Diese Fehler sind mit dem Smooth-Pinsel leicht zu korrigieren.
Jedoch sind diese Fehler auch leicht zu übersehen.


\textit{Texture} Bereich\citep{blender:tex_mask}:

Im \textit{Texture} Bereich kann man dem Pinsel eine Maske hinzufügen. Eine Maske ist eine schwarz-weiß Textur.
Die Maske bestimmt die Intensität des Pinsels. An den schwarzen Stellen der Textur wird das Objekt nicht
beeinflusst. An den weißen Stellen wird der Pinsels mit eingestellten Intensität beeinflusst.
An den grauen Stellen wird der Prozentsatz des Grauwerts in Relation zu der eingestellten
Stärke des Pinsels verwendet. (\textit{Abbildung \ref{picture:sculpt_tex_mask}})


\begin{figure}[H]
    \centering
    \includegraphics[width=.9\textwidth]{SEI_CharakterGestaltung_Arten_Sculpting-SculptMask.png}
    \caption{Die Auswirkung einer Textur-Maske auf den Sculpt-Pinsel}
    \label{picture:sculpt_tex_mask}
\end{figure}

\textit{Dyntopo} Bereich\citep{blender:dyn_top}:

Der \textit{Dyntopo} Bereich beinhaltet die Schlüsseleinstellungen um detaillierte Objekte zu erstellen.
\textit{Dyntopo} ist die Abkürzung für \textit{Dynamic Topology}. Wenn die dynamische Topologie aktiviert ist,
werden die meisten Pinsel dem Objekt Flächen hinzufügen. Deshalb wird bei der Subdivision Charaktermodellierung
das Sculpt-Tool ohne \textit{Dyntopo} verwendet, da man die Anzahl der Flächen unter Kontrolle halten möchte.
Sobald \textit{Dyntopo} aktiviert wird, wird das Objekt automatisch ein dreieckige Flächen eingeteilt.
Die Detaileinstellungen können sehr schnell zu viele Flächen generieren, was zum Absturz von Blender führen kann,
daher sollte man bei diesen Einstellungen vorsichtig vorgehen. Bei der Abbildung (\textit{Abbildung \ref{picture:dynTopo}}) wurden nur die
Einstellungen im \textit{Dyntopo} Bereich geändert (Außer bei der Fläche im linken unteren Teil des Bildes, dort
wurde auch der Radius des Pinsels verändert).

\begin{figure}[H]
    \centering
    \includegraphics[width=.9\textwidth]{SEI_CharakterGestaltung_Arten_Sculpting-Dyntopo.png}
    \caption{Darstellung der verschiedenen \textit{Dyntopo} Einstellungen. (BU = Blender Units)}
    \label{picture:dynTopo}
\end{figure}

Die erste Einstellung in dem Bereich \textit{Dyntopo} bestimmt, die Stärke der Unterteilungen. Diese Einstellung
unterteilt sich - abhängig von der Einstellung Detail-Arten - in eine Auflösung in Blender-Einheit, in die
Detailgröße in Prozenten und in die Detailgröße in Pixel.

Die Detail-Art Einstellung unterteilt sich in Constant Detail, Brush Detail und Realtive Detail.

Constant Detail fügt Details auf der gesamten Fläche hinzu (Diese Einstellung ist am Performance-
lastigsten und kann sehr schnell zum Absturz führen).

Brush Detail fügt Flächen in prozentualem Anteil zum eingestellten Radius des Pinsels hinzu.

Relative Detail fügt Details abhängig von dem Radius und dem Abstand zum Objekt
hinzu. Umso näher man ist, desto mehr Details werden hinzugefügt. Umso weiter entfernt man ist, desto
weniger Details. Wenn man mit "Relative Detail" arbeitet, muss man aufpassen, nicht bereits modellierte Details
zu zerstören, weil man mit einem Pinsel zu weit entfernt, an bereits vorhandenen detailreichen Stellen des
Objektes arbeitet.

\textit{Symetry / Lock} Bereich\citep{blender:sym_lock}:

Im \textit{Symetry / Lock} Bereich kann man das symetrische Verhalten einstellen. Standardmäßig werden alle
Manipulationen entlang der X-Achse gespiegelt. Das kann man im Bereich \textit{Symetry / Lock} unter Mirror
deaktivieren. Die anderen Einstellungen wurden für die Modellierung des Antagonisten nicht verwendet.



Umsetzung des Antagonisten:

Der Antagonist wurde zum größten Teil mittels dem Sculpt-Tool erstellt. Die einzige Ausnahme sind die Hände. Die
Hände wurden zu Beginn mittels der Subdivision-Methode modelliert und im Nachhinein mit dem Sculpt Tool verfeinert. Die Teile des
Antagonisten (Kopf - Rüstung - Jacke - Hände - Hose - Schuhe) wurden einzeln erstellt und am Ende zusammengefügt.

Die dynamische Topologie war bei der gesamten Entwicklung aktiviert. Als Detail-Art wurde Relative Detail
verwendet. Die oben genannten Pinsel wurden verwendet um die einzelnen Teile zu modellieren.

Zu Beginn wurde ein Objekt verwendet, welches die annähernde Grundform hat. Beim Kopf war das eine Kugel. Mit
einem Pinsel, der bei aktiver Dyntopo details hinzufügt, wurde das Objekt in mehrere Flächen unterteilt (Wenn die
Stärke des Pinsels auf 0 gesetzt ist, wird das Modell nicht verändert, allerdings werden Flächen hinzugefügt). In Folge
wurde mit dem Grab-Pinsel die Grundform des Kopfes hergestellt. Der Schädel wurde heraus und
das Kinn nach vorne gezogen. Dann wurde mit dem Sculpt-Pinsel das Gesicht vorgezeichnet (Nase, Ohren,
Augen, Augenbrauen, Mund, Kinn, Wangen und Nacken). Ab dann begann das genauere Sculpten, wo regelmäßig zwischen den
verschiedenen Pinseln gewechselt wurde. Nachdem das Gesicht und die Proportionen passten, wurden als letzter Schritt
die Details hinzugefügt: Narben, Gesichtsfalten, Hautirritationen, Beulen, etc. (\textit{Abbildung \ref{picture:antagonist_process}})

\begin{figure}[H]
    \centering
    \includegraphics[width=.9\textwidth]{SEI_CharakterGestaltung_Arten_Sculpting-GWKopfModellierungProzess.png}
    \caption{Der Prozess des Sculpten anhand des Kopfes}
    \label{picture:antagonist_process}
\end{figure}

Wenn man das Sculpt-Tool von Blender produktiv verwenden möchte, muss man sich in der 3D-Welt bewegen
können. Vorallem wenn man mit relativen Details arbeitet ist es wichtig immer die richtige Perspektive zu erreichen.
Die Tabelle (\textit{Tabelle \ref{table:blender_movement}}) zeigt die wichtigsten Shortcuts um sich in Blender richtig bewegen zu können.

\begin{longtable}{|p{4cm}|p{9.6cm}|}
    \hline
    \endfirsthead
    \textbf{Shortcut} & \textbf{Beschreibung}\\
    \hline
    \endhead

    MMB & Bewegung um den Sichtpunkt \\
    \hline
    NUMPAD(ENTF) & Setzt den Sichtpunkt auf das ausgewählte Objekt / die ausgewählten Objekte \\
    \hline
    SHIFT + Scrollen & Bewegung um den Sichtpunkt \\
    \hline
    ALT + Scrollen & Bewegt den Sichtpunkt nach rechts / links \\
    \hline
    SHIFT+ MMB & Bewegt den Sichtpunkt nach oben / unten / rechts / links \\
    \hline
    Scrollen & Zoomt grob heran / heraus \\
    \hline
    STRG + MMB & Zoomt fein heran / heraus \\
    \hline

    \caption{Shortcuts um sich in Blender einfach zu bewegen. (MMB = Mittlere Maustaste)}
    \label{table:blender_movement}
\end{longtable}

Wenn man sich mit der mittleren Maustaste nur um sich selbst drehen kann und nicht mehr bewegen kann, hat man
so nahe herangezoomt, dass man sich genau auf dem Sichtpunkt befindet. In diesem Fall muss man herauszoomen, bis
man sich wieder bewegen kann.



%---------------Subdivision Modelling--------------Subdivision Modelling---------------Subdivision Modelling------------
\subsection{Subdivision Modelling}
\label{sec:subdivision}
Bei der Subdivision-Methode werden einige Modifikatoren, Modi und Tools verwendet, die unten angeführt sind:

Edit Modus\citep{blender:edit_mode}:

Der Edit Modus in Blender wird verwendet um ein Grundobjekt zu manipulieren. Entgegengesetzt zum Objekt Modus - in
welchem man die Größe, Position und Rotation verändern kann - kann man in dem Edit-Modus die Flächen, Punkte und Kanten
eines Objektes manipulieren.

Loop-Cuts\citep{blender:loop_cut_slide}:

Ein Loop-Cut wird eine umrundete Unterteilung eines Objektes genannt. Bei quadratischen Flächen ist ein Loop-Cut leicht
zu erstellen. Flächen die von der Kantenanzahl unterschiedlich sind können keinen durchgängigen Loop-Cut durchführen.
Standardmäßig wird der Loop-Cut in Mitte des Objektes gesetzt. Durch die Tastenkombination \keys{STRG+R},
wird ein Loop-Cut gesetzt. Mittels dem Mausrad wird die Anzahl der Unterteilungen verändert. Die rechte Maustaste
platziert den Loop-Cut / die Loop-Cuts in kontinuierlichen Abständen; beginnend in der Mitte. Die linke Maustaste
platziert den Loop-Cut an der aktuellen Position. Wenn ein Loop-Cut an einer komplexen Stelle gesetzt wurde kann dieser
mit \keys{ALT + Rechtsklick} markiert werden und mit \keys{2xG} im relativen Weg zu dem Objekt verschoben werden.
(\textit{Abbildung \ref{picture:loop_cut_slide}})

\begin{figure}[H]
    \centering
    \includegraphics[width=.9\textwidth]{SEI_CharakterGestaltung_Arten_haendisch-LoopCuts.png}
    \caption{Verschiedene Loop-Cuts bei einfachen und komplexen Objekten}
    \label{picture:loop_cut_slide}
\end{figure}


Subdivision Surface Modifikator\citep{blender:mod_subd}:

Der Subdivision Surface Modifikator fügt einem Objekt eine bestimmte Iteration an Flächen hinzu. Dadurch wird ein
glatter und detaillierter Effekt erzielt.


Mirror Modifikator\citep{blender:mod_mirror}:
Der Mirror-Modifikator spiegelt das Objekt in einer bestimmten Achse um den Ursprung des Objektes.
(Der Ursprung des Objektes ist der orange kleine Punkt. Der Ursprung kann mit \keys{STRG + SHIFT + ALT + C} neu gesetzt werden.)

Extrudieren\citep{blender:extrude}:
Extrudieren bedeutet, dass man eine Kopie einer Auswahl erstellt und diese automatisch mit der Auswahl verbindet.
Wenn man einen Punkt auswählt und extrudiert (Shortcut \keys{E}), wird ein weiterer Punkt und eine Kante erstellt.
Bei einer Kante, eine weitere Kante und eine Fläche, bei einer Fläche mit vier Punkten werden 5 weitere Flächen erstellt
(\textit{Abbildung \ref{picture:extrude}}).

\begin{figure}[H]
    \centering
    \includegraphics[width=.9\textwidth]{SEI_CharakterGestaltung_Arten_haendisch-Extruding.png}
    \caption{Die verschiedenen Möglichkeiten zu extrudieren}
    \label{picture:extrude}
\end{figure}


Proportional Edit\citep{blender:prop_editing}:
Beim proportional Editing werden die Punkte in einem bestimmten Bereich ebenfalls von einer Manipulation beeinflusst.
Die Stärke und Art der Beeinflussung wird selbst bestimmt.


Wenn man den Charakter mittels der Subdivision-Methode modelliert beginnt man in den meisten Fällen mit einem Würfel.
Mittles Loop-Cut and Slide halbiert man den Würfel und entfernt eine Hälfte des Würfels. Man fügt dem Objekt einen
Mirror-Modifikator hinzu, um nur eine Seite des Charakters zu modellieren. Dann erstellt man die Grund-Struktur des
Körpers (\textit{Abbildung \ref{picture:base_body}}).

\begin{figure}[H]
    \centering
    \includegraphics[width=.9\textwidth]{SEI_CharakterGestaltung_Arten_haendisch-baseCharakter.png}
    \caption{Der erste Schritt der Subdivision-Methode}
    \label{picture:base_body}
\end{figure}

Nachdem man die Grund-Struktur hat, kann man den Subdivision-Modifikator auf dem Objekt anwenden. Im Edit-Modus
modelliert man dann diverse Details in das Modell. Um die Details zu modellieren ohne die Struktur des Objektes
zu beschädigen kann man Proportional Edit aktivieren. Bei der Entwicklung des Hauptcharakters wurde das Gesicht
nur schemenhaft modelliert, da das Spiel in der Ego-Perspektive umgesetzt wird (\textit{Abbildung \ref{picture:protagonist}}).

\begin{figure}[H]
    \centering
    \includegraphics[width=.9\textwidth]{SEI_CharakterGestaltung_Arten_haendisch-protagonist.png}
    \caption{Der Protagonist mit dem Subdivision Surface Modifikator}
    \label{picture:protagonist}
\end{figure}


%---------------------UV-Unwrapping--------------------UV-Unwrapping---------------------UV-Unwrapping------------------
\subsection{UV-Unwrapping}
\label{sec:unwrapping}

Um einen Charakter zu Texturieren, muss man diesen als erstes \textit{unwrappen}. Unwrappen bedeutet, dass man die Flächen
eines Objektes aufklappt und zweidimensional darstellt. Blender bietet mehrere Möglichkeiten an, ein Objekt zu
\textit{unwrappen}. Die meist verwendeten sind "Smart UV-Projekt"\citep{blender:smart_uv} und
"Unwrap"\citep{blender:unwrap}. Für das Projekt wurde die "Unwrap" Methode verwendet. Um eine UV-Layout
(die zweidimensionale Darstellung der Objektflächen) mittels der Unwrap Methode zu erstellen, müssen zuerst sogenannte
Seams erstellt werden\citep{blender:seams}. Seams werden erstellt, indem man Kanten des Objektes selektiert und mittels
dem UV-Menü (Shortcut \keys{U}) "Mark Seam" auswählt. Wenn Seams gesetzt wurden, werden diese Kanten Rot dargestellt.
Wenn man dann das Objekt mit "Unwrap" - ebenfalls in dem UV-Menü - aufklappt, werden die durch Seams abgetrennten
Flächen auch in dem UV-Layout abgetrennt.

Dieses Layout ist an das Mesh gebunden. Das heißt, dass das UV-Layout nach einem Export und Import vorhanden bleibt.
UV-Layouts müssen in Blender erstellt werden, da Unreal-Engine keine Möglichkeit bietet UV-Layouts zu erstellen oder
zu bearbeiten. Die Seams sollten bedacht gesetzt werden, um die Übersicht nicht zu verlieren. UV-Layouts können
mittels dem Image/UV-Editor noch bearbeitet werden, damit keine Flächen übereinanderliegen (In diesem Fall
würden sie sich die gleichen Pixel teilen). Falls man eine UV-Map in einem Fotobearbeitungsprogramm bearbeiten möchte,
sollte man die einzelnen Teile des Objektes klar ersichtlich in der UV-Map gestalten, damit zuordenbar, welcher Bereich
der UV-Map zu welchen Flächen des Objektes gehört.

Bei der Umsetzung der Charaktere, wurde für jeden Teil ein eigenes UV-Layouts generiert. Es gibt auch die Möglichkeit,
ein UV-Layout für den gesamten Charakter zu erstellen, wodurch weniger Materials benötigt werden. Jedes Material kann
nur eine UV-Map texturieren (\textit{Abbildung \ref{picture:uv_layouts}}).

\begin{figure}[H]
    \centering
    \includegraphics[width=.9\textwidth]{SEI_CharakterGestaltung_UV_Unwrapping-UV-Layouts.PNG}
    \caption{Die UV Layouts des Antagonisten}
    \label{picture:uv_layouts}
\end{figure}

Bei High-Resolution Modellen, wie dem Antagonisten wird kein UV-Layout erstellt. Das UV-Unwrapping folgt in solchen
Fällen etwas später (\textit{Kapitel \ref{sec:normal_baking} Normal Baking}).


%---------------------Normal Baking--------------------Normal Baking---------------------Normal Baking------------------
\subsection{Normal Baking}
\label{sec:normal_baking}

Das Modell des Antagonisten ist zu detailreich modelliert um es für ein Spiel zu verwenden. Da der Charakter mit dem
Sculpt-Tool von Blender und der aktivierten \textit{Dyntopo} Einstellung (\textit{Kapitel \ref{sec:sculpting} Sculpting} - \textit{DynTopo})modelliert
wurde, sind über eine Million
Flächen entstanden. Dieses Modell des Antagonisten wird als High-Resolution Modell bezeichnet. Um einen verwendbaren
Charakter zu erstellen muss man aus dem High-Resolution Modell ein Low-Resolution Modell erstellen und dann die
Details mittels einer Normal-Textur \textit{gefälscht} darstellen (Mehr Informationen zu Normal Texturen im \textit{Kapitel \ref{sec:tex_normal} Bump und Normal Texturen}).
Um ein Low-Resolution Modell des Antagonisten zu bekommen haben wir den Decimate-Modifikator verwendet\citep{blender:mod_decimate}.

Der Decimate-Modifikator besitzt drei primäre Einstellungen:

Collapse:

In diesem Modus werden Punkte miteinander in der Mitte verbunden und dadurch Flächen dezimiert. Die Form des Objektes
kann dadurch sehr stark beeinflusst werden.

Un-Subdived:

Diese Methode löscht Kanten, die das Modell unkreisen. Es ist der gegenteilige Prozess von Loop-Cuts (\textit{Kapitel \ref{sec:subdivision} Subdivision Modelling}).
Daher würde diese Methode nichts bei dem Antagonisten bewirken, weil er durch die \textit{Dyntopo} Einstellung aus
Dreiecken besteht.

Planar
In dieser Einstellung kann man einen Winkel bestimmen, in welchem die Flächen verbunden werden sollen.
Unterscheiden sich zwei Flächen von ihrer Rotation unter X$^\circ$, werden sie zu einer Fläche verbunden.


Die Planar Einstellung dezimiert das Modell am besten. Die Form bleibt relativ in Takt und die Flächen werden auf eine
geringere Anzahl reduziert. Allerdings würde das Low-Resolution Modell nicht für Animationen geeignet sein, wenn die
Planar Einstellung verwendet wird, da sich die Flächen die sich entlang des Ellbogens und des Armes befinden ebenfalls
verbinden würden. Dadurch würde man den Arm nicht mehr korrekt animieren können. Für den Antagonisten wurde die
Collapse Methode verwendet. Eine optimale Flächenanzahl befindet sich zwischen 1000 und 2000 Flächen. Das garantiert
einen flüssigen Programmablauf, die Möglichkeit gute Animationen zu gestalten und mittels UV-Unwrapping ein UV-Layout
mit relativ wenig Aufwand zu gestalten.

Nachdem man das Objekt verringert hat, sollte noch im Edit-Modus kontrolliert werden, ob noch obsulete Flächen entfernt
werden können (\textit{Abbildung \ref{picture:antagonist_decimated}}). Wenn man das fertige Low-Resolution Modell hat,
erstellt man ein UV-Layout (\textit{Kapitel \ref{sec:unwrapping} UV-Unwrapping}).
Das UV-Layout wird benötigt, damit eine Normal-Textur (\textit{Kapitel \ref{sec:tex_normal} Bump und Normal Texturen}) erstellt werden kann.

\begin{figure}[H]
    \centering
    \includegraphics[width=.9\textwidth]{SEI_CharakterGestaltung_UV_Normal_Baking-LowHighPoli.png}
    \caption{Die beiden Modelle auf der linken Seite sind mittels des Decimate-Modifikators reduziert worden. Die Modelle auf der rechten Seite sind nicht reduziert}
    \label{picture:antagonist_decimated}
\end{figure}



Nun kann man die Normals des High-Resolution Modells generieren. Dafür müssen das High-Resolution und Low-Resolution
Modell die exakt gleiche Position in der 3D-Welt haben. Im Properties Modus unter dem Bereich: "Bake" wählt man als
Bake Mode: "Normals" aus. Der Normal Space wird auf: "Tangent" gesetzt. Das Auswahlkästchen "Selected to Active" muss
noch angehakt werden (\textit{Abbildung \ref{picture:baking_settings}}).

\begin{figure}[H]
    \centering
    \includegraphics[width=.9\textwidth]{SEI_CharakterGestaltung_UV_Normal_Baking-bakeSection.png}
    \caption{Die Einstellungen um eine Normal-Textur aus einem High-Resolution Modell zu generiern}
    \label{picture:baking_settings}
\end{figure}


Die übrigen Einstellungen sind standardmäßig korrekt für diesen Prozess. Mit der Taste "Bake",
wird die Normal-Textur erstellt. In dem Fesnter "UV/Image Editor" kann man überprüfen ob alles geklappt hat und die
Normal-Textur abspeichern.

\begin{figure}[H]
    \centering
    \includegraphics[width=.9\textwidth]{SEI_CharakterGestaltung_UV_Normal_Baking-Normal_Maps.png}
    \caption{Die erstellten Normal-Texturen des Antagonisten}
    \label{picture:antagonist_normals}
\end{figure}




%-----------------Texture Painting-----------------Texture Painting------------------Texture Painting-------------------
\subsection{Texture Painting}
\label{sec:texture_painting}

Bevor man mit der Texturing des Charakters beginnt sollte man sich überlegen welchen Stil man umsetzen möchte. Um
einen fotorealistischen Stil zu erreichen, ist die beste Möglichkeit, sich das UV-Layout zu exportieren und mittels
Fotografien, Körperteile und Kleidung auf das UV-Layout zu übertragen.
Eine andere Möglichkeit bietet der Blender Texture-Paint Modus\citep{blender:tex_painting}. Dieser wird eher selten
für fotorealistische Texturierung verwendet. Obwohl wir einen fotorealistischen Stil anstreben, haben wir trotzdem
den Blender Texture Paint Modus verwendet, da wir alle Texturen im Laufe dieser Diplomarbeit selbst erstellt haben.
Die erste Alternative hätte erfordert, dass wir das passende Gewand kaufen, abfotografieren und dann dem UV-Layout
anpassen. Ein passendes Gesicht für den Antagonisten zu finden und abzufotografieren, wäre die echte Herausforderung
geworden.

Um das Texture-Painting Tool von Blender gut zu verwenden, ist es praktisch den UV/Image Editor neben der 3D-View
geöffnet zu haben (\textit{Abbildung \ref{picture:tex_painting_windows}}).

\begin{figure}[H]
    \centering
    \includegraphics[width=.9\textwidth]{SEI_CharakterGestaltung_Textur_Painting-Layout.png}
    \caption{Empfohlene Fensteraufteilung im Texture-Paint Modus}
    \label{picture:tex_painting_windows}
\end{figure}

Bevor man beginnt Texturen zu zeichnen muss der Charakter ein fertiges UV-Layout und ein Material mit einer Textur
besitzen. Das Material-Prinzip von Blender \citep{blender:materials_blender} wird in diesem Buch nicht behandelt.
Die Materialien für dieses Projekt wurden in der Unreal-Engine erstellt (\textit{Kapitel \ref{sec:materials} Materialien}).
Weitere Voreinstellungen sollten in dem "Slots" Bereich vorgenommen werden:

\begin{figure}[H]
    \centering
    \includegraphics[width=.9\textwidth]{SEI_CharakterGestaltung_Textur_Painting-Slots.png}
    \caption{Der "Slots" Bereich }
    \label{picture:tex_painting_slots}
\end{figure}

Slots Einstellungen:


    Painting Mode:

    Der Painting Mode wird auf Image gewechselt.


    Canvas Image:

    Um das korrekte Canvas Image auszuwählen muss es zuerst erstellt werden.
    Dafür wird das Modell im Edit-Modus markiert, damit das UV-Layout
    angezeigt wird. Dann klickt man im UV-/Image Editor auf das  Plus neben
    dem "Image-Icon". Wenn man die gewünschte Resolution eingestellt hat klickt man
    auf okay (\textit{Abbildung \ref{picture:new_image}}).


    \begin{figure}[H]
        \centering
        \includegraphics[width=.9\textwidth]{SEI_CharakterGestaltung_Textur_Painting-NeuesBild.png}
        \caption{Ein neues Bild in Blender hinzufügen}
        \label{picture:new_image}
    \end{figure}


    Über den Bereich "Image" -> "Save Image", kann man das Bild abspeichern.
    Im Texture-Bereich von Blender kann man dieses Image dann importieren (Am
    besten gleich auf dem Material, welches man zum Zwecke des Texture-Paintings
    erstellt hat). Nach diesen Schritten muss unter der Einstellung Canvas Image
    direkt die richtige Textur ausgewählt werden.


    Save All Images:

    Dieser Button speichert alle Veränderungen in den Bild-Dateien, die auf den Texturen
    gezeichnet wurden.


Sobald die Voreinstellungen vorhanden sind, kann man auch schon beginnen die Textur zu zeichnen. Um auch wirklich das
erwünschte Ziel zu erreichen, gibt es in dem "Tools" Bereich die Einstellungen, die den Pinsel beeinflussen:


"Tools" Bereich:

    Brush:
    \begin{longtable}{|p{4cm}|p{9.6cm}|}
        \hline
        \endfirsthead
        \textbf{Pinsel/Einstellung} & \textbf{Beschreibung}\\
        \hline
        \endhead

        Fill & Befüllt die ausgewählte Textur mit einer Basis-Farbe \\
    \hline
        Smear & Dieser Pinsel verschmiert die Farben ineinander. Er wird verwendet um einen sanften Farbübergang zu gestalten \\
    \hline
        Soften & Dieser Pinsel lässt die Pixel verschwimmen. Er wird verwendet um Farbkanten abzurunden \\
    \hline
        TexDraw & Dieser Pinsel zeichnet eine ausgewählte Farbe auf Textur \\
    \hline

        \caption{Pinsel des Texture Painting Modus}
        \label{table:tex_painting_brushes}
    \end{longtable}

    Blend:

    Der Blend Type bestimmt, welche Methode der Pinsel verwendet um Farbe hinzuzufügen. Es wurden nur Mix und Add für
    das Projekt verwendet.

        Mix:

        Überdeckt die Farben. Die stärkere Farbe bleibt daher mehr erhalten als die schwächere
        (abhängig von den Farbwerten).

        Add:

        Fügt eine Farbe der vorhandenen hinzu. Gut verwendbar mit der traditionellen Farbenlehre (Auf eine rote Fläche
        wird mittels der Add-Funktionen Grün hinzugefügt. Das Ergebnis ist Gelb).
        Die Add Funktion ist auch sehr hilfreich um eine Property-Map zu erstellen.
        (Mehr Informationen zu einer Property gibt es in dem \textit{Kapitel \ref{sec:tex_inside_engine} Texturen in der Engine})



    Texture:

    In dem Texture-Bereich gibt es die Möglichkeit eine Textur hinzuzufügen, die als Vorlage für den Pinsel dient.
    Wenn man auf dem Objekt zeichnet, wird nicht die ausgewählte Farbe, sondern die Textur gezeichnet. Diese
    Einstellung kann gut für Sommersprossen verwendet werden.


    Texture Mask:

    Die Texture Mask Einstellung hat die gleiche Auswirkung wie der Texture Bereich in dem \textit{Kapitel \ref{sec:sculpting} Sculpting} - \textit{Texture}.
    Es wird ein bestimmter Bereich der Zeichenfläche nicht verwendet. Das ist von der geladenen Textur abhängig.
    Die geladenen Textur nennt man eine Maske (Masken sind in den meisten Fällen schwarz-weiß Texturen).
    Eine Maske zu erstellen ist relativ einfach. Dafür kann ein beliebiges Fotoprogramm oder Blender selbst verwendet
    werden. Im UV-/ Image Editor muss ein neues schwarzes Bild erstellt werden. Nachdem das getan wurde, wechselt man
    auf den Paint - Mode (\textit{Abbildung \ref{picture:paint_mode}}).

    \begin{figure}[H]
        \centering
        \includegraphics[width=.9\textwidth]{SEI_CharakterGestaltung_Textur_Painting-MaskeErstellen.png}
        \caption{Empfohlene Fensteraufteilung im Texture-Paint Modus}
        \label{picture:paint_mode}
    \end{figure}

    Um dann eine Maske zu zeichnen ist es nur wichtig, dass man nur mit Grauwerten arbeitet. Dafür ist es hilfreich,
    die Farbauswahl des Pinsel mittels dem HSV - Farbcode zu bestimmen. Der H-Wert (Hue) und S-Wert (Saturation) wird
    auf null gesetzt. Der V-Wert (Value) bestimmt die Graustufe (\textit{Abbildung \ref{picture:hsv_color_code}}).

    \begin{figure}[H]
        \centering
        \includegraphics[width=.9\textwidth]{SEI_CharakterGestaltung_Textur_Painting-HSV.png}
        \caption{HSV-Farbcode}
        \label{picture:hsv_color_code}
    \end{figure}
\flushbottom