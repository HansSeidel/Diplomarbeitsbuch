\section{Texturierung Vorarbeiten}
Um Texturen in der Unreal Engine richtig anwenden zu können, müssen zuerst in Blender ein paar Vorbereitungen getroffen werden.

Um mehrere Texturen auf Objekte anwenden zu können, muss man einem Objekt mehrere Materialien zuweisen. Dazu wählt man im Edit Mode die Flächen aus die man einem Material geben
möchte und fügt diese im Properties-Panel unter \menu{Materials > Assign} dem gewünschten Material hinzu.

Bevor man eine UV-Map erstellt, muss man sicher gehen, dass beim Modellieren keine Punkte im 3D-Modell überlappen.
Dazu wechselt man in den Edit Mode und wählt das ganze Objekt aus und entfernt anschließend mit \keys{W} \menu{Remove Doubles}
alle doppelten Punkte im Objekt.

Die benötigte UV-Map erstellt man indem man \keys{\SPACE} drückt und dann nach \menu{Smart UV Project } sucht, auswählt und anwendet. Sie gibt an, an welchem Ort
eine Textur auf einem Objekt erscheint. Im UV\verb-/-Image Editor Panel kann man, wenn man im 3D-View Panel in den Edit Mode
wechselt und das ganze Objekt auswählt, die UV-Map sehen.
