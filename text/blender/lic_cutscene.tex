\todo{Cutscene einem Kapitel zuordnen}
\subsection{Cutscene}

Die Cutscene ist eine Sequenz, die zu einem bestimmten Zeitpunkt im Spiel abgespielt wird.
Im Spiel gibt es zwei Cutscenes, eine Cutscene am Anfang und eine am Ende. Folgend wird die Cutscene
am Schluss vom Spiel beschrieben, weil sie aufwändiger zu erstellen war.

Bei der Schlusssequenz legt sich der Hauptcharakter in sein Grab hinein und begräbt sich anschließend selbst, indem
die Mischmaschine umgedreht wird und der Zement in das Grab fließt. Der Zement wurde mit einer Simulation gemacht.
Damit der Zementfluss auch real aussieht, musste man die Keyframes\footnote{Keyframes sind Werte von Objekten, die zu einem Bestimmten Zeitpunkt gespeichert werden, damit eine Animation auf Basis dieser Werte berechnet werden kann.}
und Objekte so setzen, dass immer weniger Zement aus der Mischmaschine hinaus kommt (\textit{Abbildung \ref{cutscene:image_modellierung}}). Damit der Zement auch hinausfließen kann, wurde die Mischmaschine bewegt.

Die ganze Scene wurde anschließend in die Unreal Engine importiert. Exportiert wurde die Cutscene wie eine Simulation (\textit{Kapitel \ref{Simulation_Heading} \dq Simulationen\dq}).
Mit Blueprints wurde dann die Funktion hinzugefügt, dass sich der
Spieler zu einem gewissen Zeitpunkt Richtung Grab bewegt und anschließend selbst begräbt. Während den Bewegungen, ist der Input für Bewegungen gesperrt.
Am Schluss wird der Bildschirm schwarz. Dadurch wird das Ende des Spiels symbolisiert.

\begin{figure}[h]
    \centering
    \includegraphics[width=.8\textwidth]{images/Cutscene_Modellierung.png}
    \caption{links: Cutscene Modellierung, rechts: Keyframes}
    \label{cutscene:image_modellierung}
\end{figure}