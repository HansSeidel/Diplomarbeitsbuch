\section{Zusammensetzung mehrerer 3D Objekte}\todo{bessere Überschrift}
\subsection{Haus}
\label{haus:ref1}
Um das Haus zu modellieren, musste zuerst die Größe gut eingeplant werden, denn es musste so groß sein, dass einige Objekte hineinpassen.
Außerdem muss alles z.B. Türen, Gänge und die Stockwerkhöhe im Haus ungefähr an die Größe der Charaktere angepasst sein.
Die Objekte im Haus sind angelehnt an echte Objekte, z.B. sind die Stufen von echten Stufen abgemessen worden und in realistischen Maßen an das Haus angepasst worden.

Beim Modellierungsvorgang, wurde zuerst das Haus modelliert und anschließend die Objekte im Haus. Die Objekte wurden pro Stockwerk auf Ebenen(Die roten Pfeile auf Abbildung \ref{Haus:image1} zeigen
die Ebenen auf) in Blender verschoben, damit man gesamte Stockwerke ausblenden kann. Um die Objekte richtig zu platzieren, wurde auf Orthogonale Ansichten umgeschaltet
und das Magnettool benutzt um die Objekte ganz genau zu den Wänden des Hauses zu verschieben. Anschließend wurde der Abstand noch auf einen realistischen Abstand angepasst.

Nachdem alle Objekte für das Haus erstellt worden sind, wurden diese einzeln exportiert und in einem neuen Blender File wieder zusammengefügt. Genauere Informationen zum Exportieren, kann
man unter dem Kapitel \ref{Exportieren_von_Blender_zu_Unreal_Engine_4:ref1} \dq Exportieren von Blender zu Unreal Engine 4\dq finden. Die Objekte werden neu zusammengefügt, damit keine Nebenprodukte, die beim Modellieren
im Haus angefallen sind, im zusammengefügten Modell bestehen bleiben. Außerdem ist es gut alle Objekte einzeln zu haben, falls man sie für einen anderen Zweck noch einmal benötigt.

\begin{figure}[h]
    \centering
    \includegraphics[width=.8\textwidth]{images/Haus-zusammenfuegen_Ebenen.png}
    \caption{Objekte des Hauses, auf mehreren Ebenen verteilt}
    \label{Haus:image1}
\end{figure}