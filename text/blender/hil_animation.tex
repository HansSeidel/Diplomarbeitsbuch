\section{Animation}
\subsection{Die 12 Prinzipien der Animation}
\label{animation:principles}
Die 12 Prinzipien der Animation wurden zuerst 1981 im Buch \dq The Illusion of Life: Disney Animation\dq, von Ollie Johnston und Frank Thomas,
niedergeschrieben und veröffentlicht. Das Buch wird von vielen als \dq Die Bibel der Animation\dq bezeichnet. Der Sinn hinter diesen Prinzipien ist es eine möglichst
akkurate Animation nach den Gesetzen der Physik zu zeichnen \verb-/- gestalten. Auch heutzutage lassen sich die Prinzipien auf moderne Computeranimation anwenden.

\subsubsection{Squash and Stretch}
Squash and Stretch (übersetzt: quetschen und dehnen) soll Objekten und Charakteren Gewicht und Flexibilität geben.
 Umso schwere ein Gummiball ist umso mehr soll er sich, je nach Höhe, beim Aufprall eindrücken und dehnen.

\subsubsection{Anticipation}
Anticipation, auf Deutsch übersetzt \dq Erwartung\dq, soll in einer Animation gegeben sein, um den Zuseher auf gewisse Aktionen vorzubereiten.
Besipiele: Bevor man springt biegt man sein Knie; Bevor man einen Fußball schießt holt man aus.

\subsubsection{Staging}
Staging, auf Deutsch übersetzt \dq Inszenierung\dq, besagt das es für jeden klar sein muss, was man mit einer Animation oder Zeichnung zeigen möchte.
Als Beispiel: Szenario: Ball wird mit Wut geworfen; Ergebnis: Ball wird mit viel Schwung und passender Mimik geworfen.

\subsubsection{Straight Ahead Action and Pose to Pose}
Dieses Prinzip bezeichnet zwei verschiedene Herangehensweisen der Animation. Die Straight Ahead Action Methode ist jede Frame hintereinander zu zeichnen.
Während man mit Pose to Pose verschiedene Posen zu verschiedenen Zeitpunkten zeichnet und später die Zwischenframes zeichnet.
In der heutigen Computeranimation hat sich die Straight Ahead Action Methode kaum umgesetzt.
Pose to Pose ist sehr praktisch heutzutage da man sich die Zwischenframes leicht mithilfe des Computers berechnen lassen kann. Es ist jedem belassen wie viele Posen er zeichnet \verb-/- posiert.

\subsubsection{Follow Through and Overlapping Action}
Follow Through bezieht sich auf in Verbindung stehenden Teile des Körpers, wie Arme oder Haare,
und dessen Trägheit beim Abbremsen oder Beschleunigen. Overlapping Action bezieht sich darauf, dass verschiedene Körperteile wie Arme und Kopf verschieden schnell bewegen.

\subsubsection{Timing}
Je nach Gewicht reagieren viele Objekte auf eine von außen oder innen kommende Kraft langsamer oder schneller.

\subsubsection{Exaggeration}
Die Übertreibung von Bewegungen, kann helfen diese realistischer aussehen zu lassen, da oft realitätsnahe Bewegungen zu statisch auf den Menschen wirken.

\subsubsection{Solid Drawing}
Dieses Prinzip beschreibt, dass es wichtig ist zu wissen wie man
drei-dimensionale Objekte, in verschiedenen Perspektiven mit leitenden Linien, zeichnet. Mit der 3D Animation hat sich dies stark geändert.

\subsubsection{Appeal}
Dies hat damit zu tun, den animierten Charakter, interessant für den Zuseher zu gestallten.

\subsection{Animation in Blender}
In Blender gibt es auch Funktionen und Tools, um Animationen umzusetzen. Außerdem gibt es eigene Fenster Konfigurationen.
Besonders wichtig in der Animation sind Keyframes, die Timeline, das Dope Sheet,
das Action Sheet und der Graph Editor. Mithilfe dieser Tools kann man Animationen in Blender erstellen, exportieren und rendern.

\subsubsection{Keyframes}
Keyframes werden gemeinsam mit der Timeline benützt. Sie werden verwendet, um Werte für einen gewissen Zeitpunkt zu speichern.
Im Fall der Animation in Blender werden Werte wie Position, Rotation und Skalierung, als Beispiel, gespeichert. Wobei man selbst entscheiden
kann was in einem Keyframe gespeichert werden soll. Keyframes ermöglichen es mit Pose to Pose zu arbeiten (\textit{Kapitel \dq \ref{animation:principles} Die 12 Prinzipien der Animation\dq}).
Um einen Keyframe hinzuzufügen kann man im 3D View die Taste \keys{i} drücken und die gewünschte Option auswählen. Alle möglichen Optionen sind im Bild abgebildet.

\begin{figure}[H]
    \centering

    \includegraphics[width=.8\textwidth]{images/animation_keyframes_options.PNG}
    \caption{Alle möglichen Optionen für einen Keyframe}
\end{figure}

\subsubsection{Timeline}
Die Timeline ist die gesamte angezeigte oder gerenderte Animation. Die Zeiteinheiten in Blender sind Frames.
Die Länge ergibt sich aus den definierten Frames (z.B. von Frame 0 bis Frame 250) und der FPS (Bilder pro Sekunde).
Keyframes werden in der Timeline gelb markiert angezeigt. Abspielen kann man die Timeline mit dem Tastenkürzel \keys{Alt + a}.

\begin{figure}[H]
    \centering

    \includegraphics[width=.8\textwidth]{images/animation_timeline.PNG}
    \caption{Beispiel einer Timeline mit Keyframes (gelb)}
\end{figure}

\subsubsection{Dope Sheet}
Das Dope Sheet ist auch eine Art Timeline. Im ursprünglichen Sinne kommt das Dope Sheet aus der gezeichneten Animation,
wo jede Änderung aufgezeichnet und dokumentiert wurde. In Blender dient das Dope Sheet dazu, dass man über die gesamte Animation,
also auch über jedes Armature und Objekt, und deren Keyframes, einen Überblick hat. Jede Keyframe eines einzelnen Bones wird mithilfe einer Timeline und Reihen dargestellt.

\begin{figure}[H]
    \centering

    \includegraphics[width=.8\textwidth]{images/animation_dope_sheet.PNG}
    \caption{Beispiel eines Dope Sheets}
\end{figure}

\subsubsection{Action Editor}
Das Action Sheet ist Teil des Dope Sheets und ist auch ähnlich aufgebaut. Nur bezieht sich das Action Sheet auf ein Objekt und dessen Actions.
Actions kann man als Datentyp für Animationen sehen. Somit ist es logisch, dass man mit dem Action Editor die verschiedenen Actions verwalten und erstellen kann.

\begin{figure}[H]
                \centering

                \includegraphics[width=.8\textwidth]{images/animation_action_editor.PNG}
                \caption{Beispiel eines Action Editors}
\end{figure}

\begin{figure}[H]
    \centering

    \includegraphics[width=.8\textwidth]{images/animation_action_editor_actions.PNG}
    \caption{Beispiel der Auswahl für die verschiedenen gespeicherten Actions}
\end{figure}

\subsubsection{Graph Editor}
Der Graph Editor basiert auch auf einer Timeline. Im Graph Editor kann man sich einen, mehrere oder alle Bones anzeigen lassen.
Im Vergleich zu dem Dope Sheet dient der Graph Editor jedoch dazu, alle Eigenschaften einer Animation (Rotation, Position, Skalierung etc.)
und deren Intensität oder Verhalten anzuzeigen. Die Eigenschaften und Intensitäten werden mithilfe von F-Curves angezeigt und bearbeitet. F-Curves
sind Kurven, sie können Konstante, Lineare oder eine Bezier Kurve sein.
Wenn man Keyframes setzt, entsteht automatisch zwischen zwei Keyframes eine dieser Kurven, und im Graph Editor kann man diese Kurve genau bearbeiten.

\begin{figure}[H]
                \centering

                \includegraphics[width=.8\textwidth]{images/animation_graph_editor.PNG}
                \caption{Beispiel eines Graph Editors}
\end{figure}

Im Bild sieht man auf der Linken Seite alle Eigenschaften, diese kann man auch einzeln anzeigen. Auf der rechten Seite sieht man alle ausgewählten Kurven.
Die schwarzen Quadrate auf den Kurven sind Ankerpunkte die auf dem gleichen Frame wie die Keyframes sitzen. Diese Ankerpunkte kann man einzeln oder in Gruppen auswählen.
Wenn man einen Ankerpunkt auswählt erscheint am Ankerpunkt eine Tangente.
Diese Tangente hat wiederum zwei Punkte, die man verschieben kann und somit die Ausrichtung der Kurve an diesem Ankerpunkt ändern kann.

\begin{figure}[H]
    \centering

    \includegraphics[width=.8\textwidth]{images/animation_graph_editor_anker.PNG}
    \caption{Beispiel eines ausgewählten Ankers}
\end{figure}

\subsubsection{Animationsarten}

\paragraph{Mit Bones und Rigs:}
Wie schon in der Umsetzung der Rigs erklärt, verwendet man am besten für Charaktere ein gebautes Armature aus Bones.
Mithilfe diese Armatures kann man, als Beispiel, sehr flüssige Geh- und Renn-Animationen erzeugen.

\begin{figure}[H]
    \centering

    \includegraphics[width=.8\textwidth]{images/rigging_body.PNG}
    \caption{Fertiges Rig eine Charakters}
\end{figure}

\paragraph{Mit Pfaden:}
Ein Pfad ist nichts anderes als eine Kurve in Blender die nicht gerendert wird. In den meisten Fällen ist es eine Bezier Kurve mit einer beliebigen
Anzahl an Ankerpunkten. Um ein Objekt an diesem Pfad entlang zu animieren braucht man den Constraint \dq Follow Path\dq.
Bei dem Constraint sind folgende Optionen zu beachten: Target, Forward, Up und Offset. Mithilfe dieser Optionen kann man das Objekt auf dem Path richtig positionieren und ausrichten.
Verwendungszweck für diese Animation ist z.B. einen Flieger fliegen zu lassen oder einen Charakter einen bestimmten weg abgehen lassen.

\begin{figure}[H]
    \centering

    \includegraphics[width=.8\textwidth]{images/animation_path_constraint.png}
    \caption{Menü für den Follow Path Constraint}
\end{figure}

\paragraph{Mit Shape Keys:}
Diese Art der Animation wird häufig bei Gesichtern angewendet. Shape Key Animation eignen sich am besten, um Deformierungen an einem Mesh durchzuführen.
Deswegen benutzt man sie hauptsächlich in Gesichtern, um Falten oder Münder zu bewegen.

\subsection{Umsetzung der Animationen}
Zu umsetzende Animationen für die beiden Charaktere sind jeweils \dq Idle\dq und \dq Walking\dq. Der Antagonist (Grabwächter)
hat eine zusätzliche \dq Running\dq Animation. Oft sind die Bewegungen von echten Menschen zu gering und deswegen ist es wichtig das Animationsprinzip \dq Exaggeration\dq
zu beachten. Animiert wird mithilfe des Pose to Pose Prinzip. Ein weiterer wichtiger Punkt ist es das die Animation in einer Endlosschleife
ohne sichtbaren Ruckler spielt. Um dies zu erreichen muss die erste Pose ident mit der letzten Pose sein.

\subsubsection{Die Idle-Animation}
Eine Idle-Animation ist eine rastende Pose für den Charakter. Diese ist meistens stehend so auch wie für die beiden Charaktere.
Da die stehende Pose von echten Menschen oft zu wenig deutlich ist und als Animation künstlich aussieht muss man sie deutlicher und übertriebener darstellen.

Die Gestaltung der beiden Idle-Animationen ist verschieden für beide Charaktere. Der Hauptcharakter ist neutral, während der Antagonist kräftig und angsteinflößend erscheinen soll.

\paragraph{Hauptcharakter:}
Für eine einfache Idle-Animation braucht man nur zwei Posen. Generell sollten die Beine etwas breiter dastehen und die Füße werden leicht nach außen gedreht.
Die Arme werden in ruhender Position nach unten bewegt. In der ersten Pose ist die Hüfte etwas höher und die Arme und Finger in einer Position gebracht.
In der zweiten ist sie niedriger und die Arme und Finger haben sich ein kleines Stück bewegt. Damit die Animation eine Endlosschleife ist, kopiert man die erste Pose \verb-/-
Keyframes einfach auf die letzte Frame. Damit der erste und letzte Frame nicht gleich ist,
was einen Ruckler erzeugen würde, setzt man die letzte angezeigte Frame, eine Frame vor der Pose. Das erste Bild zeigt Pose 1 und das zweite Pose 2.

\begin{figure}[H]
    \centering

    \includegraphics[width=.8\textwidth]{images/animation_idle.png}
    \caption{Die zwei Posen der Idle-Animation}
\end{figure}

\paragraph{Antagonist / Grabwächter:}
Die Animation für den Antagonisten ist ähnlich aufgebaut, der größte Unterschied ist der breitere Stand und aggressive Faust.
Außerdem bewegt sich sein Kopf nach links und rechts da er nach dem Hauptcharakter sucht. Es ist wieder zu beachten das keine ruckelnde Endlosschleife entsteht.

\begin{figure}[H]
    \centering

    \includegraphics[width=.8\textwidth]{images/animation_an_idle.PNG}
    \caption{Eine Pose der Idle-Animation des Antagonisten}
\end{figure}

\subsubsection{Die Walking Animation}
Die Walking Animation, oder auch Walk Cycle genannt, ist komplizierter. Es gibt vier Posen und diese gibt es doppelt gespiegelt.
Die Abstände der Frames zwischen diesen Posen ist gleich in einem typischem Walk Cycle. Ein Walk Cycle besteht aus der Contact,
Up, Passing, Down, Reverse Contact, Reverse Up, Reverse Passing,
Reverse Down und um eine Endlosschleife zu erzeugen, wieder die Contact Pose. Generell ist der Körper in der Walk Animation nach vorne gelehnt.

\begin{figure}[H]
    \centering

    \includegraphics[width=.8\textwidth]{images/animation_walk_cycle.png}
    \caption{Ein kompletter Walk Cycle}
\end{figure}

\paragraph{Hauptcharakter:}
In der Contact Pose berühren beide Füße den Boden und die Arme schwingen am meisten aus. Die Arme schwingen gegengleich zu den Beinen.
Die Hüfte ist dabei auf mittlerer Höhe. In der Down Pose ist die Hüfte am niedrigsten und die Arme schwingen wieder zurück.
Ein Bein ist unter dem Körper und das andere schwingt zurück. In der Passing Pose schwingt das eine Bein wieder vorwärts während
das andere unterm Körper bleibt. Die Hüfte ist wieder auf mittlerer Höhe und die Arme sind beide auf gleicher Höhe. In der Up Pose
ist die Hüfte am höchsten, die Arme schwingen von einander weg und das Bein geht weiter vorwärts. Beide Beine setzen jetzt
für die Reverse Contact Pose an. Um den Walk Cycle zu vervollständigen, kann man die vorhandenen Posen nehmen und diese an den entsprechenden stellen spiegeln.

\paragraph{Antagonist / Grabwächter:}
Die Walk Animation ist im Grunde gleich zu der des Hauptcharakters.
Um den Antagonisten aber besser darzustellen sind die Bewegungen stärker ausgeprägt, um dem Charakter mehr Gewicht zu geben.

\subsubsection{Die Running Animation}
Die Running Animation ist der Walking Animation sehr ähnlich.
Es gibt auch die vier Posen Contact, Up, Down und Passing. Damit aber der Running Cycle auch wie rennen aussieht ist der Körper mehr nach vorne gebeugt,
die Bewegungen mit dem Armen und Beinen intensiver und die Bewegungen rauf und runter sind schwerer gestaltet.

\begin{figure}[H]
    \centering

    \includegraphics[width=.8\textwidth]{images/animation_run_cycle.png}
    \caption{Ein kompletter Run Cycle}
\end{figure}

Ein weiterer Unterschied zum Walk Cycle ist, dass die Up Pose nicht in der Mitte von Passing und R Contact liegt,
sondern kurz vor R Contact platziert ist. Das gibt der Animation mehr Gewicht. Im, Bild ist die grün markierte Keyframe die Up Pose.

\begin{figure}[H]
    \centering

    \includegraphics[width=.8\textwidth]{images/animation_run_an_timeline.png}
    \caption{Die Timeline des Run Cycles}
\end{figure}

\subsubsection{Exportieren für Unreal Engine}
Animationen können ohne Probleme als .fbx in die Unreal Engine exportiert werden. Das Einzige, das es zu beachten gibt, ist,
dass Unreal Engine einen Root-Bone braucht. Dieser ist ohne Problem schnell hinzugefügt und mit dem Parent Constraint für das Armature ausgestattet.
