\section{Einleitung}
Bei der 3D Modellierung, werden Objekte in einem dreidimensionalen Raum erstellt. Mit anderen Worten, es gibt ein Koordinatensystem mit einer X-, Y- und Z-Achse.
In diesem werden Punkte gesetzt, die auch zu Kanten und Flächen verbunden werden können. Mehrere solcher Punkte, Kanten und Flächen ergibt dann ein 3D-Objekt.
Diese 3D-Objekte können für alle möglichen Zwecke eingesetzt werden. Damit man sich etwas darunter vorstellen kann, folgen nun ein paar Beispiele für
eine mögliche Anwendungen.

Anwendungsmöglichkeiten:
\begin{itemize}
    \item Echte Objekte nachmodellieren und digital z.B. in einem Bild verwenden.
    \item Objekte erschaffen und für z.B. Spiele verwenden.
    \item Objekte für Veranschaulichungen von Zukunftsprojekten erstellen z.B. ein Wohnhaus.
\end{itemize}
