\section{Edit Mode}
Im Edit Mode kann man alle möglichen Veränderungen an einem Objekt durchführen. Man kann zum Beispiel Punkte, Kanten und Flächen bewegen, löschen oder hinzufügen.
Es gibt aber viel mehr Funktionen, die für unser Projekt mitunter auch in Sachen Effizienz wichtig waren. Eine dieser Funktionen ist der Magnet. Mit ihm kann man Punkte, Kanten oder Flächen zu 100\%\ genau an
die Koordinaten anderer Punkte, Kanten oder Flächen verschieben ohne die Werte der Position anzugeben.

Die verwendeten Funktionen werden in \todo{r2} \autoref{sec:Modellierung_von_3D_Objekten} \dq  Modellierung von 3D Objekten\dq anhand von
Praxisbeispielen gezeigt.

\subsection{Seperate}
Im Edit Mode gibt es eine nützliche Funktion namens Seperate, falls man ein Objekt in zwei Objekte zerlegen möchte.
Das ist besonders dann nützlich wenn man etwas falsch zusammengefügt hat.
Man kann Objekte zerteilen, indem man im Edit Mode die gewünschten Punkte, Kanten oder Flächen auswählt, P drückt und
dann auf Selection drückt.