\section{Viewport Shading}
Mit Viewport Shading kann man in Blender bestimmen,\todo{Beistrich?} wie Objekte angezeigt werden. Für uns waren die Varianten Solid und Wireframe relevant.

\subsection{Solid}
Im Solid Mode wird das Objekt mit der Standardfarbe grau angezeigt.\citep{viewportshading:link} Außerdem werden die verschiedenen Materialien die ein Objekt
hat, in den jeweiligen Farben angezeigt. Mit dem Solid Mode kann man einen guten Eindruck von der Form des fertigen Objekts bekommen.

\subsection{Wireframe}
Im Wireframe Modus werden Objekte so dargestellt, dass man durch sie hindurch sehen kann.\citep{viewportshading:link}
Das ist hilfreich wenn man wissen muss wie Kanten, Punkte oder Flächen auf gegenüberliegenden Seiten des Objekts
platziert sind. Außerdem kann man durch das Objekt hindurch mehrere Punkte, Kanten oder Flächen auf einmal auswählen. So kann man zum Beispiel
in einer orthographischen Ansicht,\todo{Beistrich?} alle Kanten am Boden eines Objektes auswählen ohne es drehen zu müssen. Diese Kanten kann
man in Folge parallel verschieben.
\todo{Referenz Wireframe, bild?}