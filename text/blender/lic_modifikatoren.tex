\section{Modifikatoren}
In diesem Kapitel, werden die Modifikatoren beschrieben, die für dieses Projekt oft verwendet wurden. Es werden nur Funktionen außerdem nur
Funktionen der Modifikatoren beschrieben, die von uns genutzt wurden.

Allgemein kann man zu Modikfikatoren noch sagen, dass sie Objekte in einer gewissen Art und weiße, abhängig davon welcher Modifikator verwendet wird, verändern.
\subsection{Boolean}
\citep{blender:boolean_modifier} Der Boolean Modifikator macht aus zwei Objekten ein Objekt. Dazu gibt es
drei Boolean-Operationen, die man anwenden kann.

Diese heißen:
\begin{itemize}
    \item  Difference
    \item  Union
    \item  Intersect
\end{itemize}

Bei Difference werden bei zwei überlappenden Objekten, die überlappenden Teile ausgeschnitten. Beim Zielobjekt ist nun ein Loch an der Gewünschten stelle (Abbildung \ref{modifikatoren:image1} links).

Bei Union werden zwei Objekte zu einem Objekt zusammengfügt (Abbildung \ref{modifikatoren:image1} mitte).

Bei Intersect wird das Objekt mit dem Modifier, an den Überschneidenden stellen in das Zielobjekt eingefügt (Abbildung \ref{modifikatoren:image1} rechts).

\begin{figure}[h]
    \centering
    \includegraphics[width=.8\textwidth]{images/Modifikatoren-Boolean.png}
    \caption{Difference-, Union- und Intersect Modifikator}
    \label{modifikatoren:image1}
\end{figure}

\subsection{Mirror}
\citep{blender:mirror_modifier} Der Mirror Modifikator spiegelt ein Objekt um den eigenen Mittelpunkt.
Den Mittelpunkt kann man festlegen indem man im Editmode einen Punkt auswählt
und mit SHIFT \verb-+- S \verb+->+ Cursor to Selected den Cursor setzt und anschließend im Objectmode mit STRG \verb-+- SHIFT \verb-+- ALT \verb-+- C \verb+->+
Origin to 3D Cursor den Mittelpunkt setzt.

In Abbildung \ref{modifikatoren:image2} kann man die Anwendung des Mirror Modifikatoren sehen. Der Orangene Teil des Objekts, ist in allen drei Beispielen das
Grundobjekt, welches gespiegelt wird.

\begin{figure}[h]
    \centering
    \includegraphics[width=.8\textwidth]{images/Modifikatoren-Mirror.png}
    \caption{Mirror Modifikator mit spiegelung um die Z-, Y- und X-Achse}
    \label{modifikatoren:image2}
\end{figure}

\subsection{Array}
\citep{blender:array_modifier} Der Array Modifier vervielfacht ein Objekt entlang einer Achse. Außerdem kann man den relativen oder absoluten Abstand zwischen den Objekten
angeben, damit diese regelmäßig wiederholt werden. In Abbildung \ref{modifikatoren:image3} sieht man den angewendeten Array Modifikator. Die ausgewählten (orange umrandeten) Würfel werden auf der Y-Achse
wiederholt, die anderen auf der Y- und Z-Achse.
\begin{figure}[h]
    \centering
    \includegraphics[width=.8\textwidth]{images/Modifikatoren-Array.png}
    \caption{Array Modifikator}
    \label{modifikatoren:image3}
\end{figure}

\subsection{Curve}
\citep{blender:curve_modifier} Mit dem Curve Modifikator, kann man ein beliebiges Objekt an eine Bezierkurve anpassen. In Abbildung \ref{modifikatoren:image4} sieht man
in der oberen Hälfte, die
\begin{figure}[h]
    \centering
    \includegraphics[width=.8\textwidth]{images/Modifikatoren-Curve.png}
    \caption{Curve Modifikator}
    \label{modifikatoren:image4}
\end{figure}

\subsection{Bevel}
\subsection{Decimate}
\subsection{EdgeSplit}