\section{Modifikatoren}
Modikfikatoren verändern Objekte in einer gewissen Art und Weiße. Die Veränderung ist je nach Modifikator unterschiedlich.

In diesem Kapitel, werden die Modifikatoren beschrieben, die für dieses Projekt oft verwendet wurden.
Es werden außerdem nur Funktionen der Modifikatoren beschrieben, die von uns genutzt wurden.

\subsection{Boolean-Modifikator\citep{blender:boolean_modifier}}
\label{Boolean:heading}
Der Boolean-Modifikator verändert ein Objekt mittels einem zweiten Objekt. Dazu gibt es
drei Boolean-Operationen, die man anwenden kann.

Diese heißen:
\begin{itemize}
    \item  Difference
    \item  Union
    \item  Intersect
\end{itemize}

Bei Difference werden bei zwei überlappenden Objekten die überlappenden Teile ausgeschnitten. Beim Zielobjekt ist nun ein Loch an der Gewünschten stelle (\textit{Abbildung \ref{modifikatoren:image1}, links}).

Bei Union werden zwei Objekte zu einem Objekt zusammengefügt (\textit{Abbildung \ref{modifikatoren:image1}, mittig}).

Bei Intersect wird das Objekt mit dem Modifikator an den überschneidenden Stellen in das Zielobjekt eingefügt (\textit{Abbildung \ref{modifikatoren:image1}, rechts}).

\begin{figure}[h]
    \centering
    \includegraphics[width=.8\textwidth]{images/Modifikatoren-Boolean.png}
    \caption{Difference-, Union- und Intersect-Operation}
    \label{modifikatoren:image1}
\end{figure}

\subsection{Mirror-Modifikator\citep{blender:mirror_modifier}}
\label{Mirror:heading}
Der Mirror-Modifikator spiegelt ein Objekt um den Origin.
Den Origin kann man festlegen indem man im Edit Mode einen Punkt im Koordinatensystem auswählt, dann die Position des Cursors setzt
und anschließend im Object Mode den Origin setzt.

In \textit{Abbildung \ref{modifikatoren:image2}} kann man die Anwendung des Mirror Modifikators sehen. Der Orangene Teil des Objekts ist in allen drei Beispielen das
Grundobjekt, welches gespiegelt wird.

\begin{figure}[h]
    \centering
    \includegraphics[width=.8\textwidth]{images/Modifikatoren-Mirror.png}
    \caption{Mirror-Modifikator mit Spiegelung um die Z-, Y- und X-Achse}
    \label{modifikatoren:image2}
\end{figure}

\subsection{Array-Modifikator\citep{blender:array_modifier}}
\label{Array:heading}
Der Array-Modifikator vervielfacht ein Objekt entlang einer Achse. Man kann den relativen oder absoluten Abstand zwischen den Objekten
angeben, damit diese regelmäßig wiederholt werden. In \textit{Abbildung \ref{modifikatoren:image3}} sieht man den angewendeten Array-Modifikator.
Die ausgewählten (orange umrandeten) Würfel werden auf der Y-Achse
wiederholt, die anderen auf der Y- und Z-Achse.
\begin{figure}[h]
    \centering
    \includegraphics[width=.8\textwidth]{images/Modifikatoren-Array.png}
    \caption{Array-Modifikator}
    \label{modifikatoren:image3}
\end{figure}

\subsection{Curve-Modifikator\citep{blender:curve_modifier}}
\label{Curve:heading}
Mit dem Curve-Modifikator, kann man ein beliebiges Objekt an eine Bezierkurve anpassen. In \textit{Abbildung \ref{modifikatoren:image4}} sieht man
in der oberen Hälfte das Objekt und die Bezierkurve, in der unteren beide zusammengefügt.
Außerdem zu erwähnen ist, dass das Rechteck mehrfach unterteilt ist damit es sich der Bezierkurve anpassen kann.
\begin{figure}[h]
    \centering
    \includegraphics[width=.8\textwidth]{images/Modifikatoren-Curve.png}
    \caption{Curve-Modifikator}
    \label{modifikatoren:image4}
\end{figure}

\subsection{Bevel-Modifikator\citep{blender:bevel_modifier}}
\label{Bevel:heading}
Mit dem Bevel-Modifikator kann man Kanten, durch Flächen, die nahe der Kanten hinzugefügt werden, abrunden (\textit{Abbildung \ref{modifikatoren:image5}}).
Wichtige Einstellungen sind der Abstand von der am äußersten hinzugefügten und der ursprünglichen Kante und die Anzahl der Kanten die hinzugefügt werden.
Der Bevel Modifikator wurde oft verwendet, damit die erstellten Objekte echt aussehen, weil jedes Objekt z.B. eine Tischkante nie abrupt
aufhört, sondern zumindest ein bisschen abgerundet ist.
\begin{figure}[h]
    \centering
    \includegraphics[width=.8\textwidth]{images/Modifikatoren-Bevel.png}
    \caption{Bevel-Modifikator}
    \label{modifikatoren:image5}
\end{figure}

\subsection{Decimate-Modifikator\citep{blender:decimate_modifier}}
\label{Decimate:heading}
Mit dem Decimate-Modifikator kann man bei einem Objekt die Flächen reduzieren. Zum Einsatz kommt er bei komplizierten
Objekten, die viele Flächen brauchen, damit sie detailliert genug dargestellt werden können. Man entfernt dann aber möglichst viele Flächen
wieder, so dass das Objekt noch gut genug aussieht. Das entfernen der überflüssigen Flächen verbessert die Performance sehr stark.
In \textit{Abbildung \ref{modifikatoren:image6}} sieht man den ursprünglichen Würfel (orange) und den Würfel mit dem angewendeten
Decimate Modifikator.
\todo{Vergleich - Detailed Mesh für Texturen.}
\begin{figure}[h]
    \centering
    \includegraphics[width=.8\textwidth]{images/Modifikatoren-Decimate.png}
    \caption{Decimate-Modifikator}
    \label{modifikatoren:image6}
\end{figure}

\subsection{Edge-Split-Modifikator\citep{blender:edgesplit_modifier}}
\label{Edge_Split:heading}
Mit dem Edge Split Modifikator kann man bei einem Objekt bestimmen, ab welchem Winkel Smooth Shading angewendet wird.
In \textit{Abbildung \ref{modifikatoren:image7}} links, wird Smooth Shading gar nicht angewendet, in der Mitte wird es nur an
den Kanten, mit dem geringsten Winkel zu den Flächen angewendet und rechts wird Smooth Shading überall angewendet.

Mehr zu Smooth Shading findet man in \textit{Kapitel \ref{objectMode:smoothshading} \dq Smooth Shading\dq}.
\begin{figure}[h]
    \centering
    \includegraphics[width=.8\textwidth]{images/Modifikatoren-Edgesplit.png}
    \caption{Edge-Split-Modifikator}
    \label{modifikatoren:image7}
\end{figure}