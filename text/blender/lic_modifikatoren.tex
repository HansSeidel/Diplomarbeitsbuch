\section{Modifikatoren}
In diesem Kapitel, werden die Modifikatoren beschrieben, die für dieses Projekt oft verwendet wurden.

Allgemein kann man zu Modikfikatoren noch sagen, dass sie Objekte in einer gewissen Art und weiße, abhängig davon welcher Modifikator verwendet wird, verändern.
\subsection{Boolean}
\citep{blender:boolean_modifier} Der Boolean Modifikator macht aus zwei Objekten ein Objekt. Dazu gibt es
drei Boolean-Operationen, die man anwenden kann.

Diese heißen:
\begin{itemize}
    \item  Difference
    \item  Union
    \item  Intersect
\end{itemize}

Bei Difference werden bei zwei überlappenden Objekten, die überlappenden Teile ausgeschnitten. Beim Zielobjekt ist nun ein Loch an der Gewünschten stelle (Abbildung \ref{modifikatoren:image1} links).

Bei Union werden zwei Objekte zu einem Objekt zusammengfügt (Abbildung \ref{modifikatoren:image1} mitte).

Bei Intersect wird das Objekt mit dem Modifier, an den Überschneidenden stellen in das Zielobjekt eingefügt (Abbildung \ref{modifikatoren:image1} rechts).

\begin{figure}[h]
    \centering
    \includegraphics[width=.8\textwidth]{images/Modifikatoren-Boolean.png}
    \caption{Difference-, Union- und Intersect-Modifikator}
    \label{modifikatoren:image1}
\end{figure}

\subsection{Mirror}
\subsection{Array}
\subsection{Curve}
\subsection{Bevel}
\subsection{Decimate}
\subsection{EdgeSplit}