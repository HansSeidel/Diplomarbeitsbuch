\section{Modellierung von 3D Objekten}
Ein großer Teil um das Spiel zu machen, war das modellieren. In diesem Kapitel wird gezeigt, wie man etwas kompliziertere
Modelle macht und auf was man achten muss.

Es ist sehr wichtig, dass das Spiel am Ende mit einer guten Performance spielbar ist. Deswegen wurde beim Modellieren darauf geachtet,
das es so wenig Flächen wie möglich pro Modell gibt \citep{unreal:modellierungVon3dObjekten_performance}, denn diese müssen vom Computer berechnet werden.
Darunter kann dann die Performance im Spiel leiden.

\label{sec:Modellierung_von_3D_Objekten}
\todo{r2}
\subsection{Paracelsus Grab}
Das Grab des Paracelsus ist ein Modell welches einem echten Objekt entspricht. Um ein originalgetreues Ergebnis zu erhalten,
modelliert man das Objekt von einem Foto nach. Dazu muss man ein Foto in Blender importieren. Um ein Foto zu importieren muss man sich im \dq Object Mode\dq befinden und \dq N\dq
drücken. Im Menü sieht man dann das Feld Background Images (Rot umrandet in Abbildung \ref{Paracelsus_Grab:image1}). Dort kann man dann den Pfad des Bildes angeben
und einstellen bei welchen Winkel zu welcher Achse sichtbar ist. In diesem Fall ist das Bild sichtbar, wenn man genau auf die Achsen X und Z schaut. Man kann außerdem die
durchsichtigkeit des Bildes einstellen, in diesem Fall ist es zu 100\% sichtbar, damit man während dem modellieren die Linien im Bild noch gut erkennt.

\begin{figure}[h]
    \centering
    \includegraphics[width=.8\textwidth]{images/Paracelsus-Grab_Background-Images.png}
    \caption{Import von Bildern}
    \label{Paracelsus_Grab:image1}
\end{figure}

Nun modelliert man mit Formen das Grab nach und verschiebt im Editmode einzelne Punkte so das sie passen. Man muss allerdings darauf achten, das in diesem Bild
(Abbildung \ref{Paracelsus_Grab:image2}) Verzerrungen aufgrund des Aufnahmewinkels des Fotos auftreten, deshalb sind einige Stellen nicht genau, sondern
durch eine Schätzung der Größen nachmodelliert worden.

\subsubsection{Vase}


\subsubsection{Gesicht}
Für die Modellierung des Gesichts, wurde ein Brush verwendet. Diese Methode braucht zwar viele Flächen, was die Performance des Spiels beeinträchtigen kann, war
aber notwendig, damit nachher noch eine andere Textur mit einer eigenen Bumpmap verwendet werden kann. Um das Gesicht zu erstellen wurde zuerst auf einer Seite der Säule,
die Fläche mit dem Shortcut W \verb-+- Subdivide (Edit Mode) unterteilt. Dann wurde im Sculptmode eine Schwarzweiß Textur eingefügt (Abbildung \ref{Paracelsus_Grab:image4}),
damit man mit dem Brush Höhen und Tiefen Zeichnen kann. Außerdem wurden die Werte Strenght und Radius angepasst.

\begin{figure}[h]
    \centering
    \includegraphics[width=.8\textwidth]{images/Paracelsus-Grab_BandW.png}
    \caption{Einfügen der Schwarz/Weiß Textur}
    \label{Paracelsus_Grab:image4}
\end{figure}

\subsubsection{Verzierungen}
Nachdem die Verzierung des Paracelsus Grabes (Abbildung \ref{Paracelsus_Grab:image3}) sehr ähnlich einer bereits vorhandenen Form sieht, nämlich der Verzierung des Mausoleums, wurde diese übernommen und
mit dem Sockel des Grabes zusammengefügt. Dazu wurde zuerst mit einem Würfel und dem Boolean-Modifikator ein Stück aus dem Sockel herauseschnitten, damit
sich die Verzierung nicht mit dem Sockel überschneidet. Danach wurden der Sockel und die Verzierung mit STRG \verb-+- J zusammengefügt. Die Verzierung selber, wurde mit einer
Fläche modelliert, die dann im Editmode so angepasst wurde, dass sie aussieht wie eine Verzierung.
\begin{figure}[h]
    \centering
    \includegraphics[width=.8\textwidth]{images/Paracelsus-Grab_Verzierung.png}
    \caption{Verzierung des Paracelsus Grabs}
    \label{Paracelsus_Grab:image3}
\end{figure}


\begin{figure}[h]
    \centering
    \includegraphics[width=.8\textwidth]{images/Paracelsus-Grab_Nachmodellierung.png}
    \caption{Paracelsus Grab Nachmodellierung}
    \label{Paracelsus_Grab:image2}
\end{figure}
\subsection{Bettdecke}
\todo{Gehört die Bettdecke zu Simulationen?}