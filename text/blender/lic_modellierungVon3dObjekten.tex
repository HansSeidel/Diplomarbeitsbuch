\section{Modellierung von 3D Objekten}
\label{sec:Modellierung_von_3D_Objekten}
Ein großer Teil um das Spiel zu machen, war das modellieren. In diesem Kapitel wird gezeigt, wie man etwas kompliziertere
Modelle macht und auf was man achten muss.

Es ist sehr wichtig, dass das Spiel am Ende mit einer guten Performance spielbar ist. Deswegen wurde beim Modellieren darauf geachtet,
dass es pro Modell so wenig Flächen wie möglich gibt \citep{unreal:modellierungVon3dObjekten_performance}, denn diese müssen vom Computer berechnet werden.
Darunter kann dann die Performance im Spiel leiden.

Zwei wesentliche Gebäude des Spieles, sind das Haus des Grabwächters und das Mausoleum inklusive aller enthaltenen Objekte.
Weiters gibt es einige Objekte die für die Story relevant sind und Objekte die für modelliert wurden um das Spiel zu füllen.
\todo{Modellierte Objekte referenzieren}

Auflistung der Objekte:
\todo{Kapitel Dusche schreiben}
\todo{Kapitel schreiben}

\begin{tabular}{l|l|l|l}
    \textbf{Zweck} & \textbf{Objekt} & \textbf{Entwickelt von} & \textbf{Anmerkung} \\
    \hline
    \textbf{Haus} &  &  & Kapitel \ref{haus:ref1}\\
    \hline
    & Stufen & Lichtenstein & \\
    & Haustür & Lichtenstein & \\
    & Tür & Lichtenstein & \\
    & Fenster & Lichtenstein & \\
    & Sessel & Lichtenstein & \\
    & Toilette & Lichtenstein & \\
    & Waschbecken & Lichtenstein & \\
    & Dusche & Lichtenstein & \\
    & Bett & Lichtenstein & \\
    & Bettdecke & Lichtenstein & Kapitel \ref{bettdecke:ref1}\\
    & Polster & Lichtenstein & \\
    & Nachtkasten & Lichtenstein & \\
    & Nachtkastenlampe & Lichtenstein & \\
    & Kleiderkasten & Lichtenstein & \\
    & Steckdose & Lichtenstein & \\
    & Küche Arbeitsfläche & Lichtenstein & \\
    & Küche Schubladen & Lichtenstein & \\
    & Küche Herd & Lichtenstein & \\
    & Küche Backrohr & Lichtenstein & \\
    & Deckenleuchte & Lichtenstein & \\
    \hline
    \textbf{Mausoleum} &  &  & \\
    \hline
    & Stufen & Lichtenstein & \\
    & Säule & Lichtenstein & \\
    & Verzierung & Lichtenstein & \\
    & Tafel 1 & Lichtenstein & \\
    & Tafel 2 & Lichtenstein & \\
    & Altar & Lichtenstein & \\
    & Hauptgrab & Lichtenstein & \\
    & Hauptgrab klein & Lichtenstein & \\
    & Nebengrab & Lichtenstein & \\
    & Wandgrabstein & Lichtenstein & \\
    \hline
    \textbf{Hauptobjekte} &  &  & \\
    &  &  & \\
    \hline
    \textbf{Füllobjekte} &  &  & \\
    & Nebengrab 1 & Lichtenstein & \\
    & Nebengrab 2 & Lichtenstein & \\
    & Nebengrab 3 & Lichtenstein & \\
    & Nebengrab 4 & Lichtenstein & \\
    & Paracelsus Grab & Lichtenstein & Kapitel \ref{paracelsusgrab:ref1} \\
    & Fass & Lichtenstein & \\
    & Türstopper & Lichtenstein & \\
    \hline
\end{tabular}

\subsection{Paracelsus Grab}
\label{paracelsusgrab:ref1}
Das Grab des Paracelsus ist ein Modell, welches einem echten Objekt entspricht. Um ein originalgetreues Ergebnis zu erhalten,
modelliert man das Objekt von einem Foto nach. Dazu muss man ein Foto in Blender importieren. Um ein Foto zu importieren muss man sich im Object Mode befinden und \keys{N}
drücken. Im Menü sieht man dann das Feld \menu{Background Images} (Abbildung \ref{Paracelsus_Grab:image1}). Dort kann man dann den Pfad des Bildes angeben
und einstellen bei welchen Winkel zu welcher Achse sichtbar ist. In diesem Fall ist das Bild sichtbar, wenn man genau auf die Achsen X und Z schaut. Man kann außerdem die
Durchsichtigkeit (Opacity) des Bildes einstellen, in diesem Fall ist es zu 100\% sichtbar, damit man während dem modellieren die Linien im Bild noch gut erkennt.

\raggedbottom
\begin{figure}[H]
    \centering
    \includegraphics[width=.8\textwidth]{images/Paracelsus-Grab_Import-von-Bildern.png}
    \caption{Import von Bildern}
    \label{Paracelsus_Grab:image1}
\end{figure}

Nun modelliert man mit Formen das Grab nach und verschiebt im Edit Mode einzelne Punkte, so dass sie passen. Man muss allerdings darauf achten, dass in diesem Bild
(Abbildung \ref{Paracelsus_Grab:image2}) Verzerrungen aufgrund des Aufnahmewinkels des Fotos auftreten, deshalb sind einige Stellen nicht genau, sondern
durch eine Schätzung der Größen nachmodelliert worden.

\begin{figure}[H]
    \centering
    \includegraphics[width=.8\textwidth]{images/Paracelsus-Grab_Nachmodellierung.png}
    \caption{Paracelsus Grab Nachmodellierung (Lizenz zum Bild: \citep{paracelsusgrab:bild})}
    \label{Paracelsus_Grab:image2}
\end{figure}
\todo{Muss man Nutzungsrechte für das Bild einfügen?}

\subsubsection{Vase}
Um die Vase für das Paracelsus Grab zu modellieren, wurde eine Sphere benutzt und anschließend im Edit Mode verändert. Anschließend wurden die Verzierungen für die Vase gemacht.
Dazu wurde zwei Curves erstellt (Abbildung \ref{Paracelsus_Grab:image5}), die obere für die Einkerbungen und die untere für die Verzierung der Vase. Damit die Curves eine
Breite haben, muss man sie mit einem Bezier-Circle verbinden, der die Breite bestimmt. Nach dem Erstellen der Curve,
wurde sie mit \keys{\Alt + C} zu einem Mesh konvertiert, damit sie dann mit der Vase verbunden werden können.

\begin{figure}[H]
    \centering
    \includegraphics[width=.8\textwidth]{images/Paracelsus-Grab_Vase-Curve.png}
    \caption{Links: Vase mit den Verzierungen. Rechts: Becier-Circle}
    \label{Paracelsus_Grab:image5}
\end{figure}

\subsubsection{Gesicht}
Für die Modellierung des Gesichts, wurde ein Brush verwendet. Diese Methode braucht zwar viele Flächen, was die Performance des Spiels beeinträchtigen kann, war
aber notwendig, damit nachher noch eine andere Textur mit einer eigenen Bumpmap verwendet werden kann. Um das Gesicht zu erstellen wurde zuerst auf einer Seite der Säule,
die Fläche mit dem Shortcut \keys{W} \menu{Subdivide} (Edit Mode) unterteilt. Dann wurde im Sculpt Mode eine Schwarzweiß Textur, gefertigt aus dem ursprünglichen Bild des Paracelsus Grabes
eingefügt (Abbildung \ref{Paracelsus_Grab:image4}),
damit man mit dem Brush Höhen und Tiefen Zeichnen kann. Außerdem wurden die Werte Strenght und Radius angepasst. Nun wurde der Brush angewendet. Danach sind noch mit dem
Decimate-Modifikator so viele Flächen wie möglich entfernt worden, ohne dass das Aussehen der Statue stark beeinträchtigt wurde.

\begin{figure}[H]
    \centering
    \includegraphics[width=.8\textwidth]{images/Paracelsus-Grab_Brush.png}
    \caption{Einfügen der Schwarz/Weiß Textur (Lizenz zum ursprünglichen Bild der Schwarz/Weiß Textur: \citep{paracelsusgrab:bild})}
    \label{Paracelsus_Grab:image4}
\end{figure}

\subsubsection{Verzierungen}
Nachdem die Verzierung des Paracelsus Grabes (Abbildung \ref{Paracelsus_Grab:image3}) einer bereits vorhandenen Form sehr ähnlich sieht, nämlich der Verzierung des Mausoleums, wurde diese übernommen und
mit dem Sockel des Grabes zusammengefügt. Dazu wurde zuerst mit einem Würfel und dem Boolean Modifikator ein Stück aus dem Sockel herauseschnitten, damit
sich die Verzierung nicht mit dem Sockel überschneidet. Danach wurden der Sockel und die Verzierung mit \keys{\ctrl + J} zusammengefügt. Die Verzierung selber, wurde mit einer
Fläche modelliert, die dann im Edit Mode so angepasst wurde, dass sie aussieht wie eine Verzierung.
\begin{figure}[H]
    \centering
    \includegraphics[width=.8\textwidth]{images/Paracelsus-Grab_Verzierung.png}
    \caption{Verzierung des Paracelsus Grabs}
    \label{Paracelsus_Grab:image3}
\end{figure}

\subsection{Bettdecke}
\label{bettdecke:ref1}
Die Bettdecke sollte etwas zerknüllt aussehen. Deshalb wurde sie mit einer Simulation modelliert. Damit sich die Decke verformt, muss sie genug Flächen haben.
Dies kann man erreichen, indem man eine Plane erstellt, in den Edit Mode wechselt und mit \keys{W} \menu{Subdivide} meherere Flächen erzeugt. Anschließend wurde auf die Decke ein Solidify Modifikator
angewendet, damit die sie auch eine Dicke hat.
Dann wurden Forcefields hinzugefügt (Abbildung \ref{Bettdecke:image1}) um die Decke in Bewegung zu bringen. Damit sich die Decke aber auch wirklich bewegt muss man die Decke auswählen,
in den Physics Tab wechseln und Cloth auswählen. Jetzt verhält sich die Decke wie ein Stoffstück. Wenn man nun auf der Timeline auf Play drückt, bewegt sich die Bettdecke und verformt sich.
Wenn die Form passt, pausiert man die Simulation. Dann drückt man \keys{\Alt + C} und wählt \menu{Mesh from Curve/Meta/Surf/Text} aus, somit kann man das fertige Modell
frei bewegen.

\begin{figure}[H]
    \centering
    \includegraphics[width=.8\textwidth]{images/Bettdecke_Forcefields.png}
    \caption{Bettdecke mit Forcefields(orange)}
    \label{Bettdecke:image1}
\end{figure}
\flushbottom