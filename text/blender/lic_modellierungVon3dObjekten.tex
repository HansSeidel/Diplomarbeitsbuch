\section{Modellierung von 3D Objekten}
\label{sec:Modellierung_von_3D_Objekten}
Ein großer Teil in der Spielentwicklung, war das Modellieren. In diesem Kapitel wird gezeigt, wie man kompliziertere
Modelle macht und auf was man achten muss.

Es ist sehr wichtig, dass das Spiel mit einer guten Performance spielbar ist. Deswegen wurde beim Modellieren darauf geachtet,
dass es pro Modell so wenig Flächen wie möglich gibt \citep{unreal:modellierungVon3dObjekten_performance}, denn diese müssen vom Computer berechnet werden.
Darunter kann dann die Performance im Spiel leiden.
Zwei wesentliche Gebäude des Spieles, sind das Haus des Grabwächters und das Mausoleum inklusive aller enthaltenen Objekte.
Weiters gibt es einige Objekte die für die Story relevant sind und Objekte, die modelliert wurden um das Spiel zu füllen.

Auflistung der Objekte:

\begin{longtable}{|p{3.4cm}|p{3.4cm}|p{3.4cm}|p{3.4cm}|}
    \hline
    \endfirsthead
    \textbf{Zweck} & \textbf{Objekt} & \textbf{Entwickelt von} & \textbf{Anmerkung} \\
    \hline
    \endhead
    \textbf{Zweck} & \textbf{Objekt} & \textbf{Entwickelt von} & \textbf{Anmerkung} \\
    \hline
    \textbf{Haus} &  &  & \textit{Kapitel \ref{haus:ref1} \dq Haus\dq}\\
    \hline
    & Stufen & Lichtenstein & \textit{Kapitel \ref{Stufen_Im_Haus:Heading} \dq Stufen im Haus\dq}\\
    & Haustür & Lichtenstein & \\
    & Tür & Lichtenstein & \\
    & Fenster & Lichtenstein & \\
    & Sessel & Lichtenstein & \\
    & Toilette & Lichtenstein & \\
    & Waschbecken & Lichtenstein & \\
    & Dusche & Lichtenstein & \\
    & Bett & Lichtenstein & \\
    & Bettdecke & Lichtenstein & \textit{Kapitel \ref{bettdecke:ref1} \dq Bettdecke\dq}\\
    & Polster & Lichtenstein & \\
    & Nachtkasten & Lichtenstein & \\
    & Nachtkastenlampe & Lichtenstein & \\
    & Kleiderkasten & Lichtenstein & \\
    & Steckdose & Lichtenstein & \\
    & Küche Arbeitsfläche & Lichtenstein & \\
    & Küche Schubladen & Lichtenstein & \\
    & Küche Herd & Lichtenstein & \\
    & Küche Backrohr & Lichtenstein & \\
    & Deckenleuchte & Lichtenstein & \\
    \hline
    \textbf{Mausoleum} &  &  & \\
    \hline
    & Stufen & Lichtenstein & \\
    & Säule & Lichtenstein & \\
    & Verzierung & Lichtenstein & \\
    & Tafel 1 & Lichtenstein & \\
    & Tafel 2 & Lichtenstein & \\
    & Altar & Lichtenstein & \\
    & Hauptgrab & Lichtenstein & \\
    & Hauptgrab klein & Lichtenstein & \\
    & Nebengrab & Lichtenstein & \\
    & Wandgrabstein & Lichtenstein & \\
    \hline
    \textbf{Hauptobjekte} &  &  & \\
    &  &  & \\
    \hline
    \textbf{Füllobjekte} &  &  & \\
    & Nebengrab 1 & Lichtenstein & \\
    & Nebengrab 2 & Lichtenstein & \\
    & Nebengrab 3 & Lichtenstein & \\
    & Nebengrab 4 & Lichtenstein & \\
    & Paracelsus Grab & Lichtenstein & \textit{Kapitel \ref{paracelsusgrab:ref1} \dq Paracelsus Grab\dq} \\
    & Fass & Lichtenstein & \\
    & Türstopper & Lichtenstein & \\
    \hline
    \caption{Auflistung aller modellierten 3D-Objekte}
\end{longtable}

\subsection{Hauptgrab}
\label{Hauptgrab:Heading}
Das Hauptgrab, ist einem echten Grab, welches unser Team auf einem Friedhof gesehen hat, nachempfunden. Der Grabsockel besteht aus einem angepassten Würfel
mit einem Bevel Modifikator. Der Grabdeckel wurde ebenfalls aus einem Würfel modelliert und anschließend mit einem Bevel-Modifikator so bearbeitet, dass
er seine Einkerbungen bekommen hat. Der Griff ist, ein mit einem Boolean-Modifikator zusammengefügter Würfel und Torus\footnote{Der Torus ist ein Objekt in Blender. Seine Gestalt ähnelt der eines Rings.}.
Der Griff wurde mittel Array-Modifikator vervielfältigt und im passenden Winkel am Grabdeckel platziert. Anschließend wurden beide Objekte zusammengefügt.

\raggedbottom
\begin{figure}[H]
    \centering
    \includegraphics[width=.8\textwidth]{images/Hauptgrab_Grab.png}
    \caption{Hauptgrab}
    \label{Hauptgrab:Image1}
\end{figure}

Verwendete Modifikatoren: \textit{Kapitel \ref{Bevel:heading} \dq Bevel\dq}\verb+;+ \textit{Kapitel \ref{Boolean:heading} \dq Boolean\dq}\verb+;+ \textit{Kapitel \ref{Array:heading} \dq Array\dq}

\subsection{Paracelsus Grab}
\label{paracelsusgrab:ref1}
Das Grab des Paracelsus ist ein Modell, welches einem echten Objekt nachempfunden ist. Um ein möglichst originalgetreues Ergebnis zu erhalten,
modelliert man das Objekt von einem Foto nach. Dafür importiert man zuerst ein Foto in Blender (\textit{Abbildung \ref{Paracelsus_Grab:image1}}).
Man kann die Durchsichtikgeit des Bildes einstellen, bei welchen Winkel und auf welcher Achse das Bild sichtbar ist.
In diesem Fall wurde das Bild auf 100\% Sichtbarkeit eingestellt, damit man während dem Modellieren die Linien im Bild noch gut erkennt.
Außerdem, wurde es nur auf einer orthographischen Ansicht angezeigt, damit man es nur in der Ansicht sieht, in der es sinnvoll ist, die Maße nachzumodellieren.

\raggedbottom
\begin{figure}[H]
    \centering
    \includegraphics[width=.8\textwidth]{images/Paracelsus-Grab_Import-von-Bildern.png}
    \caption{Import von Bildern}
    \label{Paracelsus_Grab:image1}
\end{figure}

Nun modelliert man mit Formen das Grab nach und verschiebt im Edit Mode einzelne Punkte, so dass sie die gewünschte Form darstellen.
Man muss allerdings darauf achten, dass in diesem Bild (\textit{Abbildung \ref{Paracelsus_Grab:image2}}) Verzerrungen aufgrund des
Aufnahmewinkels des Fotos auftreten, deshalb sind einige Stellen nicht genau, sondern
durch eine Schätzung der Größen nachmodelliert worden.

\begin{figure}[H]
    \centering
    \includegraphics[width=.8\textwidth]{images/Paracelsus-Grab_Nachmodellierung.png}
    \caption{Paracelsus Grab Nachmodellierung (Lizenz zum Bild: \citep{paracelsusgrab:bild}, Autor: unbekannt - Wikimedia)}
    \label{Paracelsus_Grab:image2}
\end{figure}

\subsubsection{Vase}
Um die Vase für das Paracelsus Grab zu modellieren, wurde eine Sphere benutzt und anschließend im Edit Mode verändert. Anschließend wurden die Verzierungen für die Vase gemacht.
Dazu wurde zwei Curves erstellt (\textit{Abbildung \ref{Paracelsus_Grab:image5}}). Die obere für die Einkerbungen und die untere für die Verzierung der Vase. Damit die Curves eine
Dicke haben, muss man sie mit einem Bezier-Circle verbinden, der die Dicke der Bezier-Curve bestimmt.
Nach dem Erstellen der Curves, wurden diese zu einem Mesh konvertiert, damit sie anschließend mittels Boolean-Modifikator mit der Vase verbunden
bzw. ausgeschnitten werden können.

\begin{figure}[H]
    \centering
    \includegraphics[width=.8\textwidth]{images/Paracelsus-Grab_Vase-Curve.png}
    \caption{Links: Vase mit den Verzierungen. Rechts: Becier-Circle}
    \label{Paracelsus_Grab:image5}
\end{figure}

Verwendete Modifikatoren: \textit{Kapitel \ref{Curve:heading} \dq Curve\dq}\verb+;+ \textit{Kapitel \ref{Boolean:heading} \dq Boolean\dq}

\subsubsection{Gesicht}
Für die Modellierung des Gesichts wurde ein Brush\todo{Referenz zu Hans oder Fußnote} verwendet. Diese Methode braucht zwar viele Flächen, was die Performance des Spiels beeinträchtigen kann, war
aber notwendig, damit nachher noch eine andere Textur mit einer eigenen Bumpmap\todo{Referenz zu Hans} verwendet werden kann. Um das Gesicht zu erstellen wurde zuerst auf einer Seite der Säule,
die Fläche unterteilt. Dann wurde im Sculpt Mode eine Schwarzweiß Textur, welche aus dem ursprünglichen Bild des Paracelsus Grabes gefertigt wurde,
eingefügt (\textit{Abbildung \ref{Paracelsus_Grab:image4}, rotes Rechteck}).
Somit kann man mit dem Brush, Höhen und Tiefen Zeichnen. Außerdem wurden die Werte Strenght und Radius angepasst, damit die Brushstärke so hoch ist, das bei der Bearbeitung die Flächen weit genug verschoben werden
und die Brushfläche groß genug ist, damit das Gesicht auf das Grab hinauf passt. Danach wurde der Brush angewendet. Letztendlich sind noch mit dem Decimate Modifikator so viele Flächen wie
möglich entfernt worden, ohne dass das Aussehen der Statue stark beeinträchtigt wurde.

Verwendete Modifikatoren: \textit{Kapitel \ref{Decimate:heading} \dq Decimate\dq}
\begin{figure}[H]
    \centering
    \includegraphics[width=.8\textwidth]{images/Paracelsus-Grab_Brush.png}
    \caption{Einfügen der Schwarz/Weiß Textur (Lizenz zum ursprünglichen Bild der Schwarz/Weiß Textur: \citep{paracelsusgrab:bild})}
    \label{Paracelsus_Grab:image4}
\end{figure}

\subsubsection{Verzierungen}
Nachdem die Verzierung des Paracelsus Grabes (\textit{Abbildung \ref{Paracelsus_Grab:image3}}) einer bereits vorhandenen Form sehr ähnlich sieht, nämlich der Verzierung des Mausoleums, wurde diese übernommen und
mit dem Sockel des Grabes zusammengefügt. Dazu wurde zuerst mit einem Würfel und dem Boolean-Modifikator ein Stück aus dem Sockel herausgeschnitten, damit
sich die Verzierung nicht mit dem Sockel überschneidet. Danach wurden der Sockel und die Verzierung zusammengefügt. Die Verzierung selber, wurde mit einer
Fläche modelliert, die dann im Edit Mode so angepasst wurde, dass sie aussieht wie eine Verzierung.
\begin{figure}[H]
    \centering
    \includegraphics[width=.8\textwidth]{images/Paracelsus-Grab_Verzierung.png}
    \caption{Verzierung des Paracelsus Grabs}
    \label{Paracelsus_Grab:image3}
\end{figure}

Verwendete Modifikatoren: \textit{Kapitel \ref{Boolean:heading} \dq Boolean\dq}

\subsection{Stufen im Haus}
\label{Stufen_Im_Haus:Heading}
Um die Stufen (\textit{Abbildung \ref{Stufen_Im_Haus:Bild}}) zu modellieren, wurden echte Stufen abgemessen, um die Stufen mit einer möglichst realistischen Höhe und Tiefe modellieren zu können.
Die Stufen wurden mit einem Würfel modelliert, der mit einem Array Modifikator wiederholt wurde. Das Geländer und die Halterung des Handlaufes zur Wand
wurden mit den gleichen Verfahren wie die Stufen gefertigt, nur dass als Basisobjekt kein Würfel sondern ein Zylinder genommen wurde.
Die Stufen sind ein relativ einfaches Modell, jedoch ist es wichtig zu beachten, dass die Abstände sowohl im Modell selbst, als auch innerhalb des Hauses realistisch sind.

Um das Gelände von der Höhe an das Haus anzupassen, wurde es zum Schluss mit einem Boolean-Modifikator abgeschnitten.
Der Handlauf wurde ebenfalls mit einem Boolean Modifikator gekürzt. Das gesamte Geländer wurde mit Smooth Shading (\textit{Kapitel \ref{objectMode:smoothshading} \dq Smooth Shading\dq})
versehen, damit es runder aussieht.

\begin{figure}[H]
     \centering
     \includegraphics[width=.8\textwidth]{images/Stufen-Haus_Stufen.png}
     \caption{Stufen mit Geländer}
     \label{Stufen_Im_Haus:Bild}
\end{figure}

\subsection{Bettdecke}
\label{bettdecke:ref1}
Die Bettdecke sollte etwas zerknüllt aussehen. Deshalb wurde sie mit einer Simulation modelliert. Damit sich die Decke verformt, muss sie genug Flächen haben.
Dies kann man erreichen, indem man eine Plane\footnote{Eine Plane ist ein Objekt in Blender, welches einer Fläche gleicht.} erstellt, in den Edit Mode wechselt und diese in mehrere Flächen unterteilt.
Anschließend wurde auf die Decke ein Solidify-Modifikator angewendet, damit diese auch eine Dicke hat.
Dann wurden Forcefields hinzugefügt (\textit{Abbildung \ref{Bettdecke:image1}}) um die Decke in Bewegung zu versetzen.
Damit sich die Decke aber auch wirklich bewegt muss man sie auswählen,
in den Physics Tab wechseln und die Option Cloth auswählen, damit sich die Decke wie ein Stoffstück verhält.
Wenn man nun auf der Timeline auf Play drückt, bewegt und verformt sich die Bettdecke.
Wenn die Form passend ist pausiert man die Simulation. Dadurch, das die Decke animiert wurde, ist sie kein Objekt mehr, deshalb muss man sie
am Schluss noch in ein Objekt umformen, damit man die Decke frei bewegen kann, ohne das sie in die Form vor der Simulation zurückspringt.

\begin{figure}[H]
    \centering
    \includegraphics[width=.8\textwidth]{images/Bettdecke_Forcefields.png}
    \caption{Bettdecke mit Forcefields(orange)}
    \label{Bettdecke:image1}
\end{figure}
\flushbottom