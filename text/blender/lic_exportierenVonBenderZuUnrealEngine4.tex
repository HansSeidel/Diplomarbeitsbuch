\section{Exportieren von Blender zu Unreal Engine 4}
\label{Exportieren_von_Blender_zu_Unreal_Engine_4:ref1}
Um alle Objekte die in Blender modelliert wurden, richtig in der Unreal Engine anzeigen zu können, muss man ein paar
Einstellungen an den Objekten vornehmen und die Maße in der Datei verändern.

In der Datei selbst muss man im Property-Panel die Länge auf metrisch, den Winkel auf Grad und die Skalierung auf 0.01 stellen.
Das ist notwendig, damit die Objekte in der Unreal Engine die richtige Skalierung haben. Damit man diese Einstellungen nicht bei
jeder Datei neu einstellen muss, kann man die Einstellungen als Startup File speichern.

Bei allen Objekten, die man exportieren möchte, muss man darauf achten, dass das Objekt auf den Location\footnote{Gibt die Position des 3D-Objekts im Koordinatensystem an} Koordinaten
den Wert 0 hat, damit das Objekt in der Unreal Engine seinen Origin in der Mitte des Objektes hat und um diesen gedreht, verschoben und skaliert werden kann.
Um das Objekt entsprechend auf 0 zu verschieben wird der Origin des Objektes mittig im Objekt gesetzt. Das ist außerdem sehr nützlich, falls der Origin wo anders ist und man das
Objekt nicht um seine eigene Achse drehen, verschieben und skalieren kann. Anschließend setzt man die Location Koordinaten auf 0.

Zum Schluss ist es wichtig, welches Dateiformat benutzt wird, um die Daten so zu speichern, dass beide Programme die enthaltene
Information lesen können. Die Wahl des richtigen Formates wird in den Unterkapiteln abgedeckt.

\subsection{3D-Modelle}
Damit in die Unreal Engine alles korrekt importiert werden kann, muss man die 3D-Modelle in Dateien der Endung .fbx exportieren.
Damit man nicht etwas falsches exportiert, ist es empfehlenswert nur ausgewählte Objekte zu exportieren. Dazu wählt man alle
Objekte aus die man exportieren möchte, exportiert sie und wählt vor dem Bestätigen die Option Selected Objects aus.
Um Smooth Shading (\textit{Kapitel \ref{objectMode:smoothshading} \dq Smooth Shading\dq})
auch zu exportieren, muss man den Wert Face auswählen.

\subsection{Simulationen}
\label{Simulation_Heading}
Bei Simulationen muss man die Dateien als Alembic Datei exportieren. Dazu wählt man wieder alle Objekte aus, die man exportieren möchte und
wählt diesmal beim Exportieren das Dateiformat Alembic (.abc) aus. Bei den Exporteinstellungen kann man vor dem Exportieren den Wert Triangulate auswählen,
es kann nämlich der Fall eintreten, dass die Dateien in der Unreal Engine sonst nicht richtig erkannt werden.