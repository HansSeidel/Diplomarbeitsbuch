%Author: Tobias Röhrer
\section{Intro}
Für eine Diplomarbeit an der HTL3R ist es Angabe, eine Website zu erstellen.
Deswegen, und weil wir nach außen hin gut repräsentiert werden wollen, haben wir eine Website aufgebaut.
Diese ist sowohl auf Deutsch als auch auf Englisch verfügbar, damit wir möglichst viele Menschen erreichen können.
\section{CMS}
Anfangs war geplant, die Website mithilfe des Frameworks Angular.js zu erstellen.
Jedoch stellte sich nach Gesprächen mit Lehrpersonen heraus, dass dies zu viel Arbeit sein würde und wir unser Hauptaugenmerk auf das Spiel selbst richten sollten.
Deswegen suchten wir uns ein CMS, ein Content Management System aus.
Ein CMS ist ein fertiges System, welches auf einem Webserver läuft, und es dem User ermöglicht eine Website zu erstellen, ohne dafür HTML, CSS, JavaScript oder eine serverseitige Programmiersprache wie z.B. PHP oder Java können zu müssen.

Für uns standen konkret folgende CMS zur Auswahl:
\begin{itemize}
    \item Wordpress
    \item TYPO3
    \item Joomla
    \item Drupal
\end{itemize}
Da wir uns davor noch nie mit Joomla oder Drupal auseinandergesetzt hatten, vielen dies weg.
TYPO3 lernten wir in der Schule, jedoch war dieses CMS für unseren Anwendungsfall nicht geeignet, da wir nur eine einfache, statische Website brauchten.
Mit TYPO3 ist der Aufwand ohne viel Inhalt bereits recht groß, jedoch steigt er nicht stark, wenn man sehr viel Inhalt hat.
Bei Wordpress ist der Anfangsaufwand recht klein, jedoch steigt er mit dem Inhalt umso stärker an.
Da ich persönlich mit Wordpress schon Erfahrung hatte und
\section{Plugins}
\section{Host}
\section{Inhalte}
    \subsection{Diplomarbeit}
    \subsection{Spiele}
    \subsection{Sonstiges}
\section{Design}
    \subsection{Template}
    \subsection{Abänderungen}