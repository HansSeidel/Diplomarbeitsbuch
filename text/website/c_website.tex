%Author: Tobias Röhrer
Für eine Diplomarbeit an der HTL3R ist es Aufgabe, eine Website zu erstellen.
Deswegen, und weil wir uns nach außen hin auch gut repräsentieren wollen, haben wir eine Website aufgebaut.
Diese ist sowohl auf Deutsch als auch auf Englisch verfügbar, damit wir möglichst viele Menschen erreichen können.

\section{CMS}
Anfangs war geplant, die Website mithilfe des Frameworks Angular.js zu erstellen.
Jedoch stellte sich nach Gesprächen mit Lehrpersonen heraus, dass dies zu viel Arbeit sein würde und wir unser Hauptaugenmerk auf das Spiel selbst richten sollten.
Deswegen suchten wir uns ein CMS, ein Content Management System aus.
Ein CMS ist ein fertiges System, welches auf einem Webserver läuft, und es dem User ermöglicht eine Website zu erstellen, ohne dafür HTML, CSS, JavaScript oder eine serverseitige Programmiersprache wie z.B. PHP oder Java können oder einsetzen zu müssen. \citep{website:CMSdefinition}
Für uns standen konkret folgende CMS zur Auswahl:
\begin{itemize}
    \item WordPress \citep{website:WordPress}
    \item Joomla! \citep{website:Joomla}
    \item Drupal \citep{website:Drupal}
    \item TYPO3 \citep{website:typo3}
\end{itemize}
Da wir uns davor noch nie mit Joomla! oder Drupal auseinandergesetzt hatten, standen diese nicht zur Diskussion.
Mit TYPO3 umzugehen lernten wir in der Schule, jedoch erst nachdem wir die Website erstellen mussten.
Daher, und weil ich bereits Erfahrung mit WordPress gesammelt hatte, entschieden wir uns WordPress zu verwenden.

\section{Plugins}
Um die Funktionen von WordPress noch zu erweitern und zu personalisieren, gibt es die Möglichkeit, Plugins zu installieren. Nun folgt eine Auflistung der installierten Plugins, deren Funktionalität und der Grund der Installation:
\begin{itemize}
    \item Akeeba Backup for WordPress (Akeeba Ltd) \citep{website:Akeeba}
    Dieses Plugin ermöglicht ein einfaches Erstellen von WordPress-Backups. Man hat die Möglichkeit alle Einstellungen als Template zu speichern, um für alle folgenden Backups nur noch einen Klick zu benötigen. Die dadurch entstandene Datei kann man dann vom Server des Host-Providers herunterladen, um sie lokal zu speichern.
    Die erstellte Backup-Datei kann man dann verwenden, um eine neue WordPress Installation, welche bereits alle Inhalte, Plugins und andere Daten beinhaltet, zu installieren.
    Eine Anleitung hierfür gibt es auf der Homepage von Akeeba. \citep{website:AkeebaInstructions}

    Installiert habe ich dieses Plugin, um regelmäßige Backups unserer Website zu erstellen, sodass ich im Fall eines unwiderruflichen Fehlers die Seite wiederherstellen kann, ohne erneut von Grund auf beginnen zu müssen.
    \item All In One SEO Pack (Michael Torbert) \citep{website:AllInOneSEO}

    Mit diesem Plugin lässt sich Search Engine Optimization (Das Optimieren einer Website für höhere Relevanz bei Suchmaschinen wie etwa Google) einfach direkt auf der Website durchführen.

    Dieses Plugin hilft uns dabei, unsere Seite so zu optimieren, dass sie von Google als relevanter angesehen wird und deswegen weiter oben bei einer Google-Suche angezeigt wird. Das hilft uns von potenziellen Interessenten eher gefunden zu werden.

    \item Cookie Notice (dFactory) \citep{website:CookieNotice}

    Dieses Plugin lässt den User mit geringem Aufwand einen modifizierbaren Cookie-Notice-Banner erstellen. Ein solcher Banner erscheint beim Erstbesuch unserer Website am unteren Rand des Bildschirms, um den Benutzer zu informieren, dass die Seite Cookies benützt.

    Nötig ist es für unsere Website, da das Plugin „WP Statistics“ Cookies auf dem Endgerät des Besuchers speichert und wir der DSGVO entsprechen müssen.

    \item Polylang (Frédéric Demarle) \citep{website:Polylang}

    Mit Polylang lässt sich eine WordPress Seite einfach zweisprachig gestalten. Man hat die Möglichkeit, jeder erstellten Seite eine Sprache und ein anderssprachiges Gegenstück zuzuweisen. Als Besucher hat man dann die Möglichkeit auf der Website zwischen mehreren Sprachen zu wechseln.

    Verwendet haben wir dieses Plugin, um unsere Seite sowohl auf Deutsch als auch auf Englisch anzubieten. Es gibt zwar Plugins, mit deren Hilfe man Sprachen übersetzen kann, jedoch habe ich diese Tätigkeit allein bewältigen können.

    \item Simple Custom CSS and JS (SilkyPress.com) \citep{website:SimpleCustomCSSJS}

    Dieses Plugin ermöglicht ein einfaches Einfügen von eigenem CSS (Cascading Style Sheet) und JS (Java Script).

    Da wir das Theme stilistisch etwas anpassen wollten, brauchten wir eine Möglichkeit, unser eigenes CSS einzubauen.

    \item WP Statistics (VeronaLabs) \citep{website:WPStats}

    WP Statistics bereitet diverse Statistiken über die eigene WordPress Seite wie die Anzahl der Besucher, deren Regionalität, Linkreferenzen und ähnliches vor.

    Installiert haben wir dieses Plugin hauptsächlich aus Interesse daran, wer, wie oft und von wo unsere Seite besucht. Diese Information lässt sich natürlich auch benützen, um die Website an die Besucher anzupassen.
\end{itemize}
\section{Host}
Um unsere Website für die Öffentlichkeit zugänglich zu machen, entschieden wir uns einen Domainserver zu mieten. Als Anbieter wählten wir das österreichische Unternehmen world4you.com aus. Das Paket „Domainserver Aktion 2018“ beinhaltete PHP Support sowie 5 MySQL Datenbanken, beides wichtige Komponenten, um WordPress installieren zu können, und kostete 2.99€ pro Monat für ein Jahr. Diese Merkmale machten es zur optimalen Wahl als unsere Hostinglösung. \citep{website:Host}
\section{Design}
Das Design der Website ist essenziell, um bei Besuchern einen guten Eindruck zu hinterlassen und sie zum Wiederkehren zu bewegen. Deswegen haben wir uns hierfür einige Gedanken gemacht.
\subsection{Theme}
Das gewählte Theme ist „Modern“ von „WebMan Design“. Das Theme ist out-of-the-box responsive und beinhaltet ein Hero-Image, welches für unsere Anwendung, ein Horror-Spiel, optimal angepasst werden kann. \citep{website:Theme}
\begin{figure}[H]
    \centering
    \includegraphics[width=.9\textwidth]{roh/WS_ModernThemeStock.PNG}
    \caption{Das originale Modern-Theme.}
    \label{WS:ModerThemeStock}
\end{figure}

\subsection{Abänderungen}
Um das Theme unseren Vorstellungen anzupassen, habe ich einige Änderungen vorgenommen. Diese beinhalten unter anderem:
\begin{itemize}
    \item Die Änderung der Farbe von selektiertem Text zu unserem Violett.
    \item Den Footer in der Höhe verkleinern, sodass er nicht zu viel Platz einnimmt.
    \item Den Titel verstecken, da wir als Titel unser Logo verwenden, welches als Bild importiert ist.
    \item Das Hintergrundbild einfügen und etwas abdunkeln, damit der Text gut lesbar bleibt.
    \item Den Hintergrund der weißen Textboxen durscheinend gestalten, sodass das Hintergrundbild immer noch sichtbar ist.
    \item Die Portrait-Bilder schöner anzeigen.
    \item Die Farben des Themes wurden denen des Corporate Designs angepasst.
\end{itemize}
\begin{figure}[H]
    \centering
    \includegraphics[width=.9\textwidth]{roh/WS_ModernThemeModified.PNG}
    \caption{Das modifizierte Modern-Theme.}
    \label{WS:ModerThemeModified}
\end{figure}