%<<<<<<< HEAD
\section{Kurzfassung}
\section{Abstract}
\section{Ziele}
%=======
%\section[Kurzfassung]
%\section[Abstract]
%\section[Ziele]
%Die Bezeichnung der Ziele wird an die Unterteilung der Themes angepasst. Zum Beispiel ist die Bezeichnung des ersten Ziels im Bereich „Game“: „Ziel-HA1 1“ -> Ziel-(Haupt) (Theme A) (Epic 1) (Ziel 1) Also ist „Ziel-HA1 1“ das erste Haupt-Ziel, des Themes „Game“ des Epics „Story“.
%\subsection[Pflichtziele]
%1.1.1	 	H-Ziele Theme A – Game (Story)
%Ziel-HA1 1	Skript erstellen
%
%		Es ist ein Skript erstellt. Das Skript beinhaltet:
%		 *******************************
%		 *  AUFZÄHLUNGSPUNKTE - ANFANG *
%		 *******************************
%•	„Cutscenes“
%Das Intro
%Das Outro (Begräbnis des Hauptcharakters)
%Die Credits
%•	Die Map
%o	Skizze des Grundlayouts
%Der Bereich, den der User gehen kann, sowie geographische Höhen und Tiefen der Spielwelt
%o	Skizze des computergesteuerten Gegenspielers
%Basierend auf der Skizze des Grundlayouts, der Weg, welcher von dem computergesteuerten Gegenspieler begangen werden kann
%o	Skizze der Objekte.
%Basierend auf der Skizze des Grundlayouts die Positionen der Objekte, die der User finden muss.
%o	Skizze von Häusern und Spawns (Startpunkte).
%Basierend auf der Skizze des Grundlayouts, alle Häuser und Gebäude, die es auf der Map gibt, sowie der Ort an welchem der User startet.
%•	Eine Skizze des computergesteuerten Gegenspielers
%•	UML Graphiken und Skizzen der Menüs
%•	Skizzen der Feedback Interfaces.
%Alle In-Game Informationen für den User, die er braucht um das Ziel des Spiels zu verstehen.
%•	Story-Informationen
%Es sind je fünf Berichte und Zettel erstellt, welche im Spiel gefunden werden können. Auf denen sind Geschehnisse und Informationen der Vorgeschichte des Hauptcharakters enthalten 
%
%		 *******************************
%		 *  AUFZÄHLUNGSPUNKTE - ENDE   *
%		 *******************************
%
%1.1.2	H-Ziele Theme A – Game (Models)
%Ziel-HA2 1	Hauptcharakter – Modell
%
%Es ist der Hauptcharakter einfach modelliert. Die Arme und Beine des Hauptcharakters werden genau und detailgetreu modelliert. (Der Hauptcharakter wird von dem User gesteuert. Das Spiel ist in Ego-Perspektive umgesetzt -> Man sieht nur die Arme und Beine des Hauptcharakters.)
%
%	Ziel-HA2 2 	Computergesteuerter Gegenspieler – Modell
%
%Der computergesteuerte Gegenspieler ist genau und detailgetreu modelliert.
%
%	Ziel-HA2 3	Grund-Map Modellierung
%
%Die Map (Spielwelt) ist nach der Skizze: „Skizze des Grundlayouts“ modelliert. Außerdem sind die Begrenzungen der Map modelliert.
%
%	Ziel-HA2 4	Map-Hauptgebäude Modellierung
%
%Es ist das Mausoleum des Hauptcharakters und das Haus des computergesteuerten Grabwächters modelliert. Die Modellierung ist genau und detailgetreu umgesetzt.
%
%	Ziel-HA2 5	Hauptobjekte Modellierung
%
%			Es sind die folgenden Objekte für das Spiel modelliert:
%		 *******************************
%		 *  AUFZÄHLUNGSPUNKTE - ANFANG *
%		 *******************************
%•	Ein Seil
%•	3. Stück Holz (Übereinander liegend)
%•	Ein Zementsack
%•	Eine Zementtrommel
%•	Ein Grabstein
%•	Eine Zement-Kanone
%•	Kerzen & Rosen
%		 *******************************
%		 *  AUFZÄHLUNGSPUNKTE - ENDE   *
%		 *******************************
%	Ziel-HA2 6	Selbstbegräbnis Modellierung
%
%Es ist der Mechanismus, welchen sich der Hauptcharakter baut um sich selbst zu begraben fertig modelliert. Bei der Modellierung wurde beachtet, dass in dem Outro der Mechanismus animiert ist und dieser nur aus den Objekten besteht, welche in dem Ziel-HA2 6 erstellt wurden.
%
%
%
%1.1.3	H-Ziele Theme A – Game (Interfaces)
%Ziel-HA4 1	Interfaces Layout
%
%Das Layout der folgenden Menüs ist nach den Skizzen des Ziels-HA1 1 umgesetzt. Es sind Skizzen der folgenden Menüs ausgearbeitet:
%		 *******************************
%		 *  AUFZÄHLUNGSPUNKTE - ANFANG *
%		 *******************************
%•	Start-Menü
%•	In-Game-Menü
%•	Optionen-Menü
%•	In-Game Interfaces
%		 *******************************
%		 *  AUFZÄHLUNGSPUNKTE - ENDE   *
%		 *******************************
%Dieses Ziel gilt auch für alle begonnen Skizzen und Erweiterungen des Ziels-OA1 1.
%1.1.4	H-Ziele Theme A – Game (Level Design)
%Ziel-HA5 1 	Modelle platzieren
%
%Es sind alle erstellten Modelle aus den Zielen-HA2 X (Models) in der Map (Spielwelt) platziert. Dieses Ziel gilt auch für alle zusätzlich erstellten Modelle aus den Zielen-OA2 X (Models).
%
%Ziel-HA5 2	Umsetzung der „Cutscenes“
%
%Die „Cutscenes“ sind nach der Beschreibung in dem Skript aus Ziel-HA1 1 und unter Beachtung der Atmosphäre aus dem Ziel-OA1 2 in der Map animiert. Dieses Hauptziel gilt auch für alle begonnenen zusätzlichen Ereignisse des optionalen Ziels-OA1 1.
%(Das Ziel-OA1 2 ist optional. Falls dieses Ziel nicht umgesetzt wurde, wird die Beachtung dieses Ziels aus dem Hauptziel HA5 2 nicht als Pflicht angesehen. Die übrige Beschreibung des Ziels wird nicht beeinträchtigt).
%1.1.5	H-Ziele Theme A – Game (Functionality)
%Ziel-HA7 1	Hauptcharakter – Riggs
%
%Es sind folgende Riggs (Animationsabläufe) für den Hauptcharakter fertiggestellt:
%		 *******************************
%		 *  TABELLE - ANFANG           *
%		 *******************************
%Rig	Beschreibung
%HC_Tumble	Leichte Bewegung der Arme
%HC_Walk	Eine Geh-Animation. Animation der Arme und Beine.
%
%		 *******************************
%		 *  TABELLE - ENDE             *
%		 *******************************
%
%
%
%
% 
%Ziel-HA7 2	Computergesteuerter Nebencharakter – Riggs
%
%Es sind folgende Riggs (Animationsabläufe) für den computergesteuerten Nebencharakter fertiggestellt:
%		 *******************************
%		 *  TABELLE - ANFANG           *
%		 *******************************
%Rig	Beschreibung
%GW_Tumble	Leichte Bewegung der Arme und des Oberkörpers.
%Bewegung des Kopfs (Umschauen)
%GW_Walk	Eine Geh-Animation. Animation der Arme und Beine.
%Bewegung des Kopfs (Umschauen)
%GW_Run 	Eine Rennanimation.
%
%		 *******************************
%		 *  TABELLE - ENDE             *
%		 *******************************
%
%
%
%
%
%
%
%
%Ziel-HA7 3	Hauptcharakter – Funktionalität
%
%Die aus dem Ziel-HA7 1 beschriebenen Riggs können durch folgende Aktionen von dem User gesteuert oder ausgelöst werden:
%		 *******************************
%		 *  TABELLE - ANFANG           *
%		 *******************************
%Rig	Aktion
%HC_Tumble	Kein Input und keine Bewegung
%HC_Walk	„W“ – Taste für geradeaus gehen
%„A“ – Taste für nach links gehen
%„S“ – Taste für zurück gehen
%„D“ – Taste für nach rechts gehen
%
%		 *******************************
%		 *  TABELLE - ENDE             *
%		 *******************************
%
%
%
%
%
%
%
%Ziel-HA7 4	Computergesteuerter Nebencharakter – Funktionalität
%
%Die aus dem Ziel-HA7 2 beschriebenen Riggs können durch folgende Aktionen ausgelöst werden:
%		 *******************************
%		 *  TABELLE - ANFANG           *
%		 *******************************
%Rig	Aktion
%GW_Tumble	Kein Input und kein spezieller Auslöser. Wechselt sich mit GW_Walk ab.
%GW_Walk	Kein Input und kein spezieller Auslöser. Wechselt sich mit GW_Tumble ab.
%GW_Run 	Wenn der Hauptcharakter in dem Sichtfeld des Nebencharakters ist.
%
%		 *******************************
%		 *  TABELLE - ENDE             *
%		 *******************************
%
%
%
%
%
%
%
%Ziel-HA7 5	Hauptobjekte – Funktionalität
%
%Die Hauptobjekte können von dem User mit der „E“-Taste aufgesammelt werden.  Wenn der User bei dem Ursprungsort ist, wird das jeweilige Objekt abgelegt.
% 
%Ziel-HA7 6 	„Interfaces“ – Funktionalität
%
%Die komplette Funktionalität der Interfaces wurde nach den UMLs des Ziels-HA1 1 umgesetzt. Dieses Ziel gilt auch für alle angefangenen Erweiterungen nach den UMLs des Ziels Ziel-OA1 1.
%
%
%Ziel-HA7 7	„Cutscene“ – Einbindung
%
%Die „Cutscenes“ sind nach der Beschreibung in dem Skript aus Ziel-HA1 1 und unter Beachtung der Atmosphäre aus dem Ziel-OA1 2 in dem Spiel eingebunden (programmiert). Dieses Hauptziel gilt auch für alle begonnenen zusätzlichen Ereignisse des optionalen Ziels-OA1 1.
%
%1.1.6	H-Ziele Theme B – Organisation & Marketing (Design)
%Ziel-HB1 1	Entwurf des Logos – G.O.I
%
%Es ist ein Logo für G.O.I entworfen. Dieses Logo wird Teil des Corporate Design Dokuments. Das Logo ist eine Vektorgraphik.
%
%Ziel-HB1 2	Corporate Design – G.O.I
%
%Es ist ein ausführliches und umfangreiches Corporate Design Dokument für das Spiel „G.O.I.“ entworfen.
%
%
%
%1.1.7	H-Ziele Theme B – Organisation & Marketing (Projektpräsentation)
%Ziel-HB2 1	Kommunikation
%
%Es ist die kontinuierliche Kommunikation mit allen team-externen Stakeholder organisiert. Das inkludiert Terminsuche für Meetings, Deadlines für Berichte und Protokolle. Dieses Hauptziel inkludiert auch die Kommunikation mit allen begonnen Kooperationen externer Auftragnehmer oder Sponsoren.
%
%
%	Ziel-HB2 2 	Rechtliche Absicherung
%
%Es ist sichergestellt, dass das gesamte Projekt, alle verwendeten Programme und alle verwendeten Texturen, Sounds, Modelle und Materialien für einen kommerziellen Nutzen freie Lizenz haben oder eine kommerzielle Lizenz erworben ist. Das gesamte Projekt mit Inhalten darf veröffentlicht und verkauft werden.
%
%	Ziel-HB2 3 	Erstellung einer Webseite
%
%Es ist eine Webseite zur Präsentation unseres Projekts nach den Corporate Designs (Ziel-HB1 1 & Ziel-HB1 3) umgesetzt. Die Webseite beinhaltet neben dem vorgegebenen Content einer Diplomarbeit:
%		 *******************************
%		 *  AUFZÄHLUNGSPUNKTE - ANFANG *
%		 *******************************
%•	Den Content-Punkt: „Games“
%Der Content-Punkt "Games" beinhaltet Informationen über das Spiel, das im Laufe unserer Diplomarbeit erstellt wird. Es ist eine Beschreibung der Story vorhanden, sowie Informationen über Kooperationspartner und über den aktuellen Stand der Entwicklung.
%
%•	Den Content-Punkt: „About Us“
%Der Content-Punkt „About Us” beinhaltet Informationen über die Teammitglieder sowie Informationen über alle Kooperations- Partner, Sponsoren und die Schule HTL3R.
%		 *******************************
%		 *  AUFZÄHLUNGSPUNKTE - ENDE   *
%		 *******************************
%
%
%1.1.8	H-Ziele Theme B – Organisation & Marketing (Promotion)
%Ziel-HB3 1	Zielgruppendefinition
%
%Es ist eine Zielgruppendefinition für die Zielgruppe des Spiels entworfen.
%
%Ziel-HB3 2	Marktrecherche / Marktanalyse / Konkurrenzanalyse
%
%Es ist eine Marktrecherche für Spiele und deren Trends im Genre „Horror“ entworfen. Diese wird analysiert und ausgewertet.
%
%Ziel-HB3 3	Marketingziele
%
%Es ist sowohl ein Dokument für die Marketingziele des internen Marketings erstellt, als auch eines für das externe Marketing.
%
%Ziel-HB3 4	Marketingstrategie
%
%Es ist ein Marketingstrategien Dokument erstellt, in dem eine Strategie entworfen ist die verschiedenen Faktoren, bezogen auf das Spiel berücksichtigt und die zu erstellenden und nutzenden Marketinginstrumente (Flyer, Promotion Video, …) vorgibt. Dieses Dokument bezieht sich auch auf die Marketing-Strategien der Schulevents (z.B. Tag der offenen Tür)
%
%Ziel-HB3 5	Marketingkonzept
%
%Es ist ein Dokument mit dem Marketingkonzept erstellt.
%In diesem wird die Ausgangssituation des Projektes, was mit dem Projekt/Spiel erreicht werden soll, der Plan/die Strategie wie die Ziele erreicht werden sollen und das Resultat analysiert.
%
%Das Marketingkonzept beinhaltet folgende Punkte:
%		 *******************************
%		 *  AUFZÄHLUNGSPUNKTE - ANFANG *
%		 *******************************
%•	Zielgruppendefinition
%•	Marktrecherche/Marktanalyse/Konkurrenzanalyse
%•	Marketingziele
%•	Marketingstrategie
%		 *******************************
%		 *  AUFZÄHLUNGSPUNKTE - ENDE   *
%		 *******************************
%
%Ziel-HB3 6	Umsetzung des Marketingkonzepts
%
%Das davor geplante Marketingkonzept wird umgesetzt.
%
%\subsection[Optionale-Ziele]
%1.2.1	O-Ziele Theme A – Game (Story)
%Ziel-OA1 1 	Skript – Erweiterung
%
%		Das erstellte Skript beinhaltet zusätzlich:
%•	Ereignisse
%Es bleibt der Kreativität und Ausdauer der Entwickler überlassen, eine beliebige Anzahl an Ereignissen dem Skript und in weiterer Folge dem Spiel hinzuzufügen.
%Ereignisse beschreiben:
%		 *******************************
%		 *  AUFZÄHLUNGSPUNKTE - ANFANG *
%		 *******************************
%o	Atmosphärische Unterstützung
%	Vögel die fliegen
%	Umfallende Bäume
%	Plötzliches Schreigeräusch
%	usw.
%o	Zusätzliche „Cutscenes“
%	Zusätzlich Animation und Geschehnisse, die für das Spiel animiert sind, mit beschränkter oder gar keiner Kontrolle des Users über den Hauptcharakter
%•	Die Map
%o	Skizze der computergesteuerten Gegenspielern.
%Basierend auf der Skizze des Grundlayouts, die Wege, welche von den computergesteuerten Gegenspielern begangen werden können
%o	Skizze von zusätzlichen Gebäuden und Ereignissen
%Basierend auf der Skizze des Grundlayouts, Skizzen von den Positionen der zusätzlichen Gebäude und Ereignisse
%o	Skizze der Umwelt.
%Basierend auf der Skizze des Grundlayouts die Bereiche der verschiedenen Umwelten (Z.B. Offenes Feld, Wald, Gräber)
%•	Eine Skizze des Hauptcharakters
%•	Weitere computergesteuerte Gegenspieler
%•	Zusätzliche Skizzen der Feedback Interfaces
%Dieser Punkt beinhaltet zusätzliches Feedback für den User, welches er sich anzeigen lassen kann oder nicht.
%Zum Beispiel: „Frame Rate anzeigen“
%•	Story-Informationen-Erweiterung
%Es sind zusätzliche Zettel und zusätzliche Berichte digital erstellt, die Geschehnisse und Informationen der Vorgeschichte des Hauptcharakters enthalten
%		 *******************************
%		 *  AUFZÄHLUNGSPUNKTE - ENDE   *
%		 *******************************
%Ziel-OA1 2 	„Perfect Walkthrough“
%
%Es ist ein Dokument fertiggestellt, das aus der Sicht des Users in Form einer Erzählung, alles Wahrgenommene und den Weg, der zum Ziel des Spiels führt, beschreibt. Außerdem ist die Umwelt und die Atmosphäre beschrieben (inklusive dem Licht Verhältnis und den auffälligsten Geräuschen oder der Musik).
%Dieses Dokument ist eine Unterstützung der Entwicklung des Spiels.
%1.2.2	O-Ziele Theme A – Game (Models)
%Ziel-OA2 1	Modelle – Erweiterung
%
%Es sind zusätzlich Modelle jeglicher Art modelliert, die das erstellte Spiel verfeinern und die Spielwelt in einem angemessenen Rahmen füllen. Bei dem Erstellen dieser Modelle soll zur Unterstützung auf das Skript, sowie auf das Ziel-OA1 2 zurückgegriffen werden. Diese Modelle können unter der Kontrolle des Ziel-HB2 2 auch von externen Personen entworfen sein und nach Erlangen einer kommerziellen Lizenz in dem Spiel eingebaut werden. In diese Kategorie fallen Modelle wie Häuser, Ruinen, Mausoleen, Grabstätten, Keller, Bäume, Sträucher, Gräser, Steine, Felsen, Tiere, Statuen, Brunnen, Grabsteine, Gräber, Laternen, Fahrzeuge, Tonnen, Schubkarren, Werkzeuge, usw.
%
%Ziel-OA2 2	Charaktere – Erweiterung
%
%Es werden weitere Charaktere modelliert. Diese Charaktere werden zur Statisterie des Spiels verwendet und haben keine große Rolle oder Funktion in dem Spiel. Diese Charaktere sind immer Mitglieder eines Ereignisses. Die Modellierung wird minimalistisch gehalten und gut genug für eine kurze Betrachtung erstellt.
%1.2.3	O-Ziele Theme A – Game (Music & Sounds)
%Ziel-OA3 1	Erstellung eines Audio-Skripts
%
%Es wird ein Audio-Skript erstellt, welches die gesamte Musik zu jedem Zeitpunkt des erstellten Spiels beschreibt. Außerdem werden alle Sounds, die durch Aktionen oder Situationen ausgelöst werden in Zusammenhang mit der Aktion oder Situation beschrieben. Das Audio-Skript beinhaltet auch alles Gesprochene, sowie eine Beschreibung der Stimme.
%
%Beispiele:
%
%Das Spiel wird gestartet. Der User befindet sich in dem Start-Menü:
%Es ist eine leise und sanfte Hintergrund-Musik (z.B von einer Geige). Außerdem wird das Start-Menü durch ein Windgeräusch unterstützt; leiser und leichter Wind. Alle 30 Sekunden (+- 10) hört man ein Knacksen, das hinter einem positioniert ist.
%
%			Der User bewegt sich über Gras:
%Man hört Schritte unterhalb des Hauptcharakters, die wie Schritte auf Gras klingen
%
%			Der User bewegt sich nicht:
%				Man hört den Hauptcharakter leise atmen.
%
%	Ziel-OA3 2 	Erstellung von Sample-Video-Dateien
%
%Für alle aus dem Ziel-OA3 1 beschriebenen Situationen wird ein Beispiel Video der Aktion in dem Spiel aufgenommen.
%
%	Ziel-OA3 3	Einbinden der Audio-Files
%
%Im Falle einer Kooperation mit „Cosmix Studios“ werden die von Cosmix Studios erstellten Audio-Files in das Spiel eingebunden.
%
%1.2.4	O-Ziele Theme A – Game (Texturing)
%	Ziel-OA6 1	Erstellung oder Erwerb von Texturen
%
%Es werden 2D-Texturen und Normal-Texturen für alle erstellten Modelle unter strenger Beachtung des Ziels-OHB2 2 erworben oder erstellt.
%
%	Ziel-OA6 2	„Materials“
%
%Es sind für alle erstellten Modelle „Materials“ erstellt, mit den dazugehörigen Texturen.
%1.2.5	O-Ziele Theme A – Game (Functionality)
%Ziel-OA7 1	Riggs – Erweiterung
%
%Es sind zusätzlich Riggs jeglicher Art fertiggestellt, die das erstellte Spiel verfeinern und die Bewegungen des Hauptcharakters und des computergesteuerten Nebencharakters in einem angemessenen Rahmen füllen.
%
%Ziel-OA7 2	Funktionalität – Erweiterung
%
%Die Funktionalitäten der aus dem Ziele-OA7 1 beschriebenen Riggs sind umgesetzt.
%
%Ziel-OA7 3 	Zusätzliche Riggs und Animationen
%
%Es sind zusätzliche frei definierbare Riggs und Animationen zur Unterstützung der Usability oder des Realismus eingebaut.
%1.2.6	O-Ziele Theme A – Game (Shading & Color-Grading)
%Ziel-OA8 1	Lightning
%
%Das gesamte Spiel hat mehrere Lichtquellen, die die Atmosphäre des Spiels nach dem Ziel-OA1 2 übermitteln.
%
%Ziel-OA8 2 	„Post Production“
%
%Alle Farben und Lichter in dem Spiel werden mittels Color-Grading Methoden nochmals der Atmosphäre aus dem Ziel-OA1 2 angepasst und verfeinert. Reflektionen, sowie Schatten und Bump-Map Texturen sind überprüft und verfeinert.
%
%1.2.7	O-Ziele Theme B – Organisation & Marketing (Organisation)
%	Ziel-OB2 1	Steam – Verkauf
%
%Das Spiel ist auf der Spiele-Plattform „Steam“ für den festgelegten Preis erhältlich. Es ist geprüft, ob und wie Steuern gezahlt werden müssen, falls die Einnahmen pro Person über dem Existenzminimum liegen.
%
%1.2.8	O-Ziele Theme B – Organisation & Marketing (Cooperation)
%Ziel-OB4 1	Sponsoring
%
%Die Kosten des Projekts und die Finanzierung der Musik und der Sounds des Spiels sind von mindestens einem Sponsor gedeckt.
%
%
%
%	Ziel-OB4 2	Vertonung
%
%Es wurde mindestens ein externes Audio-Unternehmen engagiert das Spiel zu vertonen. Zur Unterstützung sind dem Unternehmen bei Zusage die Ausarbeitung der Ziele:
%Ziel-OA3 1	Erstellung eines Audio-Skripts
%Ziel-OA3 2 	Erstellung von Sample-Video-Dateien
%gesendet.
%
%\subsection[Nicht-Ziele]
%Ziel-N1 	Audio
%
%Die Musik und Sounds sind von dem Diplomarbeitsteam umgesetzt.
%
%	Ziel-N2	Webspace
%
%Die erstellte Webseite wird auf einen Webserver der HTL3 Rennweg gehalten.
%
%	Ziel-N3	Privacy
%
%Die Spielinhalte und der Code des Spiels sind öffentlich ersichtlich.
%
%	Ziel-N4	Unternehmen
%
%			Es ist ein Unternehmen gestartet und angemeldet.
%>>>>>>> sei
