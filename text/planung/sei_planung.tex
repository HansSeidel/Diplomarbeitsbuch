\usepackage{natbib}

\section{Vorwort}

Ein qualitativ hochwertiges Spiel im Rahmen dieser Diplomarbeit und in dem Umfang der geplanten Geschichte
zu entwicklen, ist nicht möglich.
Daher hat sich das Diplomprojekt: "Graveyard of Immortals" zum Ziel gesetzt die ersten Bausteine für die Entwicklung
eines Horror-Spiels zu erstellen.
Ein weiterer Teil der Diplomarbeit war es, das Projekt mittels Marketing-Aktionen bekannt zu machen.
Auf diesen Teil der Diplomarbeit wird nicht eingeganen, da das zuständige Team-Mitglied verstorben ist.

\section{Kurzfassung}

Dieses Buch befasst sich mit der Umsetzung eines grundlegenden 3D-Spieles.
Das Spiel fällt unter das Genre Horror.

Zu Beginn des Spieles befindet sich der Spieler vor einem Mausoleum.
In der ersten Szene, hat der Spieler noch keine Kontrolle über den Spielcharakter.
Der Charakter geht die Stufen des Mausoleums hinauf und betrachtet eine Stantafel vor dem Eingang.
Es sind sieben Objekte darauf graviert, die der Spieler finden muss.
Im Laufe des Spiels wird dem Spieler bewusst, dass er sich in einer Art Alptraum befindet.
Um diesem Alptraum zu entkommen muss er die Objekte finden und sich am Ende selbst begraben.

Die Umsetzung des Spiels wurde mit dem 3D-Programme: "Blender" und der Spieleentwicklungssoftware: "Unreal Engine" umgesetzt.
In den weiteren Kapiteln werden diese Programme genauer unter die Lupe genommen.

\section{Abstract}
This book describes the realisation of a basic 3D-game.
The genre of the game belongs to horror.

At the beginning of the game the player is in front of a mausoleum.
During the first sequence the player does not have any controls above the gamecharacter.
The game character goes up the stairs and starts looking at a stone plate  in front of the entrance.
There are seven objects engraved inside the stone plate which the player must find.
While playing the game the play realises he is inside something like a nightmare.
In order to escape the nightmare the player must find the objects to bury himself.

The realisation of the game was made with the 3D-Program: "Blender" and the game-engine: "Unreal Engine".
The next chapters will give a more specifically look at these programs.


\section{Ziele}
Die Aufgabe die sich das Diplomarbeitsteam gesetzt hat war es, ein umfangreiches Wissen in dem
Bereich der 3D-Modellierung und Spiele-Entwicklung anzueignen.
Um das zu erreichen haben wir folgende Punkte bearbeitet:

\begin{itemize}
    \item  Modellierung von Objekten und Charakteren
\end{itemize}
Für die Umsetzung der Objekte und Charaktere wurde die 3D-Software Blender verwendet.
Die Grundprinzipien der Modellierung, sowie tiefgreifende Tricks und Techniken von Blender wurden erlernt.

\begin{itemize}
    \item  Erstellung von Materialien und Texturen
\end{itemize}
Es wurde viel Wissen und Erfahrung im Bereich von Materialien und Texturen erreicht.
Die verschiedenen Methoden ein Objekt zu Texturieren wurden erlernt.
Um einen teifgreifenden Einblick in dieses Thema zu bekommen wurden alle Texturen von dem Diplomarbeitsteam selbst entwickelt.

\begin{itemize}
    \item  Animation und Simulationen
\end{itemize}
Durch die Entwicklung der Animationen hat sich das Diplomarbeitsteam ein umfangreiches Wissen über Animationen und
Simulationen in einer 3D-Welt angeeignet.

\begin{itemize}
    \item Umsetzung der Funktionalität
\end{itemize}
Im Laufe des Projektes wurde der Umgang mit dem Visual-Scripting System der Unreal-Engine erlernt.
Zusätzlich wurden Erfahrungen mit KI-Algorithmen gesammelt.

\begin{itemize}
    \item Level-Design
\end{itemize}
Das Gestalten der Spielwelt und die Einstellungen der Licht Lichtverhältniss zeigte dem Diplomarbeitsteam, wie die
abschließenden Feinarbeiten an einem Spiel aussehen.
