Die in diesem Projekt vollbrachte Arbeiten, haben einen groben Überblick der wichtigsten Aspekte der Spieleentwicklung hervogehoben.
Es wurden verschiedene Ansätze vorgestellt und einige Probleme aufgezeigt, die bei der Entwicklung eines Spiels auftreten können.
Die gesammelten Erfahrungen werden den Teammitgliedern in Zukunft bei 3D-Projekten weiterhelfen und einige Prozesse beschleunigen und verbessern.
Aus der Sicht des Management hätte man einige Arbeitsprozesse besser aufteilen können und anders koordieren; beziehungsweise früher Integrations- und
Modul-Tests durchführen sollen. Das Zusammenfügen der einzelnen Arbeiten war der komplexeste Prozess, bei welchem die meisten Fehler aufgetreten sind.
Außerdem wurden einige essentielle Arbeitsschritte übersehen, die im späteren Verlauf, bereits erstellte Arbeiten unnütz gemacht haben.
Unter diesen Arbeitsschritten ist die Dezimierung von zu vielen Flächen, die Entwicklung von UV-Layouts vor dem Export und die Feinarbeiten der Charaktere, bevor sie geriggt werden.

Als Fazit kann man sagen, dass das Projekt ein Erfolg war und die erstellten Arbeiten einen guten Anfang für die Weiterentwicklung darstellen.