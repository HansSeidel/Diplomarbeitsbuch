
Thats why -- the translated text ,,Kurzfassung`` (this should be
a translation).

Im englischen Abstract sollte inhaltlich das Gleiche stehen wie in
der deutschen Kurzfassung. Versuchen Sie daher, die Kurzfassung präzise
umzusetzen, ohne aber dabei Wort für Wort zu übersetzen. Beachten
Sie bei der Übersetzung, dass gewisse Redewendungen aus dem Deutschen
im Englischen kein Pendant haben oder völlig anders formuliert werden
müssen und dass die Satzstellung im Englischen sich (bekanntlich)
vom Deutschen stark unterscheidet. Es empfiehlt sich übrigens – auch
bei höchstem Vertrauen in die persönlichen Englischkenntnisse – eine
kundige Person für das „proof reading“ zu engagieren. Die richtige
Übersetzung für „Diplomarbeit“ ist übrigens schlicht thesis, allenfalls
„diploma thesis“ oder „Master’s thesis“, auf keinen Fall aber „diploma
work“ oder gar „dissertation“\citep{hagenberg}.

Wichtig ist wegen des Abteilens ein \code{\textbackslash{}begin\{english\}}
bzw. \code{\textbackslash{}selectlanguage\{ngerman\}}.

Dieses Buch befasst sich mit der Umsetzung eines grundlegenden 3D-Spieles. Das geplante Spiel fällt unter das Gerne Horro.
Zu Beginn befindet sich der Spieler vor einem Mausoleum. In der ersten Szene, hat der Spieler noch keine Kontrolle über den Spielcharakter.
Der Charakter die Stufen des Mausoleums hinauf und betrachtet eine Stantafel vor dem Eingang. Es sind sieben Objekte darauf graviert, die der Spieler finden muss.
Im Laufe des Spiels wird dem Spieler bewusst, dass er sich in einer Art Alptraum befindet. Um diesem Alptraum zu entkommen muss er die Objekte finden und sich am Ende selbst begraben.

Die Umsetzung des Spiels wurde mit dem 3D-Programme: "Blender" und der Spieleentwicklungssoftware: "Unreal Engine" umgesetzt.
In den weiteren Kapiteln werden diese Tools und das Management genauer unter die Lupe genommen.
