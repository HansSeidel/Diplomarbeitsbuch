Our diploma-project is a horror game. We want to provide tension, fear, and jump scares in high quality. An immersive experience – in a safe environment.

The in-game environment, though, is not safe at all. You will have to navigate through graves, mausolea and other buildings to acquire important objects. Don’t think that it will be easy, as you are not alone.

The adventure is accompanied by an extensive fictional story, which will be explained in-game via various notes. In addition to this, you will be able to find out more about important real-life personas to expand or refresh your knowledge of the world.

%<<<<<<< HEAD
%=======
%Wichtig ist wegen des Abteilens ein \code{\textbackslash{}begin\{english\}}
%bzw. \code{\textbackslash{}selectlanguage\{ngerman\}}.
%
%Dieses Buch befasst sich mit der Umsetzung eines grundlegenden 3D-Spieles. Das geplante Spiel fällt unter das Gerne Horro.
%Zu Beginn befindet sich der Spieler vor einem Mausoleum. In der ersten Szene, hat der Spieler noch keine Kontrolle über den Spielcharakter.
%Der Charakter die Stufen des Mausoleums hinauf und betrachtet eine Stantafel vor dem Eingang. Es sind sieben Objekte darauf graviert, die der Spieler finden muss.
%Im Laufe des Spiels wird dem Spieler bewusst, dass er sich in einer Art Alptraum befindet. Um diesem Alptraum zu entkommen muss er die Objekte finden und sich am Ende selbst begraben.
%
%Die Umsetzung des Spiels wurde mit dem 3D-Programme: "Blender" und der Spieleentwicklungssoftware: "Unreal Engine" umgesetzt.
%In den weiteren Kapiteln werden diese Tools und das Management genauer unter die Lupe genommen.
%>>>>>>> sei
