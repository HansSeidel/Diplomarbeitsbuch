%<<<<<<< HEAD
Bei unserer Diplomarbeit handelt es sich um ein Horror-Spiel. Wir wollen ein Gruselerlebnis, Spannung, Angst und Schreckmomente in hoher Qualität darbieten. Ein Erlebnis, in welches man sich hineinfallen lassen kann – und das in einer sicheren Umgebung.

Die Umwelt im Spiel ist allerdings gar nicht sicher. Dort muss man sich durch Gräber, Mausoleen und Häuser schleichen, um wichtige Objekte zu sammeln. Doch das wird kein Spaziergang, denn man ist nicht allein.

Das Abenteuer wird begleitet von einer umfangreichen fiktiven Geschichte, welche im Spielverlauf mithilfe von Notizzetteln herausgefunden werden kann. Zusätzlich gibt es auch viele interessante Informationen über bekannte Menschen. Auf Informationstafeln kann man mehr über unsere Geschichte erfahren, oder sein vorhandenes Wissen auffrischen.
%=======
%\todo{Benni fragen, was er von dem Vorwort hält}
%\todo{Benni - passt}
%Diese Arbeit befasst sich mit der Vorbereitung zur Umsetzung und Vermarktung eines Computerspiels.
%Das Diplomprojekt: "Graveyard of Immortals" hat sich zum Ziel gesetzt, das erlernte Wissen im Rahmen der 3D-Entwicklung, des Managements und des Marketings zu nutzen und zu erweitern.
%Der Fokus ist auf Qualität und Erfahrung gesetzt. In den folgenden Kapiteln wird genaure auf die verwendeten Programme und die Arbeitschritte zur Entwicklung des Spiels eingegangen.
%
%
%\todo{Satz mit Benni/STM/MTJ abklären und verbessern}
%\todo{Benni - eher nicht, wie is das dann bei den Teilnehmern wenn der Satz mit hineinkommt? (Weil es komisch aussieht wenn keine Seiten von Tobs dabei sind.}
%Auf die Vermarktung des Spiels wird nur minimalistisch eingeganen, da das zuständige Team-Mitglied verstorben ist.
%
%----Info: Es beginnt erst ab hier die Kurzfassung
%
%Dieses Buch befasst sich mit der Umsetzung eines grundlegenden 3D-Spieles. Das geplante Spiel fällt unter das Gerne Horro.
%Zu Beginn befindet sich der Spieler vor einem Mausoleum. In der ersten Szene, hat der Spieler noch keine Kontrolle über den Spielcharakter.
%Der Charakter die Stufen des Mausoleums hinauf und betrachtet eine Stantafel vor dem Eingang. Es sind sieben Objekte darauf graviert, die der Spieler finden muss.
%Im Laufe des Spiels wird dem Spieler bewusst, dass er sich in einer Art Alptraum befindet. Um diesem Alptraum zu entkommen muss er die Objekte finden und sich am Ende selbst begraben.
%
%Die Umsetzung des Spiels wurde mit dem 3D-Programme: "Blender" und der Spieleentwicklungssoftware: "Unreal Engine" umgesetzt.
%In den weiteren Kapiteln werden diese Tools und das Management genauer unter die Lupe genommen.
%
%\todo Mehrwert???????????????????? :-(
%>>>>>>> sei
