\todo{Benni fragen, was er von dem Vorwort hält}
Diese Arbeit befasst sich mit der Vorbereitung zur Umsetzung und Vermarktung eines Computerspiels.
Das Diplomprojekt: "Graveyard of Immortals" hat sich zum Ziel gesetzt, das erlernte Wissen im Rahmen der 3D-Entwicklung, des Managements und des Marketings zu nutzen und zu erweitern.
Der Fokus ist auf Qualität und Erfahrung gesetzt. In den folgenden Kapiteln wird genaure auf die verwendeten Programme und die Arbeitschritte zur Entwicklung des Spiels eingegangen.

\todo{Satz mit Benni/STM/MTJ abklären und verbessern}
Auf die Vermarktung des Spiels wird nur minimalistisch eingeganen, da das zuständige Team-Mitglied verstorben ist.

----Info: Es beginnt erst ab hier die Kurzfassung

Dieses Buch befasst sich mit der Umsetzung eines grundlegenden 3D-Spieles. Das geplante Spiel fällt unter das Gerne Horro.
Zu Beginn befindet sich der Spieler vor einem Mausoleum. In der ersten Szene, hat der Spieler noch keine Kontrolle über den Spielcharakter.
Der Charakter die Stufen des Mausoleums hinauf und betrachtet eine Stantafel vor dem Eingang. Es sind sieben Objekte darauf graviert, die der Spieler finden muss.
Im Laufe des Spiels wird dem Spieler bewusst, dass er sich in einer Art Alptraum befindet. Um diesem Alptraum zu entkommen muss er die Objekte finden und sich am Ende selbst begraben.

Die Umsetzung des Spiels wurde mit dem 3D-Programme: "Blender" und der Spieleentwicklungssoftware: "Unreal Engine" umgesetzt.
In den weiteren Kapiteln werden diese Tools und das Management genauer unter die Lupe genommen.

\todo Mehrwert???????????????????? :-(
